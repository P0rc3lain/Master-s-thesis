\chapter*{Abstract}
Rosnąca moc obliczeniowa komponentów sprzętowych sprawiła, że trójwymiarowa grafika komputerowa może być stosowana na większości urządzeń wyposażonych w wyświetlacz.
W ramach pracy zdecydowano się przeanalizować platformy ekosystemu Apple pod kątem technologii służących do jej generowania.
Pomimo szerokiej gamy dostępnych rozwiązań dostrzeżono niszę, którą wypełnić mogłaby natywna biblioteka graficzna.
Skoncentrowano się na produkcjach z branży gier, przystosowanych do krótkiego czasu pojedynczej rozgrywki i aplikacjach użytkowych.
Specjalizacja w obrębie ograniczonych przypadków użycia i platformy pozwoliła znacząco zwiększyć przystępność programistycznego API na tle alternatywnych rozwiązań.
W niniejszej pracy udokumentowano projekt, implementację oraz przedstawiono rozumowanie, którym kierowano się w procesie rozwoju.
Weryfikacja możliwości i ocena użyteczności oparte zostały o refleksje powstałe podczas tworzenia gry szachowej, wykorzystującej uprzednio stworzoną biblioteki.
\\
\par
The increasing computational power of hardware components has enabled the use of 3D computer graphics on most devices equipped with a display. 
As part of this work, the decision was made to analyze the platforms within the Apple ecosystem in terms of the technologies used for generating graphics. 
Despite a wide range of available solutions, a niche was identified that could be filled by a native graphics library. 
The focus was on game productions tailored for short gameplay sessions and utility applications. 
Specializing within limited use cases and platforms significantly increased the accessibility of the programming API compared to alternative solutions. 
This paper documents the project, implementation, and presents the reasoning that guided the development process. 
The capabilities were verified and the usability was evaluated based on reflections that emerged during the creation of a chess game utilizing the previously developed library.

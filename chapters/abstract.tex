\chapter*{Abstract}
Rosnąca moc obliczeniowa komponentów sprzętowych sprawiła, że trójwymiarowa grafika komputerowa może być zastosowana na większości urządzeń wyposażonych w wyświetlacz.
W ramach pracy zdecydowano się przeanalizować platformy ekosystemu Apple pod kątem technologii służących do jej generowania.
Pomimo szerokiej gamy dostępnych rozwiązań dostrzeżono niszę, którą wypełnić mogłaby natywna biblioteka graficzna.
Nastawiona powinna być ona na niewielkie produkcje, do których zaliczyć można gry przystosowane do krótkiego czasu rozgrywki oraz aplikacje użytkowe.
W pracy zaprojektowano, zaimplementowano oraz przedstawiono osiągniętych wyników.
Pragnąc zadbać o możliwie obiektywną ocenę działań posłużono się biblioteką do stworzenia gry będącej prezentacją działania technologii.
\\
\par
Increasing computing power of hardware components resulted in ability of using three-dimensional graphics on a variety of devices equipped with a display.
As part of the thesis research, while analyzing Apple ecosystem's, the platforms were examined to find technologies used for rendering purposes.
Despite the wide range of available solutions, it was noted that there is a gap in the market that could be filled by a native library.
The product should be tailored to facilitate small games and utility applications.
The process included design and implementation which were summarized in the detailed description of the internals and reasoning for decision-making.
In order to prove usability a chess app using the library was showcased.

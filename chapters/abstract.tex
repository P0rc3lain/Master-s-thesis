\chapter*{Abstract}
Rosnąca moc obliczeniowa komponentów sprzętowych sprawiła, że trójwymiarowa grafika komputerowa zastosowana może być na większości urządzeń.
W ramach pracy zdecydowano się przeanalizować platformy ekosystemu Apple pod kątem technologii służących do jej generowania.
Pomimo szerokiej gamy dostępnych rozwiązań dostrzeżono niszę, którą wypełnić mogłaby natywna biblioteka graficzna.
Ukierunkowana powinna być ona na niewielkie produkcje do których zaliczyć można proste gry oraz aplikacje użytkowe.
Podjęto się projektu, implementacji oraz przedstawienia osiągniętych wyników.
Pragnąc zadbać o możliwie obiektywną ocenę działań posłużono się biblioteką do stworzenia gry będącej prezentacją działania technologii.
\linebreak
\par
Raising computing power of hardware components resulted in ability of using three-dimensional graphics on variety of devices.
As part of the thesis' research, Apple ecosystem's platforms were analyzed, being examined to find technologies used for rendering purposes.
Despite wide range of available solutions, it was noted that there is a gap in the market that could be filled by a native library.
The product should be tailored to facilitate small games and utility applications.
The process included design and implementation what was summarized in detailed description of the internals and reasoning for decision making.
In order to proof usability, supporting in the same time conclusions that were made a chess app using the library was showcased.

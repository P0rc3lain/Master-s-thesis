%
%  Copyright © 2022 Mateusz Stompór. All rights reserved.
%

\chapter{Prezentacja działania}
Ostatni rozdział pracy poświęcony jest przedstawieniu wykorzystania biblioteki w prawdziwym zastosowaniu.
Decyzja pozwoli sprawdzić stabilność, wydajność oraz przystępność interfejsu.
Będzie także okazją na dostrzeżenie potencjalnych problemów oraz nakreślenie kierunku przyszłego rozwoju.
\section{Aplikacja demonstracyjna}
W trakcie rozwijania projektu tworzono wiele scen poglądowych, które w zamyśle miały weryfikować i pokazywać możliwości.
Były one jednak często ściśle skrojone pod daną funkcjonalność i nie dawały pełnego obrazu w kwestii jakości integracji pomiędzy aplikacją, a biblioteką.
Zdecydowano, że reprezentatywny przykład powinien być dziełem kompletnym pod względem funkcjonalności i realizować przypadek użycia zadany podczas projektowania silnika.
Z tego względu zdecydowano się na tworzenie aplikacji, będącej grą.
Jako cel wybrano szachy.
\figh{images/chess/gameplay.png}
     {Gra szachowa stworzona na potrzeby projektu}
     {fig:chess}
     {\linewidth}
\subsection{Forma}
Aplikacja stworzona została z przeznaczeniem na platformę macOS.
Oparto ją o interfejs okienkowy.
Dystrybuowana jest natywnie dla systemu, za pośrednictwem paczki w formacie \textit{.app}.
\subsection{Logika gry}
Aplikacja zawiera pełną implementację logiki szachowej z uwzględnieniem \textit{el passant}, \textit{castlingu} oraz promocji figur.
Użytkownicy mogą anulować wybór figury i zmienić ją już po wybraniu.
Nie poddano restrykcji czasowej także czasu przeznaczonego na turę zawodnika.
\subsection{Sterowanie}
Interakcja z aplikacją oparta jest o wykorzystanie myszki i klawiatury.
Za pomocą klawiszy \textit{A} oraz \textit{D} użytkownik może dokonywać rotacji kamery zmieniając pozycję z której obserwuje scenę.
Selekcja figury dokonywana jest za pomocą kursora myszy.
Aplikacja sygnalizuje wybór unosząc figurę nieznacznie ponad obszar planszy.
Wówczas użytkownik może wybrać docelowe pole na które chce poruszyć lub kliknąć poza obszar mapy, aby zmienić wybrany pionek.
\subsection{Ruch}
Chcąc wykorzystać możliwie dużo z możliwości biblioteki zadbano o animacje.
Tak jak zaznaczone wcześniej wybór figury jest sygnalizowany zmianą pozycji, która następuje za pomocą płynnego ruch.
Podobnie w sytuacji gdy wybór figury jest anulowany.
Przejście z jednego pola do innego wykorzystuje podobny efekt.
Figura unoszona jest ponad plansze, przenoszona do docelowego pola, a następnie opuszczana.

%
%  Copyright © 2022 Mateusz Stompór. All rights reserved.
%

%
%  Copyright © 2022 Mateusz Stompór. All rights reserved.
%

\chapter{Omówienie projektu}
Niniejszy rozdział poświęcony jest opisowi biblioteki graficznej stworzonej na potrzeby pracy.
Odzwierciedla ona wizję autora na rozwiązanie przygotowane z myślą o produkcji niewielkich gier
oraz wizualizacji wkomponowanych w aplikacje użytkowe.
\par
Przybliżenie rozpocznie się od przedstawienia zastosowanych technologii oraz struktury plików.
W dalszej części poruszona zostanie kwestia architektury, by możliwie było przejście do głównych modułów służących do zarządzania sceną oraz renderowania.
Finalnie przedstawiona zostanie dokumentacja oraz działania podjęte na rzecz uzyskania kodu możliwie wolnego od błędów.

%
%  Copyright © 2022 Mateusz Stompór. All rights reserved.
%

\section{Technologie}
Sekcja przedstawienia warstwę technologiczną na którą składają się wspierane platformy, użyte języki programowania oraz biblioteki zewnętrzne.
%
%  Copyright © 2022 Mateusz Stompór. All rights reserved.
%

\subsection{Języki programowania}
Kod źródłowy biblioteki składa się z ponad dziesięciu tysięcy linii kodu.
Obsługa grafiki, przez wzgląd na użycie programów cieniujących, narzuca konieczność skorzystania z więcej niż jednego języka programowania.
W projekcie wykorzystano trzy \longpause Swift, C oraz MSL.
\par
Większość biblioteki \longpause 80\% \longpause oparta jest o język Swift w wersji piątej.
Stworzona jest w nim cała zawartość przeznaczona do wykonywania na CPU.
Za jego pośrednictwem udostępniony jest interfejs zewnętrzny.
\par
Shadery, będące odpowiedzialne za 10\% udział, wykorzystują Metal Shading Language.
W zależności o zastosowania są to programy cieniujące dla wierzchołków, fragmentów i obliczeniowe.
Jest to część wewnętrzna, poza zasięgiem użytkownika.
\par
GPU daję możliwość skorzystania z buforów danych oraz tekstur.
Muszą one dowiązane być przy użyciu indeksu, współdzielonego pomiędzy kartą graficzną a procesorem.
Z tego względu, by uniknąć pomyłek, w bibliotece znajduje się pewna ilość plików nagłówkowych zawierająca stałe w języku C.
Zdefiniowano tam indeksy dowiązań dla poszczególnych programów.
Wyeksportowane są one do języka Swift za pomocą modułu.
MSL kompatybilny jest z C, dzięki czemu możliwe jest zapewnienie pomostu pomiędzy obiema technologiami.
Podobny mechanizm zastosowano dla struktur danych współdzielonych pomiędzy GPU oraz CPU.
Stanowi to drugi, ostatni pomocniczy moduł.
%
%  Copyright © 2022 Mateusz Stompór. All rights reserved.
%

\subsection{Biblioteki zewnętrzne}
Wykonanie biblioteki wiązało się z koniecznością użycia szeregu rozwiązań autorstwa osób trzecich.
Wpłynęły one na prędkość rozwoju, jednak zachowały pomocniczy charakter i nie realizowały celu zbieżnego z tym postawionym w pracy.
Starano się, aby możliwie duża ilość pochodziła wprost od producenta.
Decyzję motywowano chęcią zapewnienia bezpieczeństwa, stabilności i wsparcia na wypadek wewnętrznego błędu.
Wszystkie są także integralną częścią systemu operacyjnego, więc nie narzuca to konieczności korzystania z zewnętrznych źródeł w celu ich uzyskania.
\par
Na listę kompletnych rozwiązań składają się:
\paragraph*{simd}\longpause skorzystano ze struktur wspomagających algebrę liniową takich jak kwaterniony, wektory, macierze i najczęstsze operacje na nich wykonywane.
\paragraph*{AppKit, UIKit}\longpause użyte zostały w celu zapewnienia użytkownikowi widoku skonfigurowanego uprzednio na potrzeby generowania grafiki za pośrednictwem silnika.
\paragraph*{Metal}\longpause moduł udostępnił zbiór funkcji do interakcji z kartą graficzną.
Za jego pomocą kolejkowane są zadanie do wykonania na GPU i pośredniczy w uzyskiwaniu zasobów.
\paragraph*{Combine}\longpause funkcjonalność przeznaczona jest do programowania reaktywnego. 
Skorzystano z niej na potrzeby odświeżania danych w miejscach gdzie zachodziły zależności pomiędzy stanami obiektów, a także by oddać użytkownikowi możliwość obserwowania zmian z poziomu interfejsu zewnętrznego.
\paragraph*{MetalKit}\longpause biblioteka Metal tworzona była przybierając możliwie jak najmniejszą formę. 
Część funkcji niebędących koniecznymi do sprawowania jej działania przeniesiona została do osobnego modułu gromadzącego dodatkowe narzędzia.
Użyto między innymi metod to wczytywania tekstur z plików binarnych czy uprzednio przygotowanego widoku, mogącego być bezpośrednio osadzonym w aplikacji.
\paragraph*{ModelIO}\longpause zakres pracy nie objął obsługi poszczególnych formatów plików zawierających dane na temat sceny.
Użyto gotowego rozwiązania opublikowanego przez producenta. 
Obsługuje ono szeroki zakres wspieranych typów, między innymi \textit{fbx}, \textit{obj}, \textit{usdz}, \textit{dae}.
Działa na zasadzie fasady i konwertuje binarną reprezentację na jednolity format, dostępny z poziomu języka programowania.
\paragraph*{CoreGraphics}\longpause biblioteka zapewnia stosunkowo niskopoziomowy dostęp do operowania na grafice dwuwymiarowej.
Jej użycie ograniczało się do wykorzystania struktur koniecznych do wyznaczenia obszaru renderowania czy punktu, z którym użytkownik dokonał interakcji za pomocą wskaźnika lub dotyku.
\paragraph*{MetalPerformanceShaders}\longpause Apple opracowało w formie gotowych programów cieniujących listę najczęściej używanych efektów graficznych.
Zdecydowano się nie duplikować istniejącej funkcjonalności i posłużyć nimi do wsparcia efektów postprocesowych.
\paragraph*{Swift Standard Library}\longpause poza szeregiem modułów przygotowanych przez producenta specjalnie dla ekosystemu Apple skorzystano także z biblioteki standardowej języka Swift.
Wewnątrz niej znajdują się definicje typów podstawowych jak liczby całkowite czy zmiennoprzecinkowe a także struktury takie jak tablica czy słownik.
\paragraph*{Foundation}\longpause funkcje systemowe nie wchodzą w skład funkcjonalności dostępnych wraz z językiem. 
Obsługa daty, wyrażeń regularnych czy dostępu do plików wewnątrz archiwum biblioteki to wybrane przykłady z których skorzystano.
\paragraph*{XCTest}\longpause część kodu weryfikowana jest pod kątem poprawności za pomocą przypadków testowych.
Walidację wyników oparto o funkcje zaimplementowane w stworzonym pod tym kątem produkcie.
Dodatkowo, integracja z IDE XCode zapewniła wgląd do procentowego pokrycia kodu źródłowego przez testy.
\subsection{Wspierane platformy}
Projekt przeznaczony jest wyłącznie na urządzenia z ekosystemu Apple.
Składa się na to macOS, iOS oraz tvOS, w wersjach 11.5, 15.0, 16.1 odpowiednia dla każdego wariantu.
\par
Ze wsparcia zdecydowano się wyłączyć inteligentne zegarki z systemem watchOS.
Producent nie udostępnia dla nich interfejsu OpenGL, ani Metal.
Nie oznacza to, że nie ma możliwości wsparcia trójwymiarowej grafiki.
Producent posiada kilka aplikacji, które ją używają.
Dla programistów jednak udostępniono tylko i wyłącznie moduły będące w stanie generować dwuwymiarowe kompozycje.
W teorii za pomocą odpowiednich przekształceń istnieje możliwość uzyskania trójwymiaru.
Jednak przez wzgląd na silne użycie CPU wydajność takiego rozwiązania byłaby niezadowalająca.
Bezpośrednio wpłynęło to na decyzję o odrzuceniu platformy spośród docelowych systemów.
\section{Struktura Projektu}
Starano się, aby projekt był możliwie jak najprzystępniejszy do analizy.
Z tego powodu dokonano jego ustrukturyzowania.
Przekłada się to na organizację plików wewnątrz katalogów.
Ponadto, ustandaryzowano styl i podejście programistyczne zastosowane w kodzie źródłowym.

\subsection{Organizacja źródeł}
Rozwijanie wielu funkcjonalności projektu wiązało się z eksperymentowaniem.
Kod modyfikowano w celu wykonania prototypu i zweryfikowania możliwości.
Wpływało to przejściowo na stabilność, niekiedy zmiany ostatecznie zostawały wycofywane.
Manualna obsługa scenariusza była pracochłonna.
Narzędziem pomagającym w tej sytuacji był system kontroli wersji.
Zdecydowano skorzystać się z wiodącego na rynku - git.
Dzięki niemu zadbano o stabilność głównej gałęzi rozwoju.
Dodatkowo, funkcjonalności budowane w pobocznych gałęziach, pozostawały tam do czasu stabilizacji.
Wcielano je do głównego strumienia projektu po upewnieniu się, że nie powodują regresji.
\par
Na potrzeby pracy wykonano dwa projekty \longpause Silnik graficzny oraz grę szachową.
Starano się, aby możliwie uniezależnić je od siebie.
Z tego względu stworzona na ich potrzeby dwa osobne repozytoria.
Biblioteka graficzna jest niezależna, natomiast gra szachowa, która posiada na niej zależność używa funkcjonalności submodułów w celu pobrania źródeł.
Dzięki temu uniknięto duplikacji i zadbano, aby repozytoria były możliwie niewielkie.
\par
Komplementując zarządzanie wersją zdecydowano się na integrację z platformą GitHub.
Za jej pośrednictwem utworzono organizację do której podpięto repozytoria z projektami~\cite{porcelain_project}.
Dzięki temu zapewniono bezpieczeństwo na wypadek utraty danych z lokalnego dysku.
Otworzyło to także szereg możliwości zarządzania zadaniami, automatyzacji i zapewnienia jakości.
Aspekty te, podobnie jak gra szachowa omówione zostaną w dalszej części pracy.
Sekcja koncentrowała będzie się na stworzonej bibliotece graficznej.
\subsubsection{Hierarchia}
\listdirs{[Engine[Core]
                 [MetalBinding]
                 [Shaders]]
          [EngineTests[Engine][Extensions]]
          [...]
}
\linebreak
Repozytorium projektu posiada dwa główne katalogi.
\textit{Engine} zawiera w sobie właściwy kod projektu.
Funkcjonalności operate o język Swift przynależą do \textit{Core}.
\textit{MetalBinding} stanowi pomost pomiędzy programami cieniującymi, a wysokopoziomowym Swift.
Wewnątrz niego zorganizowane są pliki nagłówkowe języka C.
\textit{Shaders} z kolei jest miejscem w którym zgromadzono programy cieniujące.
Wewnątrz \textit{EngineTests} znajduje się katalog z dokładnym odwzorowaniem plików biblioteki silnika z przypadkami testowymi, a dodatkowo rozszerzenie funkcjonalności modułu do testowania.

\listdirs{[Core[UI]
               [Aliases]
               [Scene]
               [Animation]
               [Loaders]
               [Extensions]
               [Configuration]
               [Rendering]
               [...]]
}

Rdzeń projektu agreguje wiele komponentów.
Wśród nich wyróżnić można logiczny podział na części odpowiedzialne za interfejs użytkownika czy służące do konfiguracji projektu.
Najistotniejsze jednak są te odpowiedzialne za zarządzanie sceną, animacją i renderowaniem.
Ponadto chcąc zapewnić możliwie wysoką czytelność skorzystano z szeregu alternatywnych odwołań do typów.
Użyto także możliwości języka Swift, która pozwala rozszerzać istniejące funkcjonalności.
W zależności od modułu, do którego się odnosiły umieszczono je w odpowiednich podkatalogach.
\subsection{Konwencje}
Decyzja o wybraniu formy biblioteki wpłynęła na cele, które starano się osiągnąć.
Jednym z nich było uzyskanie przystępności i możliwie niewielkiego wysiłku potrzebnego do zrozumienia kodu przez osoby trzecie.
Pomimo charakteru badawczego pracy brano pod uwagę możliwość chęci kontrybucji ze strony społeczności.
Istnieją narzędzia, które na podstawie statycznej analizy są w stanie wskazać błędy przez które zadanie będzie utrudnione.
Od pewnego progu jednak czytelność postrzegana jest subiektywnie.
Zdecydowano się wprowadzić standard w odniesieniu do tworzonych plików i możliwie zachować go dla całego kodu źródłowego.
\subsubsection{Zawartość plików}
Pliki w projekcie podzielone są w taki sposób, aby zawartość w nich osadzona odnosiła się do pojedynczego komponentu.
Negatywnie wpływa to na ich ilość \longpause projekt posiada około 350.
Pozwala jednak zwiększyć ziarnistość podczas organizacji w katalogi, dzięki czemu precyzyjniej są one zgrupowane.
Dodatkowo, tak jak w przypadku innych zabiegów miało to ograniczyć narzut konieczny do zapoznania się z kodem.
\par
Pliki Swift posiadają w sobie pojedyncze klasy, struktury czy protokoły.
Zgrupowanie wielu definicji występuje jedynie w przypadku rozszerzeń oraz aliasów.
\par
Programy cieniujące dla pojedynczej jednostki translacyjnej posiadają definicję całego potoku, ale rozdzielono je dla różnych z nich.
Oznacza to, że dla pliku wspólnie zdefiniowane są shadery wierzchołków i fragmentów, a w przypadku programów obliczeniowych definicja występuje pojedynczo.
\par
W przypadku współdzielonego kodu w C przyjęto hybrydową strategię.
Struktury dostępne dla Swift i MSL zdefiniowane są w pojedynczych plikach.
Natomiast stałe odnoszące się do indeksów buforów pogrupowane są wspólnie w zależności od przeznaczenia.
Między sobą występują indeksy dla różnych shaderów fragmentów, ale shadery wierzchołków posiadają definicję gdzie indziej.
\subsubsection{Styl programistyczny}
Zadbano o uspójnienie stylu przyjętego w źródłach.
Określono standard w odniesieniu do formatowania.
Dokonywano jego weryfikacji za pomocą Swiftlint w przypadku kodu języka Swift.
Clang-tidy odpowiadał za źródła MSL oraz C.
Upewniono się także by każdy z plików posiadał stosowny nagłówek informujący o przynależności praw autorskich.
\lstinputlisting[language={Swift}, texcl=true, caption=Styl programistyczny na przykładzie struktury animacji]{code/PNKeyframeAnimation.swift}
\subsubsection{Nazewnictwo}
Biblioteki w ekosystemie Apple używają ustalonych przedrostków do zdefiniowania wprowadzanych typów.
Historycznie wynikało to z ograniczeń Objective-C, gdzie istniała jedyna, globalna przestrzeń nazw.
Wsparcie wsteczne sprawiło, że konwencja utrzymała się.
Z tego względu w projekcie zdecydowano się ją podtrzymać i wybrano przedrostek \textit{PN} do własnego użytku.
Dokonano jednak rozróżnienia ze względu na przeznaczenie danego typu.
W przypadku protokołów i struktur, które nie posiadają zachowania użyta jest podstawowa forma przedrostka.
Klasy bądź struktury posiadające zachowanie lub implementujące interfejsy używają przedrostka \textit{PNI}.
\subsubsection{Aliasy}
Powszechne jest, że jeden typ danych wykorzystywany jest w różnych zastosowaniach.
Przykładowo, liczba zmiennoprzecinkowa może posłużyć do określenia miary kąta, jak i odległości.
Nie zawsze jasny jest zakres, pomimo znajomości zastosowania.
Oczekując kąta jako parametru funkcji nie jest wiadome czy chodzi o formę stopni czy radiany.
Z tego powodu zdecydowano się na wprowadzenie szeregu aliasów, które miały pomóc użytkownikom uniknąć niejednoznaczności.
Określając specyfikację kamery zamiast oczekiwać miary pola widzenia w liczbie zmiennoprzecinkowej oczekuje się podania liczby typu \textit{PNRadians}.
Konwencja ta kontynuowana jest w bardzo wielu miejscach kodu, również by pozwolić autorowi nie zapomnieć o pewnych niuansach.

%
%  Copyright © 2022 Mateusz Stompór. All rights reserved.
%

\section{Architektura}
Dotychczasowo przedstawione sekcje pozwoliły uzyskać informacje na temat podstawowych założeń.
Wraz z nakreśleniem struktury otworzyło to możliwość wyjaśnienia koncepcji zastosowanych podczas projektowania komunikacji pomiędzy komponentami.
W dalszych rozważaniach podjęta zostanie próba omówienia procesu, który wykonywany jest każdorazowo w momencie akcji generowania nowej klatki obrazu.
Perspektywa obejmująca przepływ danych wewnątrz silnika graficznego obejmie zagadnienie hermetyzacji.
\subsection{Podejście programistyczne}
Swift jest językiem pozwalającym operować w oparciu o wielu paradygmatów programowania.
Z tego względu na etapie projektowania należało podjąć decyzję który z nich wykorzystać do wykonania biblioteki.
Głównymi kryteriami mającymi ukierunkować wybór była chęć zapewnienia wysokiej wydajności oraz minimalizacja błędów.
Dodatkowo, starano się sprawić by interfejs biblioteki był możliwie spójny z innymi dostępnymi rozwiązaniami.
\par
Początkowo rozważano użycie podejścia funkcyjnego.
Czynników przemawiających na korzyść było kilka.
Między innymi styl programistyczny, który wymuszony jest przez zasady sprawia, że kod w różnych obszarach aplikacji jest do siebie podobny.
Korzystanie z funkcji jasno definiujących dane wejściowe i zwracane pozytywnie wpływa też na testowalność.
Zdecydowano jednak o odrzuceniu pomysłu.
Programowanie funkcyjne wymagałoby wykorzystania dodatkowych bibliotek lub autorskich rozwiązań w celu sprawienia by biblioteka standardowa mogła obsłużyc paradygmat.
W sieci znaleźć można projekty takie jak Swiftz czy swift-overture, które wychodzą naprzeciw w kwestii problemu.
Wprawdzie zniwelowałyby konieczny nakład pracy, jednak analiza przykładów ich użycia ujawnia ich największą wadę.
Programowanie funkcyjne wśród technologii obecnych w Swift jest nadal ciekawostką.
Pomimo iż zastosowanie podejścia byłoby wykonalne to negatywnie wpłynęłoby na odbiór rozwiązania.
\par
Zwrócenie w stronę imperatywnego podejścia sprawiło, że do dyspozycji pozostał wyłącznie paradygmat proceduralny oraz obiektowy.
Dążenie do spójności z ekosystemem Apple zadecydowało o wyborze stylu zorientowanego obiektowo.
Zdecydowano się jednak na nieznaczną modyfikację podejścia.
Natywne w języku Swift obiektowe podejścia obecne w innych językach programowania zastąpione zostało przez programowanie zorientowane na protokoły.
Główne założenia narzucają, żeby abstrakcje budowane były przez implementację protokołów zamiast dziedziczenia klas.
W efekcie użytkownik uzyska możliwie wysoką spójność z pozostałymi bibliotekami.
Konieczne będzie włożenie dużego wysiłku, aby sprawić by tworzone komponenty możliwe łatwo pokryte mogły być przez testy.
W stosunku do podejścia funkcyjnego programowanie zorientowane na protokoły pozwala na znacznie większą dowolność w kwestii projektowania, co w tym przypadku niekoniecznie oznaczać musi zaletę.
\par
Określenie paradygmatu stosowanego podczas tworzenia silnika graficznego nadało pewien kształt w oparciu o który architektura byłaby rozwijana.
Nadal jednak dowolność w kwestii projektowania poszczególnych komponentów pozostawała wysoka.
Jednym z zabiegów, który zdecydowano się zastosować było rozdzielenie dane od zachowania w miejscach gdzie występuje taka możliwość.
Oznacza to, że modelowanie architektury oparte zostało o trzech głównych aktorów.
Dane, w formie złożonych pojemników o różnych typach.
Definicję interfejsu zachowania operującego na danych w formie protokołu oraz implementację wykonaną przez klasę.
Rozdzielność danych od zachowania sprawia, że zmiana algorytmu jest mniej złożona.
Pomaga to w integrowaniu alternatywnych implementacjach przynoszących nowe funkcjonalność, ale również w przypadku gdy należy zaślepić klasę podczas testów.
% mogę jako przykład podać pnkeyframe animation i różne samplery
\par
Moment publikacji SwiftUI sprawił, że Apple oficjalnie otworzyło się deklaratywne podejście w kwestii budowy UI oraz wzór projektowy MVVM.
Warunkiem koniecznym do skorzystania z funkcjonalności jest oparcie części aplikacji \longpause ViewModelu \longpause o programowanie reaktywne.
Tworzona biblioteka graficzna sama w sobie nie będzie posiadała modułów do obsługi UI.
Jeśli użytkownik planował będzie skomponować interfejs graficzny będzie musiał oprzeć go o komponenty systemowe.
Z tego względu konicznym będzie oddać użytkownikowi możliwość integracji z biblioteką reaktywną Combine gdzie to możliwe.
\par
W większości projektów rozważania na temat ogólnego podejścia można byłoby zakończyć po powyższym opisie.
Specyfika renderowania grafiki sprawia, że należy uwzględnić specyfikę działania karty graficznej w celu uzyskania optymalnej wydajności.
Przede wszystkim komunikacja z GPU sprawia, że następować musi synchornizacja pomiędzy procesorem graficznym, a głównym.
Maksymalna wydajność uzyskiwana jest kiedy minimalizuje się ilość tych okoliczności.
Aplikacje chcące zapewnić wysoką wydajność powinny możliwie rzadko zmieniać używane bufory i preferować dane zorganizowane w długie fragmenty.
Programowanie, które buduje abstrakcje wokół sprzętu na którym aplikacja będzie operowała nazywa się data-oriented design.
Zdecydowano się oprzeć o nie część biblioteki, która odpowiedzialna będzie za renderowanie obrazu
Pozostanie ona niewidoczna dla użytkownika, dzięki czemu nie wpłynie to negatywnie na przystępność.

%
%  Copyright © 2022 Mateusz Stompór. All rights reserved.
%

\subsection{Dobór Typów}
Struktury oraz klasy w Swift posiadają miedzy sobą fundamentalną różnice.
Pierwsze z nich przekazywane są przez wartość, drugie przez referencję.
W przypadku wykonania kopii obiekty będące strukturami są niezależne od siebie.
Instancje klas natomiast wskazywały będą na ten sam obiekt.
Zarówno klasy, jak i struktury mogą implementować protokoły, natomiast tylko klasy mogą tworzyć hierarchię dziedziczenia.
Dlatego o wyborze podczas definiowania nowego typu decydował obsługiwany przypadek użycia.
Zgodnie z zaleceniami producenta starano się korzystać ze struktur jeśli nie zachodziła inna potrzeba.

\subsection{Komponenty silnika}
\begin{figure}[H]
    \begin{center}
        \includegraphics[width=15cm]{images/pnengine/components.png}
    \end{center}
    \caption{Architektura silnika}
    \label{fig:components}
\end{figure}
Z perspektywy użytkownika chcącego dokonać renderowania najistotniejszą klasą jest \textit{PNIEngine}.
W zamyśle powinien on uzyskać instancję obiektu podczas inicjacji swojej aplikacji, a następnie, w metodzie odpowiedzialnej za odświeżanie ekranu wywołać metodę \textit{draw()} na silniku.
Wówczas abstrakcyjna definicja sceny zostanie przetworzona na nową klatkę wyświetloną na ekranie urządzenia.
Jednak by wynikowy obraz nie był pusty należy uprzednio zaaranżować scenę.
\par
Relacja pomiędzy klasą silnika, a sceny oparta jest o typ \q{jeden do jednego}.
Klasa sceny, której detale omówione zostaną w dalszej części pracy, reprezentuje wykreowany świat za pomocą acyklicznego grafu skierowanego.
Manipulując właściwościami obiektów znajdujących się w jego strukturze tworzyć można animacje i wykrywać interakcje pomiędzy użytkownikiem, a biblioteką.
\par
Częstą koniecznością, która zachodzi równolegle do tworzenia nowej klatki jest aktualizacja danych zsynchronizowana z renderowaniem.
Funkcjonalność realizująca tę potrzebę zaimplementowana jest w klasie \textit{PNITaskQueue}.
Wykorzystywana jest ona wewnętrznie przez bibliotekę, jednak może być również użyta przez użytkownika do uruchomienia zgeneralizowanego zadania.
\par 
Wywołanie metody \textit{draw()} na silniku propagowane jest wraz z definicją sceny oraz kolejką zadań do klasy \textit{PNIWorkloadManager}.
Sama w sobie traktowana może być jako adapter sceny pomiędzy strukturami wygodnymi w użyciu przez użytkownika, a efektywnymi do procesowania przez kartę graficzną.
Delegując procesowanie grafu sceny do \textit{PNITranscriber} zamienia go ona na postać niezagnieżdżoną.
Obiekty będące w scenie przepisywane są do struktury \textit{PNSceneDescription} gdzie pogrupowane są w zależności od rodzaju i przechowywane w formie tablicy.
Dodatkowo, dane, które zmieniają się w czasie - pozycje, właściwości świateł - i GPU potrzebuje do stworzenia klatki przekazywane są do \textit{PNIBufferStore}, gdzie następuje synchronizacja pomiędzy pamięcią RAM CPU, a GPU.
\par
Bufory oraz opis sceny przekazywane są dalej do klasy \textit{PNIRenderingCoordinator}.
Jest to miejsce, w którym rozpoczyna się właściwa komunikacja z potokiem GPU.
Do tego momentu klasy obecne w silniku posiadały dostęp jedynie w celu populacji buforów danych.
Koordynator natomiast rozpoczyna zlecanie zadań do wykonania przez GPU za pomocą udostępnionych w tym celu przez Metal kolejek.
Renderowanie klatki jest jednak stosunkowo złożone, wieloetapowe, więc dokonano podziału na niezależne od siebie kroki.
Koordynator przekazuje dane wejściowe do \textit{PNIPipeline}.
Otrzymawszy wynikową teksturę przedstawiający aktualną klatkę obrazu deleguje ją do wyświetlenia na ekranie urządzenia.
\par
Etapy - klasy implementujące interfejs \textit{Stage}, mające adekwatny przyrostek w nazwie - opierają się podobnie jak klasa potoku na prostej idei.
Każdy z nich posiada dane wejściowe oraz wyjściowe w formie tekstur, a dodatkowo podczas renderowanie przekazywane są do niego dane odnoszące się do opisu sceny.
\begin{figure}[H]
    \begin{center}
        \includegraphics[width=15cm]{images/pnengine/stage.png}
    \end{center}
    \caption{Podział etapów na klasy wykonujące specyficzne zadania}
    \label{fig:stage}
\end{figure}
Podobnie jak w poprzednich przypadkach instancje klas typu \textit{Stage} posiadają metodę draw.
Wykonują one jednak tylko część odpowiedzialności koniecznej do stworzenia fragmentu klatki.
Mianowicie posiadają dostęp do obiektów metal typu \textit{MTLRenderPassDescriptor}.
Deskryptory te stanowią dokładny opis programu cieniującego. 
Składają się na to formaty danych wyjściowych i wejściowych, odnośniki do funkcji wykorzystywanych podczas cieniowania.
Na ich podstawie tworzone są enkodery, które wykorzystać można by wskazać jakie dane należy wykorzystać w celu wygenerowania klatki.
\lstinputlisting[language=Swift, caption=Logika renderowania wykorzystywana w klasach pochdnych \textit{PNRenderJob}]{code/PNShadowJob.swift}
\par
Na końcu łańcucha wykonywanego przez CPU znajdują się zadania - ujęte jako pochodne interfejsu \textit{PNRenderJob} lub \textit{PNComputeJob}.
Przechodzą one przez opis sceny oraz zlecają zadanie rysowania dla każdego obiektu kwalifikującemu się według kryteriów.

% Co napisać w tej sekcji?
% \subsection{Przepływ danych}
%
%  Copyright © 2022 Mateusz Stompór. All rights reserved.
%

\subsection{Hermetyzacja}
Jednym z założeń projektu było zachowanie przystępności dla użytkownika.
Z tego powodu zdecydowano się, również przez wzgląd na uwzględnienie praktyk inżynierii oprogramowania powszechnie uważanych za słuszne, ukryć znaczną część komponentów odpowiedzialnych za tworzenie klatki.
Do dyspozycji użytkownika pozostają jedynie klasy \textit{PNScene} oraz \textit{PNITaskQueue} do których dostęp może uzyskać za pomocą \textit{PNIEngine}.
Konieczność zmiany formatu sceny z drzewiastej na płaską pozostaje niepubliczna.
%
%  Copyright © 2022 Mateusz Stompór. All rights reserved.
%

\subsection{Współdzielenie kodu}
Programy cieniujące, podobnie jak kod wykonywany na CPU używają struktur w celu zgrupowania typów prostych w spójną abstrakcję.
Metal nie jest kompatybilny ze Swift, w celu przesłania danych producent w przykładach stosuje dwukrotną deklarację typów dla każdego z języków.
Zdecydowano się jednak wykorzystać inny pomysł, który mógłby być mniej podatny na pomyłki i zredukować ilość zduplikowanego kodu.
Podstawowa definicja struktury stworzona jest w kodzie C, kompatybilnym z Metal.
Dzięki temu może być wykorzystana w programach cieniujących.
W następnym kroku nagłówki C grupowane są w moduły, które kompilator XCode udostępnia do kodu Swift.
Dzięki temu za pomocą pojedynczego importu modułu dane opisane w kodzie Swift mogą być w łatwy sposób przekazywane do GPU, a pojedyncza definicja sprawia, że ustrzec można się przed błędami w przypadku braku propagacji zmian.
Dodatkowo Swift choć wewnętrznie zapewnia to standard nie daje gwarancji w kwestii uporządkowania pól dla struktury, co jest kluczowe w celu właściwej interpretacji danych.
\code{C}
     {Definicja struktury w języku C}
     {code/code_sharing/Camera.h}
\code{Swift}
     {Deklaracja modułu Swift składającego się z kodu C}
     {code/code_sharing/module.map}
\code{Swift}
     {Użycie modułu w kodzie Swift}
     {code/code_sharing/PNSceneDescription.swift}

%
%  Copyright © 2022 Mateusz Stompór. All rights reserved.
%

\section{Scena}
%
%  Copyright © 2022 Mateusz Stompór. All rights reserved.
%

Cała zawartość sceny, którą użytkownik może stworzyć w oparciu o silnik konfigurowalna jest przy pomocy klasy \textit{PNScene}.
Zawiera ona graf służący w zamyśle do przechowania obiektów mogących być scharakteryzowanym przez pozycję.
Dodatkowo, scena posiada definicję mapy otoczenia, która opcjonalnie może zostać wygenerowana.
Nie przynależy ona do grafu, ponieważ jest unikalna dla całej sceny i opis położenia w jej przypadku pozbawiony jest sensu.
Z podobnego powodu światła kierunkowe przechowywane są poza nim.
\figh{images/pnengine/scene/scene.png}
     {Klasa sceny służąca do opisu świata przedstawionego}
     {fig:scene}
     {9cm}

%
%  Copyright © 2022 Mateusz Stompór. All rights reserved.
%

\subsection{Graf sceny}
Graf zaimplementowany został w oparciu o drzewo.
Pomiędzy węzłami występuje relacja rodzic-potomek i wymagane jest, aby dany węzeł występował w strukturze dokładnie jeden raz.
Oznacza to, że dwa różne węzły nie mogą posiadać tego samego potomka.
Dozwolona jest dowolna ilość potomków.
\par
Każdy z węzłów zbudowany jest z trojga informacji.
Referencji do rodzica, listy dzieci oraz odnośnika do samych danych.
Korzeń drzewa nie posiada rodzica, jego referencja będzie pusta.
\par
Struktura danych jest generyczna.
Może posłużyć do przechowania informacji dowolnego typu.
Drzewo posiada zaimplementowane najpopularniejsze operacje, takie jak wyszukiwanie, wstawianie elementu, usuwanie.
Ze względu na separacje zachowania od samych danych wykonywane jest to za pomocą pochodnych interfejsu \textit{PNNodeInteractor}.
\figh{images/pnengine/scene/node.png}
     {Definicja węzła w grafie sceny}
     {fig:scene_node}
     {5.4cm}
% TODO: Napisać po co w ogóle ta hierarchia została wprowadzona
%
%  Copyright © 2022 Mateusz Stompór. All rights reserved.
%

\subsection{Właściwości węzła}
Dane przechowywane w grafie biblioteki oparte muszą być o interfejs \textit{PNSceneNode}.
Zawiera on w sobie deklarację kilku istotnych dla silnika rzeczy.
Dotyczy to pozycji, otoczki zewnętrznej obiektu oraz jego identyfikatora.
Węzeł posiada informacja nie tylko odnoszące się do lokalnej pozycji i otoczki dla obecnego obiektu, ale także wartości w relacji do układu współrzędnych świata.
Dzięki temu użytkownik wygodnie porównywać może pozycje obiektów między sobą, a ponieważ wartość jest przechowywana w węźle nie trzeba jej każdorazowo obliczać.
\par
Oprócz właściwości węzeł sceny posiada także zachowanie.
Może on implementować metodę \textit{update()}, która wykonywana jest przy generowaniu każdej klatki.
Najistotniejszą jednak jest \textit{write()} służąca do zamiany hierarchicznej struktury drzewa na płaską, wydajną podczas renderowania.
Obiekt musi być w stanie zakodować siebie wewnątrz struktury \textit{PNSceneDescription}, opisanej w dalszej części pracy.
\lstinputlisting[language=Swift, caption=Interfejs bazowy dla obiektów w węzłach grafu sceny]{code/PNSceneNode.swift}
%
%  Copyright © 2022 Mateusz Stompór. All rights reserved.
%

\subsection{Wybrane węzły}
Użytkownik jest w stanie na własną rękę implementować interfejs \textit{PNSceneNode} w celu personalizacji węzłów, jednak nie jest to konieczne do skorzystania z biblioteki.
W celu wygenerowania obrazu i aranżacji sceny skorzystać można z szeregu uprzednio stworzonych węzłów, które rozróżnić można na kilka typów.
Zdecydowano się stworzyć wiele niewielkich, skoncentrowanych typów.
Dzięki temu wyspecjalizowane są one w kierunku pełnionej przez siebie funkcjonalności i łatwiej można je analizować niż w przypadku alternatywnego podejścia, gdzie węzeł jednocześnie mógłby pełnić wiele ról.
\subsubsection{Transformacja}
Podstawowym typem są węzły transformacji.
Ich zadaniem jest grupowanie obiektów.
Dzięki temu umożliwiają modyfikację pozycji wszystkich potomnych węzłów za pomocą pojedynczej zmiany.
Możemy do nich zaliczyć interfejs \textit{PNSceneNode} w podstawowej formie oraz jego pochodną \textit{PNAnimatedSceneNode}.
\lstinputlisting[language=Swift, caption=Węzeł obsługujący animacje]{code/PNAnimatedNode.swift}
Główną różnicą występującą pomiędzy nimi jest źródło używane do próbkowania pozycji.
Choć interfejs tego nie narzuca to w przypadku \textit{PNSceneNode} zakłada się, że pozycja określana jest na podstawie zmiennych modyfikowanych przez użytkownika.
\textit{PNAnimatedNode} ma w tym celu używać animatora, który pozycje uzyskuje z szeregu animacji kluczowych dla zadanego punktu w czasie.
\figh{images/pnengine/scene/transform_node.png}
     {Hierarchia węzłów służących do reprezentacji transformacji}
     {fig:scene_transform}
     {10cm}
\subsubsection{Kamera}
Kolejnym istotnym elementem sceny są węzły kamery.
Służą one do przedstawienia perspektywy z której scena zostanie wyrenderowana.
Rozwijają one koncepcje wdrożone podczas prezentacji węzłów transformacji.
Pozycja kamery oparta może być o zmienną reprezentującą konkretną, modyfikowalną wartość lub być próbkowana z klatek kluczowych.
\figh{images/pnengine/scene/camera.png}
     {Hierarchia węzłów służących do reprezentacji kamery}
     {fig:scene_camera}
     {8cm}
\par
Sam interfejs węzła w stosunku rozszerza podstawowy interfejs o dodanie referencji do kamery.
Istnieje szansa, że osadzona na scenie zostanie więcej niż jedna kamera.
W takiej sytuacji wykorzystywana zostaje także zmienna \textit{priority} na podstawie której kamery są sortowane.
Do renderowania wybrana zostanie ta z najwyższym priorytetem.
\lstinputlisting[language=Swift, caption=Interfejs węzła kamery]{code/PNCameraNode.swift}
\subsubsection{Model}
\figh{images/pnengine/scene/mesh.png}
     {Hierarchia węzłów służących do reprezentacji siatek}
     {fig:scene_mesh}
     {\linewidth}
Do dyspozycji oddano także węzły służące do reprezentacji siatek.
W podstawowej formie \textit{PNMeshNode} może posłużyć do aranżacji statycznej siatki, która dodatkowo obsłuży animacje kluczowe w rozszerzeniu \textit{PNAnimatedMeshNode}.
\lstinputlisting[language=Swift, caption=Interfejs węzeła siatki]{code/PNMeshNode.swift}
Biblioteka wspiera także animację szkieletową dla siatek.
\textit{PNRiggedMeshNode} jest pochodną interfejsu \textit{PNMeshNode} i wzbogaca deklarację o odnośnik do szkieletu.
Podobnie jak w innych przypadkach animację wzbogacić można o technikę animacje na podstawie klatek kluczowych.
\lstinputlisting[language=Swift, caption=Interfejs węzła siatki obsługującej animację szkieletową]{code/PNRiggedMeshNode.swift}
\subsubsection{Światła}
Aby obraz w ogóle mógł powstać konieczne są źródła światła, która sprawią, że kamera zarejestruje kształty modeli.
\lstinputlisting[language=Swift, caption=Interfejs węzła światła nadającego wpływ komponentu ambientowego]{code/PNAmbientLightNode.swift}
Poza światłem kierunkowym definiowanym poza grafem sceny wykorzystać można źródła ambientowe, punktowe dookólne oraz skierowane.
\lstinputlisting[language=Swift, caption=Interfejs węzła światła punktowego\, kierunkowego]{code/PNSpotLightNode.swift}
We wszystkich przypadkach interfejsy są zwięzłe i oczekują jedynie referencji do wybranego typu światła.
\lstinputlisting[language=Swift, caption=Interfejs węzła światła punktowego\, dookólnego]{code/PNOmniLightNode.swift}
\figh{images/pnengine/scene/lights.png}
     {Hierarchia węzłów służących do reprezentacji świateł}
     {fig:scene_lights}
     {10cm}
\subsubsection{Efekty cząsteczkowe}
Ostatnim typem węzła, który może zostać użyty w grafie jest węzeł efektów cząsteczkowych.
Służy on do osadzenia emitera na scenie.
\figh{images/pnengine/scene/particle.png}
     {Hierarchia węzłów służących do reprezentacji efektów cząsteczkowych}
     {fig:scene_particle}
     {4cm}
\lstinputlisting[language=Swift, caption=Interfejs węzła służącego do obsługi efektów cząsteczkowych]{code/PNParticleNode.swift}
%
%  Copyright © 2022 Mateusz Stompór. All rights reserved.
%

\subsection{Interakcja}
Poza jednostkami stacjonarnymi biblioteka przeznaczona jest do współpracy z urządzeniami przenośnymi wyposażonymi w ekrany dotykowe.
Z tego powodu należało zadbać o wykrywanie interakcji pomiędzy użytkownikiem, a modelami osadzonymi na scenie.
\figh{images/raycasting.jpg}
     {Ilustracja idei ray-castingu}
     {fig:ray-casting}
     {8cm}
Do dyspozycji oddano klasę realizującą interfejs \textit{PNScreenInteractor}.
Przeszukuje ona graf sceny na podstawie współrzędnych punktu wybranego na ekranie.
W przypadku powodzenia zwraca obiekt będący najbliżej kamery.
\lstinputlisting[language=Swift, caption=Węzeł kamery]{code/PNScreenInteractor.swift}
W tym celu wykorzystano technikę rzucania promieni.
Promień osadzany jest w punkcie początkowym kamery z uwzględnieniem jej transformacji, a następnie wykonywany jest algorytm przecinania półprostej z ramkami ograniczającymi modeli.

\section{Komponenty sceny}
Poprzednia sekcja pozwoliła uzyskać wiedzę na temat przekroju typów obiektów, które umieścić można na scenie.
Chcąc dokładniej zrozumieć ich strukturę budowy i mechanizmy, które w sobie zawierają warto przyjrzeć im się nieco bliżej.
Z tego powodu odsunięta zostanie chwilowo perspektywa grafu sceny, a koncentracja obejmie poszczególne typy obiektów dokonując ich charakteryzacji.
% \subsection{Transformacje}
% \subsection{BoundingBox}
\subsection{Kamera}
Typ Kamery jest protokołem.
Wewnątrz niego najistotniejszą właściwością opisującą kamerę jest macierz projekcji.
Wykorzystywana jest ona podczas renderowania obrazu.
Reguluje zasięg widoczności oraz kąt, który obejmuje pole widzenia.
Ściśle powiązana z macierzą projekcji jest otoczka.
Używana jest w procesie filtrowania niewidzialnych obiektów.
Ostatnim parametrem charakteryzującym kamerę jest zmienna definiująca czy jest ona włączona.
Przydatna jest w momencie gdy na scenie umieszczonych jest mnoga ilość kamer i zachodzi potrzeba dynamicznej zmiany z jednej na inną.
Biblioteka posiada dwie implementacje interfejsu \longpause kamerę perspektywiczną (\textit{PNPerspectiveCamera}) oraz ortograficzną (\textit{PNOrthographicCamera}).
\lstinputlisting[language=Swift, caption=Protokół kamery]{code/PNCamera.swift}
\subsection{Siatki}
\figh{images/pnengine/scene/mesh-details.png}
          {Budowa siatki}
          {fig:mesh_details}
          {10cm}
\subsection{Materiały}
Wygląd fragmentów siatki, bądź jej całości konfigurowany może być za pomocą materiału.
Użyty model oświetlenia, oparty o podstawy fizyczne, sprawia, że konieczne do określenia są trzy parametry \longpause kolor podstawowy, surowość oraz metaliczność.
Dodatkowo, w celu zwiększenia rozdzielczości próbkowania wektorów normalnych istnieje możliwość zdefiniowania ich w formie tekstury.
Choć parametry te przechowywane są w formie tekstury to użytkownikowi oddano metody, które na podstawie wartości przygotują jednolitą mapę.
Materiał posiada także identyfikator i informację na temat jego przezroczystości.
Pozwala to modułowi renderowania odroczyć nanoszenie przejrzystych siatek na dalszy etap renderowania.
\lstinputlisting[language=Swift, caption=Definicja materiału]{code/PNMaterial.swift}
%
%  Copyright © 2022 Mateusz Stompór. All rights reserved.
%

\subsection{Światła}
Zgodnie z tym co wyjaśniono w poprzedniej sekcji biblioteka wspiera trzy podstawowe źródła światła.
Są tą źródła kierunkowe (imitujące słońce), punktowe (żarówki) oraz punktowe, kierunkowe (latarki, reflektory samochodów).
Każde z nich opisane jest przy pomocy wektora reprezentującego kolor w formacie RGB.
W celu udogodnienia procesu generowania wartości biblioteka pozwala także na wytworzenie koloru na podstawie temperatury barwowej światła białego podanego w Kelvinach.
Moc źródeł opisuje parametr intensywności.
Użytkownik może skonfigurować czy dane źródło rzuca cienie.
Wpływa to znacząco na wydajność generowanego obrazu, natomiast zwiększa jego realizm.
W przypadku źródeł kierunkowych możliwa jest także modyfikacja ich orientacji.
\lstinputlisting[language=Swift, caption=Protokół światła kierunkowego]{code/lights/PNDirectionalLight.swift}
\par
Nieco odmienne przeznaczenie, choć nadal przynależące wykorzystaniem w stosunku do pozostałych świateł ma światło ambientowe.
Podobnie jak w przypadku pozostałych źródeł charakteryzuje się je za pomocą barwy oraz intensywności.
Działa ono jednak obszarowo w sposób stały \longpause wpływ istnieje lub nie, ale nie zmienia się on wraz ze zmianą odległości.
Źródło ambientowe ma sprawić, że ekran nie będzie czarny w momencie gdy w danym miejscu sceny nie znajduje się żadne źródło.
\lstinputlisting[language=Swift, caption=Interfejs źródła ambientowego]{code/lights/PNAmbientLight.swift}
\subsection{Efekty cząsteczkowe}
\figh{images/pnengine/scene/particle-details.png}
          {Hierarchia komponentu efektów cząsteczkowych}
          {fig:particle_details}
          {10cm}
\subsection{Techniki animacji}
Biblioteka wspiera dwie techniki wspierające nadanie obiektom płynnego ruchu.
Są to animacje bryły sztywnej oraz animacje szkieletowe.
Mogą być one łączone ze sobą.
\subsubsection{Animacja bryły sztywnej} 
Przypadek użycia tego rodzaju animacji jest szeroki.
W przypadku grafu sceny skorzystać można z niego dla wszystkich rodzajów obiektów.
\figh{images/pnengine/animation/keyframe.jpg}
    {Animacja na podstawie klatek kluczowych}
    {fig:animation_keyframe}
    {10cm}
\par
Technika oparta jest o użycie klatek kluczowych.
Użytkownik definiuje pozycje obiektu i powiązuje je z punktem na osi czasu.
W przypadku konieczności skorzystania z pośredniego punktu pozycja jest interpolowana na podstawie sąsiednich.
Do definicji klatek kluczowych w bibliotece służy struktury \textit{PNKeyframeAnimation}.
\lstinputlisting[language=Swift, caption=Interfejs służący do obsługi animacji]{code/PNKeyframeAnimation_short.swift}
\par
Koordynacja procesu generowania aktualnej pozycji wykonywana jest za pośrednictwem interfejsu \textit{PNAnimator}.
Z punktu widzenia zachownia jest on odpowiedzialny za pojedyncze zadanie.
Na podstawie klatek kluczowych skali, rotacji oraz transformacji zamkniętej w \textit{PNAnimatedCoordinateSpace} musi wyznaczyć finalną pozycje w formie macierzy transformacji.
Do dyspozycji użytkownika oddana została klasa \textit{PNIAnimator} implementująca interfejs.
Może on wykorzystać ją wypełniając interesujące go parametry dzięki czemu skorzysta z gotowego koordynatora procesu.
\lstinputlisting[language=Swift, caption=Interfejs służący do obsługi animacji]{code/PNAnimator.swift}
\par
Wyznaczanie klatki animator rozpoczyna od sprawdzenia aktualnego czasu na chronometrze.
Na jego podstawie odpytuje on instancję implementującą \textit{PNSampleProvider} o wskazanie klatek będących najbliżej punktu.
Dzięki modyfikacji jego zachowania możliwy jest do uzyskania szereg efektów takich jak ruch wsteczny animacji, zapętlanie czy animacja poklatkowa.
Klatki kluczowe przekazywane są następnie do interpolatora.
Wyznacza ona pozycję pośrednią pomiędzy nimi.
Biblioteka wspiera jedynie interpolację liniową jednak możliwe jest rozszerzenie funkcjonalności we własnym zakresie.
W ostatnim kroku dane na temat skali składane są w macierz jako iloczyn \textit{TRS} lub textit{RTS} w zależności od potrzeb użytkownika.
\begin{figure}
    \[ M_{result} = M_{translation} * M_{rotation} * M_{scale} \]
    \[ M_{result} = M_{rotation} * M_{translation} * M_{scale} \]
    \caption{Dwie możliwości składania transformacji}
\end{figure}

\lstinputlisting[language=Swift, caption=Interfejs samplera]{code/PNSampleProvider.swift}
\subsubsection{Animacja szkieletowa}
\figh{images/pnengine/animation/skeleton.png}
    {Idea animacji szkieletowej}
    {fig:animation_skeletal}
    {10cm}
\clearpage

\section{Renderowanie}
\begin{figure}[H]
    \begin{center}
        \includegraphics[width=15cm]{images/pnengine/toy_drummer.jpg}
    \end{center}
    \caption{Klatka prezentująca wyrenderowany model zabawki}
    \label{fig:toy_drummer}
\end{figure}
\begin{figure}[H]
    \begin{center}
        \includegraphics[width=15cm]{images/pnengine/sponza_scene.jpg}
    \end{center}
    \caption{Scena z popularnego modelu firmy Crytek \longpause Sponza}
    \label{fig:sponza}
\end{figure}
\begin{figure}[H]
    \begin{center}
        \includegraphics[width=15cm]{images/pnengine/bloom.jpg}
    \end{center}
    \caption{Przykład efektu bloom na podstawie klatki stworzonej przez bibliotekę}
    \label{fig:bloom}
\end{figure}
    \subsection{Ogólne podejście}
    \subsection{Potok renderowania}
    \subsection{Zawarte techniki}
        \subsubsection{PBR}
        \subsubsection{Cienie}
        \subsubsection{Mapowanie Normalnych}
        \subsubsection{Przeźroczystość}
        \subsubsection{Efekty cząsteczkowe}
        \subsubsection{Bloom}
        \subsubsection{Environment mapping}
        \subsubsection{SSAO}
        \subsubsection{Postprocessing}
\clearpage
\section{Wczytywanie danych zewnętrznych}
    \subsection{Modele}
    \subsection{Tekstury}
\clearpage
\section{Interfejs programistyczny}
    \subsection{Komponenty publiczne}
    \subsection{Testowalność}
    \subsection{Konfiguracja}
\clearpage
\section{Wydajność Silnika}
%
%  Copyright © 2022 Mateusz Stompór. All rights reserved.
%

\section{Jakość Projektu}
Systematyczny rozwój projektu bezpośrednio wpływał na zwiększanie się jego objętości.
Dążąc do zapewnienia rozwiązania możliwie wolnego od błędów cyklicznie wykonywano manualną weryfikacje funkcjonalności.
Wymagania czasowe tego procesu starano się zminimalizować za pomocą narzędzi dostępnych na rynku.
Nie była to jednak jedyna płaszczyzna w jakiej postrzegano jakość.
Skierowanie w stronę kodu otwartego źródłowo wymaga by kryterium objęło także sferę czytelności.
\par
Pierwszym krokiem w tym kierunku była decyzja o użyciu analizatorów kodu.
Ich działania opiera się o statyczne podejście.
Wnioski wyciągane są na podstawie zawartości plików, a nie efektów które potencjalnie daje wywołanie funkcji.
\par
W przypadku języka Swift wykorzystano \textit{SwiftLint}.
Umożliwia wychwycenie błędy kompilacji na wczesnym etapie.
Wyraźnie pomaga również w momentach gdy następuje dereferencja wartości opcjonalnej.
Wówczas linter weryfikuje czy nie dokonano tego w sposób, który mógłby objawić się błędem krytycznym.
Większość reguł, które implementuje odnoszą się jednak do stylu programistycznego.
Wypracowane zostały one w oparciu o przemyślenia płynące od twórców Swift.
Uwzględnienie sugestii sprawia, że kod staje się spójny wewnętrznie, ale także z innymi publicznymi projektami Swift, które zaaplikowały podejście.
\par
Drugim analizatorem, który zdecydowano się zastosować był \textit{markdownlint}.
Obszar, który pokrył obejmował dokumentację.
Pozwolił zadbać o poprawne formatowanie i wykluczenie błędów syntaktycznych.
\par
Dążąc do zapewnienia poprawności działania algorytmów wykonywano testy automatyczne.
Oparto je o natywną bibliotekę \textit{XCTest} i zaprogramowane zostały, podobnie jak sam silnik graficzny, w języku Swift.
Starano się możliwie maksymalizować ich użycie, jednak ze względu na nakład czasowy jaki narzuca ich rozwój ograniczono się do newralgicznych aspektów silnika.
Postrzegano za nie miejsca w których zachodzi użycie aparatu matematycznego lub złożoność algorytmu jest wysoka.
Oznacza to, że w głównej mierze wykorzystane zostały przy weryfikacji mechanizmów obecnych przy grafie sceny, animacjach, przecięciach pomiędzy obiektami na scenie.
Ostatnim przypadkiem użycia były struktury danych używane w wielu miejscach aplikacji, jak pomocnicze metody do operacji wektorowych czy macierzowych.
Sumarycznie pokrycie testami kodu Swift oscylowało na poziomie 30\%.
\par
Nie należy zapominać, że około 10\% z całego kodu który został stworzony dotyczył programów cieniujących.
Wyłączono je całkowicie poza obszar objęty testami automatycznymi.
Wynikało to z niemożności przeprowadzenia takiej operacji.
Na rynku istnieje niewielka ilość rozwiązań skierowanych na to zagadnienie.
Największą popularnością cieszy się produkt Google \longpause \textit{Amber}~\cite{google_amber} \longpause jednak nie wspiera on platformy Apple.
Rozważano rozwój własnego rozwiązania w ramach pracy, jednak nakład czasowy wymagany do spełnienia warunku używalności sprawił, że zrezygnowano z tego pomysłu.

\section{Dokumentacja}
Projekt posiada dokumentację występującą w dwóch formach.
Pierwszą z nich jest opis repozytorium przy użyciu pliku \textit{README}, drugi aspekt, zebrany w formie paczki, dotyczy zaś architektury i interfejsu.
\par
\textit{README} przygotowany jest z myślą o pierwszym kontakcie z potencjalnym odbiorcą.
Nie zakłada on profilu jaki osoba posiada.
W zamyśle krótko podsumowuje kluczowe cechy.
Traktowany może być jako wizytówka projektu.
Opisuje przeznaczenie, funkcjonalności, wymagania oraz proces budowy.
Wyjaśnia w jaki sposób uzyskać dostęp do dokumentacji interfejsu biblioteki w celu wykonania integracji.
Zadbano by dostarczyć kilka klatek w formie grafik poglądowych.
\par
Referencja kodu stworzona została w oparciu o powszechnie stosowany mechanizm.
Archiwum ze stronami dokumentacji generowane jest na podstawie specjalnych komentarzy dołączonych do źródeł.
Zawierają one opisy parametrów wejściowych, wartości zwracanych, a także samego celu stojącego za daną metodą czy klasą.
Objęły one publiczne komponenty, a dzięki dołączeniu dodatkowych plików poruszyły także zagadnienie architektury.
\lstinputlisting[language=Swift, caption=Dokumentacja kodu w języku Swift]{code/docc.swift}
\par
W podstawowym przypadku symbole, które udokumentowano pogrupowane są w zależności od typu - na klasy, interfejsy, struktury, itd.
Zdecydowano jednak wprowadzić się podział na podstawie ich semantyki.
Rozróżnienie występuje w zależności od funkcjonalności do jakiej dany symbol należy.
Struktura dokumentacji przypomina nieco strukturę projektu i wśród wielu innych wymienić można takich zagadnienia jak: zarządzanie sceną, elementy sceny, interfejs graficzny.
Starano się aby możliwie w przypadku problemu symbol, który będzie analizowany przez użytkownika zestawiony był z pokrewnymi sobie.
\par
Wykorzystanym narzędziem, które posłużyło do generowania dokumentacji było DocC.
Jest ono natywne dla platform Apple. 
Stworzone zostało przez producenta i dostarczane jest w ramach środowiska programistycznego XCode.
Narzędzie potrafi wytworzyć paczkę \textit{docarchive}, która następnie możne podlegać dystrybucji i być zaimportowana w innym środowisku XCode.
\par
Największą zaletą DocC jest fakt, że tworzy ono interfejs graficzny spójny z natywnymi bibliotekami systemowymi.
Dzięki temu konsumpcja treści jest bardziej przystępna.
Programiści tworzący natywne aplikacje nie będą mieli problemu, aby odnaleźć się podczas nawigacji.
\par
Dystrybucja dokumentacji przy użyciu pojedynczego pliku \textit{docarchive} może wydawać się wygodna, ma jednak pewne wady.
Wymusza na użytkowniku pobranie paczki, posiadanie środowiska XCode, a także import archiwum.
Z tego względu zdecydowano się na dodatkową formę publikacji.
W tym celu skorzystano z możliwości generowania plików HTML w narzędziu DocC.
Następnie przy użyciu funkcjonalności platformy GitHub \longpause Pages \longpause osadzono dokumentacje w formie strony internetowej.
Dzięki temu dostęp do najnowszej wersji referencji jest natychmiastowy, a w razie braku łączności sieciowej do dyspozycji pozostaje skorzystanie z archiwum \textit{docarchive}.
\begin{figure}[H]
    \begin{center}
        \includegraphics[width=15cm]{images/docs/camera_docs.png}
    \end{center}
    \caption{Dokumentacja na przykładzie klasy kamery}
    \label{fig:camera_docs}
\end{figure}

%
%  Copyright © 2022 Mateusz Stompór. All rights reserved.
%

\section{Infrastruktura projektu}

% zagaić jakoś, że cały czas się rozwijało 
% projekt stawał się coraz bardziej rozwinięty i trudno było pamiętać o wielu rzeczach
% z pomocą przyszedł serwer ciągłej integracji
% O Serwerze ciągłej integracji}
% napisz o buildzie docsów
% napisz o archive biblioteki dla obu platform
% napisz o unit testach
% napisz o lincie

% Napisać, że dopełnieniem CI było odpowiednie skonfigurowanie githuba

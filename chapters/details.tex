%
%  Copyright © 2022 Mateusz Stompór. All rights reserved.
%

%
%  Copyright © 2022 Mateusz Stompór. All rights reserved.
%

\chapter{Omówienie projektu}
Niniejszy rozdział poświęcony jest opisowi biblioteki graficznej stworzonej na potrzeby pracy.
Odzwierciedla ona wizję autora na rozwiązanie przygotowane z myślą o produkcji niewielkich gier
oraz wizualizacji wkomponowanych w aplikacje użytkowe.
\par
Przybliżenie rozpocznie się od przedstawienia zastosowanych technologii oraz struktury plików.
W dalszej części poruszona zostanie kwestia architektury, by możliwie było przejście do głównych modułów służących do zarządzania sceną oraz renderowania.
Finalnie przedstawiona zostanie dokumentacja oraz działania podjęte na rzecz uzyskania kodu możliwie wolnego od błędów.

%
%  Copyright © 2022 Mateusz Stompór. All rights reserved.
%

\section{Technologie}
Sekcja przedstawienia warstwę technologiczną na którą składają się wspierane platformy, użyte języki programowania oraz biblioteki zewnętrzne.
%
%  Copyright © 2022 Mateusz Stompór. All rights reserved.
%

\subsection{Języki programowania}
Kod źródłowy biblioteki składa się z ponad dziesięciu tysięcy linii kodu.
Obsługa grafiki, przez wzgląd na użycie programów cieniujących, narzuca konieczność skorzystania z więcej niż jednego języka programowania.
W projekcie wykorzystano trzy \longpause Swift, C oraz MSL.
\par
Większość biblioteki \longpause 80\% \longpause oparta jest o język Swift w wersji piątej.
Stworzona jest w nim cała zawartość przeznaczona do wykonywania na CPU.
Za jego pośrednictwem udostępniony jest interfejs zewnętrzny.
\par
Shadery, będące odpowiedzialne za 10\% udział, wykorzystują Metal Shading Language.
W zależności o zastosowania są to programy cieniujące dla wierzchołków, fragmentów i obliczeniowe.
Jest to część wewnętrzna, poza zasięgiem użytkownika.
\par
GPU daję możliwość skorzystania z buforów danych oraz tekstur.
Muszą one dowiązane być przy użyciu indeksu, współdzielonego pomiędzy kartą graficzną a procesorem.
Z tego względu, by uniknąć pomyłek, w bibliotece znajduje się pewna ilość plików nagłówkowych zawierająca stałe w języku C.
Zdefiniowano tam indeksy dowiązań dla poszczególnych programów.
Wyeksportowane są one do języka Swift za pomocą modułu.
MSL kompatybilny jest z C, dzięki czemu możliwe jest zapewnienie pomostu pomiędzy obiema technologiami.
Podobny mechanizm zastosowano dla struktur danych współdzielonych pomiędzy GPU oraz CPU.
Stanowi to drugi, ostatni pomocniczy moduł.
%
%  Copyright © 2022 Mateusz Stompór. All rights reserved.
%

\subsection{Biblioteki zewnętrzne}
Wykonanie biblioteki wiązało się z koniecznością użycia szeregu rozwiązań autorstwa osób trzecich.
Wpłynęły one na prędkość rozwoju, jednak zachowały pomocniczy charakter i nie realizowały celu zbieżnego z tym postawionym w pracy.
Starano się, aby możliwie duża ilość pochodziła wprost od producenta.
Decyzję motywowano chęcią zapewnienia bezpieczeństwa, stabilności i wsparcia na wypadek wewnętrznego błędu.
Wszystkie są także integralną częścią systemu operacyjnego, więc nie narzuca to konieczności korzystania z zewnętrznych źródeł w celu ich uzyskania.
\par
Na listę kompletnych rozwiązań składają się:
\paragraph*{simd}\longpause skorzystano ze struktur wspomagających algebrę liniową takich jak kwaterniony, wektory, macierze i najczęstsze operacje na nich wykonywane.
\paragraph*{AppKit, UIKit}\longpause użyte zostały w celu zapewnienia użytkownikowi widoku skonfigurowanego uprzednio na potrzeby generowania grafiki za pośrednictwem silnika.
\paragraph*{Metal}\longpause moduł udostępnił zbiór funkcji do interakcji z kartą graficzną.
Za jego pomocą kolejkowane są zadanie do wykonania na GPU i pośredniczy w uzyskiwaniu zasobów.
\paragraph*{Combine}\longpause funkcjonalność przeznaczona jest do programowania reaktywnego. 
Skorzystano z niej na potrzeby odświeżania danych w miejscach gdzie zachodziły zależności pomiędzy stanami obiektów, a także by oddać użytkownikowi możliwość obserwowania zmian z poziomu interfejsu zewnętrznego.
\paragraph*{MetalKit}\longpause biblioteka Metal tworzona była przybierając możliwie jak najmniejszą formę. 
Część funkcji niebędących koniecznymi do sprawowania jej działania przeniesiona została do osobnego modułu gromadzącego dodatkowe narzędzia.
Użyto między innymi metod to wczytywania tekstur z plików binarnych czy uprzednio przygotowanego widoku, mogącego być bezpośrednio osadzonym w aplikacji.
\paragraph*{ModelIO}\longpause zakres pracy nie objął obsługi poszczególnych formatów plików zawierających dane na temat sceny.
Użyto gotowego rozwiązania opublikowanego przez producenta. 
Obsługuje ono szeroki zakres wspieranych typów, między innymi \textit{fbx}, \textit{obj}, \textit{usdz}, \textit{dae}.
Działa na zasadzie fasady i konwertuje binarną reprezentację na jednolity format, dostępny z poziomu języka programowania.
\paragraph*{CoreGraphics}\longpause biblioteka zapewnia stosunkowo niskopoziomowy dostęp do operowania na grafice dwuwymiarowej.
Jej użycie ograniczało się do wykorzystania struktur koniecznych do wyznaczenia obszaru renderowania czy punktu, z którym użytkownik dokonał interakcji za pomocą wskaźnika lub dotyku.
\paragraph*{MetalPerformanceShaders}\longpause Apple opracowało w formie gotowych programów cieniujących listę najczęściej używanych efektów graficznych.
Zdecydowano się nie duplikować istniejącej funkcjonalności i posłużyć nimi do wsparcia efektów postprocesowych.
\paragraph*{Swift Standard Library}\longpause poza szeregiem modułów przygotowanych przez producenta specjalnie dla ekosystemu Apple skorzystano także z biblioteki standardowej języka Swift.
Wewnątrz niej znajdują się definicje typów podstawowych jak liczby całkowite czy zmiennoprzecinkowe a także struktury takie jak tablica czy słownik.
\paragraph*{Foundation}\longpause funkcje systemowe nie wchodzą w skład funkcjonalności dostępnych wraz z językiem. 
Obsługa daty, wyrażeń regularnych czy dostępu do plików wewnątrz archiwum biblioteki to wybrane przykłady z których skorzystano.
\paragraph*{XCTest}\longpause część kodu weryfikowana jest pod kątem poprawności za pomocą przypadków testowych.
Walidację wyników oparto o funkcje zaimplementowane w stworzonym pod tym kątem produkcie.
Dodatkowo, integracja z IDE XCode zapewniła wgląd do procentowego pokrycia kodu źródłowego przez testy.
\subsection{Wspierane platformy}
Projekt przeznaczony jest wyłącznie na urządzenia z ekosystemu Apple.
Składa się na to macOS, iOS oraz tvOS, w wersjach 11.5, 15.0, 16.1 odpowiednia dla każdego wariantu.
\par
Ze wsparcia zdecydowano się wyłączyć inteligentne zegarki z systemem watchOS.
Producent nie udostępnia dla nich interfejsu OpenGL, ani Metal.
Nie oznacza to, że nie ma możliwości wsparcia trójwymiarowej grafiki.
Producent posiada kilka aplikacji, które ją używają.
Dla programistów jednak udostępniono tylko i wyłącznie moduły będące w stanie generować dwuwymiarowe kompozycje.
W teorii za pomocą odpowiednich przekształceń istnieje możliwość uzyskania trójwymiaru.
Jednak przez wzgląd na silne użycie CPU wydajność takiego rozwiązania byłaby niezadowalająca.
Bezpośrednio wpłynęło to na decyzję o odrzuceniu platformy spośród docelowych systemów.
\section{Struktura Projektu}
Starano się, aby projekt był możliwie jak najprzystępniejszy do analizy.
Z tego powodu dokonano jego ustrukturyzowania.
Przekłada się to na organizację plików wewnątrz katalogów.
Ponadto, ustandaryzowano styl i podejście programistyczne zastosowane w kodzie źródłowym.

\subsection{Organizacja źródeł}
Rozwijanie wielu funkcjonalności projektu wiązało się z eksperymentowaniem.
Kod modyfikowano w celu wykonania prototypu i zweryfikowania możliwości.
Wpływało to przejściowo na stabilność, niekiedy zmiany ostatecznie zostawały wycofywane.
Manualna obsługa scenariusza była pracochłonna.
Narzędziem pomagającym w tej sytuacji był system kontroli wersji.
Zdecydowano skorzystać się z wiodącego na rynku - git.
Dzięki niemu zadbano o stabilność głównej gałęzi rozwoju.
Dodatkowo, funkcjonalności budowane w pobocznych gałęziach, pozostawały tam do czasu stabilizacji.
Wcielano je do głównego strumienia projektu po upewnieniu się, że nie powodują regresji.
\par
Na potrzeby pracy wykonano dwa projekty \longpause Silnik graficzny oraz grę szachową.
Starano się, aby możliwie uniezależnić je od siebie.
Z tego względu stworzona na ich potrzeby dwa osobne repozytoria.
Biblioteka graficzna jest niezależna, natomiast gra szachowa, która posiada na niej zależność używa funkcjonalności submodułów w celu pobrania źródeł.
Dzięki temu uniknięto duplikacji i zadbano, aby repozytoria były możliwie niewielkie.
\par
Komplementując zarządzanie wersją zdecydowano się na integrację z platformą GitHub.
Za jej pośrednictwem utworzono organizację do której podpięto repozytoria z projektami~\cite{porcelain_project}.
Dzięki temu zapewniono bezpieczeństwo na wypadek utraty danych z lokalnego dysku.
Otworzyło to także szereg możliwości zarządzania zadaniami, automatyzacji i zapewnienia jakości.
Aspekty te, podobnie jak gra szachowa omówione zostaną w dalszej części pracy.
Sekcja koncentrowała będzie się na stworzonej bibliotece graficznej.
\subsubsection{Hierarchia}
\listdirs{[Engine[Core]
                 [MetalBinding]
                 [Shaders]]
          [EngineTests[Engine][Extensions]]
          [...]
}
\linebreak
Repozytorium projektu posiada dwa główne katalogi.
\textit{Engine} zawiera w sobie właściwy kod projektu.
Funkcjonalności operate o język Swift przynależą do \textit{Core}.
\textit{MetalBinding} stanowi pomost pomiędzy programami cieniującymi, a wysokopoziomowym Swift.
Wewnątrz niego zorganizowane są pliki nagłówkowe języka C.
\textit{Shaders} z kolei jest miejscem w którym zgromadzono programy cieniujące.
Wewnątrz \textit{EngineTests} znajduje się katalog z dokładnym odwzorowaniem plików biblioteki silnika z przypadkami testowymi, a dodatkowo rozszerzenie funkcjonalności modułu do testowania.

\listdirs{[Core[UI]
               [Aliases]
               [Scene]
               [Animation]
               [Loaders]
               [Extensions]
               [Configuration]
               [Rendering]
               [...]]
}

Rdzeń projektu agreguje wiele komponentów.
Wśród nich wyróżnić można logiczny podział na części odpowiedzialne za interfejs użytkownika czy służące do konfiguracji projektu.
Najistotniejsze jednak są te odpowiedzialne za zarządzanie sceną, animacją i renderowaniem.
Ponadto chcąc zapewnić możliwie wysoką czytelność skorzystano z szeregu alternatywnych odwołań do typów.
Użyto także możliwości języka Swift, która pozwala rozszerzać istniejące funkcjonalności.
W zależności od modułu, do którego się odnosiły umieszczono je w odpowiednich podkatalogach.
\subsection{Konwencje}
Decyzja o wybraniu formy biblioteki wpłynęła na cele, które starano się osiągnąć.
Jednym z nich było uzyskanie przystępności i możliwie niewielkiego wysiłku potrzebnego do zrozumienia kodu przez osoby trzecie.
Pomimo charakteru badawczego pracy brano pod uwagę możliwość chęci kontrybucji ze strony społeczności.
Istnieją narzędzia, które na podstawie statycznej analizy są w stanie wskazać błędy przez które zadanie będzie utrudnione.
Od pewnego progu jednak czytelność postrzegana jest subiektywnie.
Zdecydowano się wprowadzić standard w odniesieniu do tworzonych plików i możliwie zachować go dla całego kodu źródłowego.
\subsubsection{Zawartość plików}
Pliki w projekcie podzielone są w taki sposób, aby zawartość w nich osadzona odnosiła się do pojedynczego komponentu.
Negatywnie wpływa to na ich ilość \longpause projekt posiada około 350.
Pozwala jednak zwiększyć ziarnistość podczas organizacji w katalogi, dzięki czemu precyzyjniej są one zgrupowane.
Dodatkowo, tak jak w przypadku innych zabiegów miało to ograniczyć narzut konieczny do zapoznania się z kodem.
\par
Pliki Swift posiadają w sobie pojedyncze klasy, struktury czy protokoły.
Zgrupowanie wielu definicji występuje jedynie w przypadku rozszerzeń oraz aliasów.
\par
Programy cieniujące dla pojedynczej jednostki translacyjnej posiadają definicję całego potoku, ale rozdzielono je dla różnych z nich.
Oznacza to, że dla pliku wspólnie zdefiniowane są shadery wierzchołków i fragmentów, a w przypadku programów obliczeniowych definicja występuje pojedynczo.
\par
W przypadku współdzielonego kodu w C przyjęto hybrydową strategię.
Struktury dostępne dla Swift i MSL zdefiniowane są w pojedynczych plikach.
Natomiast stałe odnoszące się do indeksów buforów pogrupowane są wspólnie w zależności od przeznaczenia.
Między sobą występują indeksy dla różnych shaderów fragmentów, ale shadery wierzchołków posiadają definicję gdzie indziej.
\subsubsection{Styl programistyczny}
Zadbano o uspójnienie stylu przyjętego w źródłach.
Określono standard w odniesieniu do formatowania.
Dokonywano jego weryfikacji za pomocą Swiftlint w przypadku kodu języka Swift.
Clang-tidy odpowiadał za źródła MSL oraz C.
Upewniono się także by każdy z plików posiadał stosowny nagłówek informujący o przynależności praw autorskich.
\lstinputlisting[language={Swift}, texcl=true, caption=Styl programistyczny na przykładzie struktury animacji]{code/PNKeyframeAnimation.swift}
\subsubsection{Nazewnictwo}
Biblioteki w ekosystemie Apple używają ustalonych przedrostków do zdefiniowania wprowadzanych typów.
Historycznie wynikało to z ograniczeń Objective-C, gdzie istniała jedyna, globalna przestrzeń nazw.
Wsparcie wsteczne sprawiło, że konwencja utrzymała się.
Z tego względu w projekcie zdecydowano się ją podtrzymać i wybrano przedrostek \textit{PN} do własnego użytku.
Dokonano jednak rozróżnienia ze względu na przeznaczenie danego typu.
W przypadku protokołów i struktur, które nie posiadają zachowania użyta jest podstawowa forma przedrostka.
Klasy bądź struktury posiadające zachowanie lub implementujące interfejsy używają przedrostka \textit{PNI}.
\subsubsection{Aliasy}
Powszechne jest, że jeden typ danych wykorzystywany jest w różnych zastosowaniach.
Przykładowo, liczba zmiennoprzecinkowa może posłużyć do określenia miary kąta, jak i odległości.
Nie zawsze jasny jest zakres, pomimo znajomości zastosowania.
Oczekując kąta jako parametru funkcji nie jest wiadome czy chodzi o formę stopni czy radiany.
Z tego powodu zdecydowano się na wprowadzenie szeregu aliasów, które miały pomóc użytkownikom uniknąć niejednoznaczności.
Określając specyfikację kamery zamiast oczekiwać miary pola widzenia w liczbie zmiennoprzecinkowej oczekuje się podania liczby typu \textit{PNRadians}.
Konwencja ta kontynuowana jest w bardzo wielu miejscach kodu, również by pozwolić autorowi nie zapomnieć o pewnych niuansach.

\clearpage
\section{Architektura}
\cundercontruction
    \subsection{Podejście programistyczne}
    \subsection{Dobór Typów}
    \subsection{Komponenty silnika}
    \subsection{Przepływ danych}
    \subsection{Hermetyzacja}
    \subsection{Współdzielenie kodu}
\clearpage
\section{Scena}
    \subsection{Drzewo sceny}
    \subsection{Właściwości węzła}
    \subsection{Wybrane węzły}
        \subsubsection{Kamera}
        \subsubsection{Transformacja}
        \subsubsection{Model}
        \subsubsection{Światła}
    \subsection{Interakcja}
    \subsection{Techniki animacji}
        \subsubsection{Animacja szkieletowa}
        \subsubsection{Animacja bryły sztywnej}
\clearpage
\section{Renderowanie}
\begin{figure}[H]
    \begin{center}
        \includegraphics[width=15cm]{images/pnengine/toy_drummer.jpg}
    \end{center}
    \caption{Klatka prezentująca wyrenderowany model zabawki}
    \label{fig:toy_drummer}
\end{figure}
\begin{figure}[H]
    \begin{center}
        \includegraphics[width=15cm]{images/pnengine/sponza_scene.jpg}
    \end{center}
    \caption{Scena z popularnego modelu firmy Crytek \longpause Sponza}
    \label{fig:sponza}
\end{figure}
\begin{figure}[H]
    \begin{center}
        \includegraphics[width=15cm]{images/pnengine/bloom.jpg}
    \end{center}
    \caption{Przykład efektu bloom na podstawie klatki stworzonej przez bibliotekę}
    \label{fig:bloom}
\end{figure}
    \subsection{Ogólne podejście}
    \subsection{Potok renderowania}
    \subsection{Zawarte techniki}
        \subsubsection{PBR}
        \subsubsection{Cienie}
        \subsubsection{Mapowanie Normalnych}
        \subsubsection{Przeźroczystość}
        \subsubsection{Efekty cząsteczkowe}
        \subsubsection{Bloom}
        \subsubsection{Environment mapping}
        \subsubsection{SSAO}
        \subsubsection{Postprocessing}
\clearpage
\section{Wczytywanie danych zewnętrznych}
    \subsection{Modele}
    \subsection{Tekstury}
\clearpage
\section{Interfejs programistyczny}
    \subsection{Komponenty publiczne}
    \subsection{Testowalność}
    \subsection{Konfiguracja}
\clearpage
\section{Wydajność Silnika}
\clearpage
\section{Jakość Projektu}
\clearpage
\section{Dokumentacja}
Projekt posiada dokumentację występującą w dwóch formach.
Pierwszą z nich jest opis repozytorium przy użyciu pliku \textit{README}, drugi aspekt, zebrany w formie paczki, dotyczy zaś architektury i interfejsu.
\par
\textit{README} przygotowany jest z myślą o pierwszym kontakcie z potencjalnym odbiorcą.
Nie zakłada on profilu jaki osoba posiada.
W zamyśle krótko podsumowuje kluczowe cechy.
Traktowany może być jako wizytówka projektu.
Opisuje przeznaczenie, funkcjonalności, wymagania oraz proces budowy.
Wyjaśnia w jaki sposób uzyskać dostęp do dokumentacji interfejsu biblioteki w celu wykonania integracji.
Zadbano by dostarczyć kilka klatek w formie grafik poglądowych.
\par
Referencja kodu stworzona została w oparciu o powszechnie stosowany mechanizm.
Archiwum ze stronami dokumentacji generowane jest na podstawie specjalnych komentarzy dołączonych do źródeł.
Zawierają one opisy parametrów wejściowych, wartości zwracanych, a także samego celu stojącego za daną metodą czy klasą.
Objęły one publiczne komponenty, a dzięki dołączeniu dodatkowych plików poruszyły także zagadnienie architektury.
\lstinputlisting[language=Swift, caption=Dokumentacja kodu w języku Swift]{code/docc.swift}
\par
W podstawowym przypadku symbole, które udokumentowano pogrupowane są w zależności od typu - na klasy, interfejsy, struktury, itd.
Zdecydowano jednak wprowadzić się podział na podstawie ich semantyki.
Rozróżnienie występuje w zależności od funkcjonalności do jakiej dany symbol należy.
Struktura dokumentacji przypomina nieco strukturę projektu i wśród wielu innych wymienić można takich zagadnienia jak: zarządzanie sceną, elementy sceny, interfejs graficzny.
Starano się aby możliwie w przypadku problemu symbol, który będzie analizowany przez użytkownika zestawiony był z pokrewnymi sobie.
\par
Wykorzystanym narzędziem, które posłużyło do generowania dokumentacji było DocC.
Jest ono natywne dla platform Apple. 
Stworzone zostało przez producenta i dostarczane jest w ramach środowiska programistycznego XCode.
Narzędzie potrafi wytworzyć paczkę \textit{docarchive}, która następnie możne podlegać dystrybucji i być zaimportowana w innym środowisku XCode.
\par
Największą zaletą DocC jest fakt, że tworzy ono interfejs graficzny spójny z natywnymi bibliotekami systemowymi.
Dzięki temu konsumpcja treści jest bardziej przystępna.
Programiści tworzący natywne aplikacje nie będą mieli problemu, aby odnaleźć się podczas nawigacji.
\par
Dystrybucja dokumentacji przy użyciu pojedynczego pliku \textit{docarchive} może wydawać się wygodna, ma jednak pewne wady.
Wymusza na użytkowniku pobranie paczki, posiadanie środowiska XCode, a także import archiwum.
Z tego względu zdecydowano się na dodatkową formę publikacji.
W tym celu skorzystano z możliwości generowania plików HTML w narzędziu DocC.
Następnie przy użyciu funkcjonalności platformy GitHub \longpause Pages \longpause osadzono dokumentacje w formie strony internetowej.
Dzięki temu dostęp do najnowszej wersji referencji jest natychmiastowy, a w razie braku łączności sieciowej do dyspozycji pozostaje skorzystanie z archiwum \textit{docarchive}.
\begin{figure}[H]
    \begin{center}
        \includegraphics[width=15cm]{images/docs/camera_docs.png}
    \end{center}
    \caption{Dokumentacja na przykładzie klasy kamery}
    \label{fig:camera_docs}
\end{figure}

\section{Infrastruktura projektu}
    \subsection{Zarządzanie zadaniami}
    \subsection{Serwer ciągłej integracji}
\clearpage
\section{Analiza błędów}
    \subsection{Metal image capture}
    \subsection{Memory leaks}
    \subsection{Time profiling}
    \subsection{Własne narzędzia}

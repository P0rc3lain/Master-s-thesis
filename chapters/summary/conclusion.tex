%
%  Copyright © 2022 Mateusz Stompór. All rights reserved.
%

\section{Wnioski}
Celem pracy było stworzenie biblioteki graficznej, czasu rzeczywistego dla platformy sprzętowej Apple.
Zadanie to udało się wykonać, co udokumentowane zostało w kolejnych rozdziałach pracy.
\par
Stworzona biblioteka realizuje szereg najpopularniejszych algorytmów z dziedziny animacji oraz renderowania.
Przeznaczona jest do zastosowania w niewielkich produkcjach z branży gier oraz jako narzędzie, którym można posłużyć się we wstawkach trójwymiarowych aplikacji użytkowej.
Wypełnia ona lukę, którą stworzona została w ekosystemie.
Zapewnia nowoczesny interfejs oparty o luźne powiązania, a dzięki otwartemu kodowi źródłowemu otwarta jest na interakcje ze strony społeczności
\par
Wraz z aplikacją stworzono aplikację demonstracyjną.
Weryfikuje ona możliwości, którymi deklaruje się biblioteka i sprawdza w realnym przypadku użycia.
W tym celu wykorzystano grę szachową, kompletną pod względem logiki z przeznaczeniem na komputery osobiste.
\par
Użytkownicy skorzystać mogą ze stworzonej dokumentacji.
Wyjaśnia ona założenia podjęte w bibliotece oraz podsumowuje przeznaczenie dostępnych w niej komponentów.
Stworzono ją w oparciu o natywne narzędzia dzięki czemu jej nawigacja i szata graficzna identyczna jest z bibliotekami systemowymi.
\par
Rozwijając produkt starano zadbać się możliwie o jego wysoką jakość.
Stworzono szereg testów jednostkowych i wykorzystano lintery weryfikujące poprawność w zakresie języków Swift, Yaml oraz Markdown.
\par
Proces testowania oraz budowania paczek, a także publikacji dokumentacji został całkowicie zautomatyzowany.
Każdorazowa zmiana dodana do głównej gałęzi rozwoju aplikacji skutkuje w aktualizacji publicznej witryny z materiałami oraz wytworzeniem artefaktów do wykorzystania na wszystkich platformach.
Zabezpieczono się także przed regresją dzięki weryfikacji wszystkich \textit{pull requestów} za pomocą każdorazowego uruchomienia testów.
Błąd w potoku uniemożliwia wcielenie zmian.

%
%  Copyright © 2022 Mateusz Stompór. All rights reserved.
%

\section{Propozycje dalszego rozwoju}
Pomimo znacznego czasu poświęconego zarówno na rozwój projektu, jak i jego dokumentację nadal znaleźć można obszary, które można byłoby poddać ulepszeniu.
\par
Jednym z nich jest warstwa graficzna.
Rosnąca moc urządzeń, włączając w to mobilne jednostki sprawia, że część efektów mogłaby zostać oparta o śledzenie promieni.
Dzięki globalnemu wpływowi światła nadać mogłoby to znacznie wyższy realizm.
\par
Spróbować można byłoby także dać użytkownikom możliwość tworzenia własnych programów cieniujących.
Otworzyłoby to możliwość tworzenia własnych efektów specjalnych.
Byłoby to wartościowe zarówno w stosunku do postprocesingu, jak i renderowania siatek.
\par
Ostatnim aspekt, który nasunął się podczas rozwoju pracy dotyczy testowania programów cieniujących.
Jest to zagadnienie niezwykle niszowe.
Na platformę Apple nie jest dostępna żadna biblioteka, która mogłoby pomóc w realizacji tego zadania.
Stworzenie takowej mogłoby pomóc społeczności i przyczynić się do polepszenia jakości także tego projektu.
% Napisać, że w sumie nie ma frameworka do testowania shaderów
% Raytracing
% Wydajność
% Więcej testów
% Obsługa wody
% Efekty cząsteczkowe na GPU
% Wczytywanie konfiga sceny z pliku
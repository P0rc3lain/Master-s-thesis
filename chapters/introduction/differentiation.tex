\section{Podział grafiki}
Wspomniane gałęzie rozwoju \longpause gry komputerowe oraz filmy \longpause kładą nacisk na wykluczające się między sobą czynniki.
Pierwsza z nich skupia się na wydajności koniecznej do zapewnienia interaktywności.
Kluczowym w tym kontekście jest określenie grupy docelowej w znaczeniu możliwości sprzętu.
Produkt dostosowywany jest pod uprzednio zadane zasoby a w sytuacji, gdy dany efekt graficzny skutkuje w spadkach wydajności \longpause jest usuwany.
Z drugiej strony branża filmowa koncentruje się na jakości obrazu.
Odbiorca nie ma wpływu na akcję, przebieg sekwencji wydarzeń każdorazowo wygląda w ten sam sposób.
Restrykcja czasowa, której podlega generowanie kolejnych klatek, jest praktycznie zaniedbywalna.
Wynika to z faktu, że fragmenty animacji tworzone mogą być niezależnie od siebie, równolegle, na wielu maszynach.
W przeciwieństwie do gier komputerowych ich wytwarzanie dokonywane jest dokładnie raz, a następnie tylko podlega odtwarzaniu.
Interaktywność jest cechą, która odróżnia grafikę czasu rzeczywistego od nierzeczywistego i wprowadza główny podział pomiędzy technologiami służącymi do produkcji obrazu.

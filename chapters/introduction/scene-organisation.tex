%
%  Copyright © 2022 Mateusz Stompór. All rights reserved.
%

\section{Organizacja obiektów}
Generowania dwuwymiarowych obrazów za pomocą komputera przypomina proces tworzenia filmów.
Nie dziwi więc fakt, że znaczna część pojęć używanych w odniesieniu do zawartości klatki czerpie właśnie z niego.
Nadrzędnym elementem zawierającym w sobie wszystkie pomniejsze jest scena.
W niej zorganizowani są aktorzy rozumiani jako postacie bezpośrednio mające wpływ na akcje, jak i pozostałe fragmenty scenerii stanowiące tło.
Za uwiecznienie toczących się wydarzeń odpowiedzialne są kamery przedstawiające świat z różnych perspektyw.
W zależności od wizji twórczej ich ilość może być zmienna, zawsze jednak tylko jedna rozpatrywana jest jako aktywna, a więc przekazująca aktualny obraz.
Niezbędnym elementem, koniecznym, aby obserwator był w stanie cokolwiek dostrzec, jest źródło światła.
Zaliczamy do nich emitery, takie jak słońce, księżyc, ogień, czy żarówki elektryczne.
\par Początkowo implementacja idei była bezpośrednia. 
Struktura sceny była płaska, a obiekty od siebie niezależne.
Podejście obarczone było pewnymi negatywnymi konsekwencjami.
Najłatwiej dostrzec je posługując się przykładem.
W tym celu rozważony zostanie fragment krótkiego scenariusza.
\begin{quote} 
    \centering 
    \q{Człowiek porusza się po lesie zbierając jagody do koszyka}
\end{quote}
Realizując animację za pomocą komputera należałoby przygotować kilka modeli \longpause lasu, człowieka, koszyka oraz jagód.
Podział w ten sposób podyktowany jest chęcią sprawienia, aby istniała możliwość użycia modeli w innych konfiguracjach \longpause sceneria pola zamiast lasu, poziomki w miejsce jagód.
Zakładając, że nie istnieje relacja posiadania pomiędzy obiektami ruch jagód, człowieka oraz koszyka byłby od siebie niezależny.
W celu zachowania immersji koniecznym byłoby nanoszenie korekt pozycji jagód i koszyka za każdym razem gdy pozycja człowieka ulegnie zmianie.
Jasnym staje się, że pomiędzy obiektami zachodzi relacja i ruch nadrzędnego elementu \longpause człowieka \longpause powinien mieć wpływ na podrzędne \longpause koszyk, jagody.
W efekcie ruchu człowieka zmiana położenia koszyka, jak i znajdujących się w nim jagód powinna następować automatycznie.
Realizacja tego pomysłu dała początek wynalezionym w latach 90-tych hierarchicznym grafom sceny, które wykorzystywane nadal współcześnie.
\figh{images/scene-graph.png}
     {Graf sceny przedstawiający hierarchię pomiędzy obiektami}
     {fig:scene-graph}
     {8cm}

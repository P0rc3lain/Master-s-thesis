%
%  Copyright © 2022 Mateusz Stompór. All rights reserved.
%

\section{Układy współrzędnych}
Wspominając o dwuwymiarowym układzie kartezjańskim przywołać można obraz pary osi prostopadłych do sobie.
Oś \textit{x} usytuowana jest w taki sposób, że wartości opisujące współrzędne punktu rosną w prawą stronę.
W przypadku osi \textit{y} kolejne, następujące po sobie liczby pną się w górę.
Wprowadzenie kolejnej koordynaty i rozszerzenie układu na potrzeby trójwymiarowości pozostawia pewną dowolność.
Dodatkowa oś \longpause \textit{z} \longpause będąca prostopadłą do dwóch poprzednich może skierowana być \textit{do ekranu} lub \textit{od ekranu}.
W zależności od wybranej opcji mówić można o praworęcznym lub leworęcznym układzie współrzędnych \longpause kciuk oraz palec wskazujący tworzą osie \textit{xy}, zaś środkowy \textit{z}.
Żadna nie przewyższa drugiej w kontekście użycia, natomiast istotnym jest aby w obliczeniach pozostać przy raz wybranej konwencji.
\begin{figure}[H]
    \begin{center}
        \includegraphics[width=10cm]{images/lh-rh-coordinate-systems.png}
        \caption{Wizualizacja idei leworęcznego oraz praworęcznego układu współrzędnych}
    \end{center}
    \label{fig:lh-rh-cs}
\end{figure}
\par
Punkt opisany jest w odniesieniu do pewnego układu współrzędnych.
Ten jednak, jak wyjaśniono w poprzedniej sekcji zagnieżdżony może być w innych.
Część z nich jest na tyle istotna, że nie pozostają anonimowe i powiązane są z konkretną nazwą.
Wyróżnić można kilka najważniejszych:
\begin{figure}[H]
    \begin{center}
        \includegraphics[width=5cm]{images/coordinate-spaces.jpg}    
        \caption{Przedstawienie mnogości układów współrzędnych na scenie}
    \end{center}
    \label{fig:coordinate-spaces}
\end{figure}

\paragraph{Układ styczny \eng{Tangent Space}} Wektory bazowe składają się z wektora normalnego do zadanego wierzchołka oraz rozpinającego przestrzeń tangensa oraz bitangensa.

\begin{figure}[H]
    \begin{center}
        \includegraphics[width=5cm]{images/tangent-space.jpg}
        \caption{Układ styczny opisany na jednym z wierzchołków sfery}
    \end{center}
    \label{fig:tangent-space}
\end{figure}

\paragraph{Układ modelu \eng{Model Space}} Powiązany jest z siatką modelu jako całością bądź jej mniejszymi fragmentami.
Pomimo występującej hierarchii w samym modelu pomniejsze układy nie rozróżniane są za pomocą oddzielnych terminów.
\paragraph{Układ świata \eng{World Space}} Tworzy korzeń hierarchii w odniesieniu do którego opisane są pozostałe układy.
\paragraph{Układ kamery \eng{Camera Space}} Osadza kamerę w początku układu współrzędnych. Z jego poziomu generowana jest klatka obrazu.
\paragraph{Układ przycinania \eng{Clip Space}} Jest bezpośrednim następcą układu kamery, z którego przejście nastąpiło na skutek projekcji.
W przypadku biblioteki Metal jest to leworęczny układ współrzędnych zadany prostopadłościanem o głębokości \textit{1} i powierzchni bocznej opisanej przy pomocy kwadratu o boku \textit{2}.
\begin{figure}[H]
    \begin{center}
        \includegraphics[width=5cm]{images/coordinate-systems/ndc.png}
        \caption{Prostopadłościan opisujący znormalizowany układ współrzędnych w Metal}
    \end{center}
    \label{fig:ndc-metal}
\end{figure}
\paragraph{Układ okna \eng{Window Space}} Stanowi dyskretną przestrzeń dwuwymiarowa stojąca na szczycie w hierarchii generowania grafiki, rozciągająca się na całej powierzchni warstwy użytej do wyświetlenia obrazu.
\begin{figure}[H]
    \begin{center}
        \includegraphics[width=5cm]{images/coordinate-systems/device-viewport.png}
        \caption{Układ współrzędnych reprezentujący ekran urządzenia}
    \end{center}
    \label{fig:device-viewport}
\end{figure}
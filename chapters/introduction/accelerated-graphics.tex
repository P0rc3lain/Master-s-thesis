\section{Grafika akcelerowana sprzętowo}
Posiadając uprzednio zaaranżowaną scenę możliwe jest jej uwiecznienie.
Przekład modeli opisanych w trójwymiarowym układzie współrzędnych na dwuwymiarowy obraz nazywamy renderowaniem.
Początkowo algorytmy realizujące to zadanie w całości oparte były o wykorzystanie procesora CPU.
Charakter operacji matematycznych wykonywanych na poszczególnych modelach występujących w ramach kadru pokazywał jednak, że są one powtarzalne i nie występują między nimi zależności.
Naturalnym krokiem mającym na celu przyspieszenie operacji było stworzenie modułu, który za pomocą c operacji zrealizuje zadanie.
Urządzenia spełniające tę role nazywane są GPU \eng{graphics processing unit}.
Ze względu na fakt, że oddzielone są od procesora, koniczne było także zapewnienie drogi komunikacji pomiędzy programistami, a jednostką.
Pierwsze API \eng{application programming interface} realizowały stosunkowo szeroki zakres funkcjonalności.
Zadanie twórcy sprowadzało się do zapewnienia modeli i rozmieszczenia ich na scenie, kalkulacja wpływu światła na poszczególne obiekty lub zasób materiałów wpływających na charakterystykę pokrywających ich powierzchni spoczywał na module akceleracyjnym.
Choć z początku podejście wydawało się być wygodne bowiem programista nie musiał rozumieć aparatu matematycznego stanowiącego bazę dla technologii, to szybko okazało się, że wysoko poziomowy interfejs jest ograniczeniem.
Artyści nie byli w stanie realizować swoich wizji.
Akceleratory zapewniały możliwość tworzenie grafik tylko i wyłącznie w oparciu o pewien charakter.
Nie istniała możliwość zmiany algorytmu kalkulacji wpływu światła na obiekt czy choćby dodanie filtra na obraz.
W efekcie wiele produkcji z lat początków sprzętowej akceleracji jest podobna do siebie pod względem wyglądu.
Odpowiedzią na to zjawisko było wprowadzenie jawnego potoku renderowania, którego część kroków podlegała modyfikacji.
W szczególności niektóre z nich mogą być w pełni programowane przy użyciu specjalnego języka.

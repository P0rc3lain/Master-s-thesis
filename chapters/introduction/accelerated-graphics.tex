%
%  Copyright © 2022 Mateusz Stompór. All rights reserved.
%

\section{Grafika akcelerowana sprzętowo}
Posiadanie uprzednio zaaranżowanej sceny otwiera możliwość na jej uwiecznienie.
Przekład modeli opisanych w trójwymiarowym układzie współrzędnych na dwuwymiarowy obraz nazwany jest renderowaniem.
Początkowo algorytmy realizujące to zadanie w całości oparte były o wykorzystanie procesora CPU.
Charakter operacji matematycznych wykonywanych na poszczególnych siatkach występujących w ramach kadru pokazywał jednak, że są one powtarzalne i nie występują między nimi zależności.
Naturalnym krokiem mającym na celu przyspieszenie przetwarzania danych było stworzenie modułu realizującego zadanie za pomocą predefiniowanych operacji.
Urządzenia spełniające tę funkcję nazywane są GPU \eng{graphics processing unit}.
Ze względu na fakt, że oddzielone są od procesora, konieczne było także zapewnienie drogi komunikacji pomiędzy programistami a jednostką.
Pierwsze API \eng{application programming interface} realizowały stosunkowo szeroki zakres funkcjonalności.
Zadanie twórcy sprowadzało się do zapewnienia modeli i rozmieszczenia ich na scenie.
Kalkulacja oddziaływania światła na poszczególne obiekty czy zasób materiałów wpływających na charakterystykę pokrywających ich powierzchni spoczywał na module akceleracyjnym.
Podejście postrzegane mogło być jako wygodne.
Minimalizowało potrzebę rozumienia aparatu matematycznego stanowiącego bazę dla technologii.
W miarę upływu czasu okazało się jednak, że wysokopoziomowy interfejs jest ograniczeniem.
Artyści nie byli w stanie realizować swoich wizji.
Akceleratory zapewniały możliwość tworzenia grafik w oparciu o określony charakter.
Nie istniała możliwość zmiany algorytmu.
W efekcie wiele produkcji z lat początków sprzętowej akceleracji jest podobna do siebie.
Odpowiedzią na zjawisko było wprowadzenie jawnego potoku renderowania, którego część kroków podlegała modyfikacji.
W szczególności niektóre z nich mogły być w pełni programowane przy użyciu dedykowanego języka.

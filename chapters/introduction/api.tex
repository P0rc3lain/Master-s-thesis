\section{API}
Niezmiennie od początku istnienia technologii do zakresu nałożonych nań obowiązków zaliczamy przekazywanie polecań wydanych przez programistę za pośrednictwem sterownika karty graficznej do jej procesora.
Współczesne rozwiązania, takie jak Metal, Direct X czy Vulkan pochwalić mogą się szerszym wachlarzem funkcjonalności.
Przede wszystkim wprowadzają dodatkową warstwę abstrakcji.
Choć producenci stosują autorskie architektury to API sprawia, że różnice w ich działaniu mogą być dla twórców oprogramowania zaniedbywalne.
Pełne uspójnienie interfejsu jest niemożliwe, ze względu na występujące sprzętowe rozbieżności, jednak w znacznej mierze dostępne instrukcje stanowiące trzon nie podlegają zmianom.
\par 
Używane obecnie technologie opisywane są jako niskopoziomowe.
W odróżnieniu od koncepcji przypadających na pierwsze lata rozwoju pomysłu współcześnie użytkownicy operują na buforach, programach cieniujących czy kolejkach komend zamiast pierwotnych, wysokopoziomowych rozkazach dotyczących zarządzaniem sceną.
Te, choć nadal obecne są we współczesnych programach, stanowią inwencję programistów korzystających z API w celu budowania łatwiejszego do zrozumienia dlań kodu.
\par
Pomimo różnic pomiędzy dostępnymi API, a także ich wariantami główny schemat wykorzystania pozostaje spójny.
Może zostać opisany następującym algorytmem:
\begin{enumerate}
    \item Inicjacja API
    \item Wczytywanie zewnętrznych zasobów \longpause modele, tekstury, filmy, itp.
    \item Aktualizacja zasobów
    \item Prezentacja
    \item Powtórzenie \textit{2}, \textit{3} oraz \textit{4} do momentu zatrzymania programu
    \item Destrukcja
\end{enumerate}
Ważnym atutem jest możliwość tworzenia kodu wykorzystującego zasoby karty graficznej i testowania go na konsumenckim sprzęcie.
Jeśli jednostka graficzna obsługuje technologię API, wówczas zakładając posiadanie podstawowych narzędzi deweloperskich można jej użyć we własnej produkcji.

% Opisz:
% * wsparcie dla wielu kart graficznych
% * wskaż komendy asynchroniczne
% * programy cieniujące

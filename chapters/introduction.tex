%
%  Copyright © 2022 Mateusz Stompór. All rights reserved.
%

\chapter{Grafika trójwymiarowa}
\section{Początki}

\begin{figure}[H]
    \begin{center}
        \includegraphics[width=6cm]{images/people/william-fetter.jpg}    
    \end{center}
    \caption{William Fetter w momencie gdy pracował dla Boeing Aircraft, 1963}
    \label{fig:william-fetter}
\end{figure}

\par % Wyjaśnienie terminu grafika komputerowa
Powszechnie uznaje się, że termin \q{grafika komputerowa} po raz pierwszy został użyty w 1960 roku przez Williama Fetera, ówczesnego pracownika firmy Boeing.
Pełniąc rolę menadżera w zespole zaawansowanych projektów graficznych otrzymał on zadanie wymyślenia nowego sposobu tworzenia rysunków technicznych za pomocą komputera.
W swoim założeniu miały przedstawiać one różne przedmioty opisane za pomocą punktów w przestrzeni trójwymiarowej z uwzględnieniem zniekształceń perspektywy.
Narzędzie miało nie tylko być w stanie stworzyć pojedynczy rysunek na papierze, ale także sekwencje przedstawiające animacje na taśmie filmowej.
Przekucie idei w działający prototyp zajęło kilka miesięcy. 
Zaprogramowane instrukcje interpretowane były przez autorski aparat matematyczny i przekładane na docelowy materiał za pomocą robotycznego ramienia.
W obrębie firmy rozwiązanie początkowo wykorzystywano do wizualizacji wnętrz kokpitów i pracy nad ich ergonomią.
Ważnym osiągnięciem będącym następstwem do generowania pojedynczych, niepowiązanych ze sobą scen było stworzenie serii rysunków przedstawiających animacje lądującego samolotu.
Pomimo wielu wad, jakimi między innymi były długi czas oczekiwania na wynik czy losowe błędy sprawiające, że rysunek musiał być ponownie wytworzony Fetter dostrzegł potencjał i zdecydował się przedstawić efekt swojej pracy szerszemu gronu odbiorców.
Swój pomysł pokazał światu posługując się sylwetką człowieka, która sięgając ręką do panelu sterowania wewnątrz samolotu ustanowiła początek ery trójwymiarowej grafiki komputerowej jaką znamy dzisiaj.
\par % Pokazanie przykładów i opowiedzenie do czego obecnie może być wykorzystywana
\begin{figure}[H]
    \begin{center}
        \includegraphics[width=6cm]{images/battlezone.jpg}    
        \caption{Battlezone, pierwsza produkcja wykorzystująca grafikę trójwymiarową, 1980}
    \end{center}
    \label{fig:battlezone}
\end{figure}
Technologia, podobnie jak i wiele innych nowatorskich rozwiązań niedługo po ujrzeniu światła dziennego stopniowo zaczęła przechodzić do masowego użytku.
Kamieniem milowym w osi czasu kreślącej rozwój tej dziedziny była publikacja gry Battlezone, która w 1980 roku pozwoliła milionom odbiorców doświadczyć po raz pierwszy animacji przestrzennych brył na płaszczyźnie telewizora.
Nie mniej ważne wydarzenie stanowiła premiera filmu \q{Toy story} z 1995 roku. 
Studio Pixar przewodzone w owym czasie przez Steve'a Jobsa wykonało pełnometrażową produkcję opartą w całości o wykorzystanie technik modelowania komputerowego i generowania cyfrowego obrazu trójwymiarowego.

\begin{figure}[H]
    \begin{center}
        \includegraphics[width=6cm]{images/toy-story.jpeg}    
        \caption{Klatka podglądu oraz finalna z filmu \q{Toy Story}, 1995}
    \end{center}
    \label{fig:toy-story}
\end{figure}

Zapisaniem się w historii i pionierstwem w dziedzinie projektowania pochwalić może się firma Boeing.
Składający się z ponad 3 milionów części samolot serii 777 pokazał pierwszym lotem w 1994 roku, że za pomocą komputerów nie tylko skraca się czas przeznaczony na rozwój, ale także zmniejsza ilość pomyłek i zwiększa bezpieczeństwo~\cite{agile_at_boeing}.
Wymienione wydarzenia stanowiły przełom w swoich branżach.
Nie sposób jednak określić jednoznacznej przynależności trójwymiarowej grafiki komputerowej.
Niewątpliwymi filarami napędzającymi rozwój są przytoczone gry komputerowe, branża filmowa i szeroko rozumiany przemysł projektowania, szacowane na odpowiednio 220~\cite{games_market_share}, 80~\cite{movies_market_share} i 50~\cite{interior_design_market_share} miliardów dolarów w rocznym ujęciu globalnym.

\section{Podział grafiki}
Wspomniane gałęzie rozwoju - gry komputerowe oraz filmy - kładą nacisk na wykluczające się między sobą czynniki.
Pierwsza z nich skupia się na wydajności koniecznej do zapewnienia interaktywności.
Kluczowym jest więc określenie grupy docelowej w znaczeniu możliwości sprzętu.
Produkt dostosowany jest pod uprzednio określone zasoby, w sytuacji gdy dany efekt skutkuje w spadkach wydajności - jest usuwany.
Z drugiej strony branża filmowa kładzie nacisk na jakość obrazu.
Odbiorca nie ma wpływu na akcję, przebieg sekwencji wydarzeń każdorazowo wyglądać będzie w ten sam sposób.
Restrykcja czasowa, której podlega generowanie kolejnych klatek jest praktycznie zaniedbywalna.
W przeciwieństwie do gier komputerowych animacja wytworzona może być raz, a następnie tylko podlegać odtwarzaniu.
Interaktywność jest cechą, która odróżnia grafikę czasu rzeczywistego od nierzeczywistego i wprowadza główne rozróżnienie pomiędzy technologiami służącymi do produkcji obrazu.

\section{Organizacja obiektów}
Generowania dwuwymiarowych obrazów za pomocą komputera przypomina proces tworzenia filmów.
Nie dziwi więc fakt, że znaczna część pojęć używanych w odniesieniu do zawartości klatki czerpie właśnie z niego.
Nadrzędnym elementem zawierającym w sobie wszystkie pomniejsze jest scena.
W niej zorganizowani są aktorzy rozumiani jako postacie bezpośrednio mające wpływ na akcje, jak i pozostałe fragmenty scenerii stanowiące tło.
Za uwiecznienie trwających wydarzeń odpowiedzialne są kamery, które przedstawiają świat z różnych perspektyw.
W zależności od wizji twórczej może być ich różna ilość, zawsze jednak tylko jedna rozpatrywana jest jako aktywna, a więc przekazująca aktualny obraz.
Niezbędnym elementem, który konieczny jest, aby obserwator był w stanie dostrzec cokolwiek jest źródło światła.
Zaliczamy do nich wszystkie emitery, takie jak słońce, księżyc, ogień, czy żarówki elektryczne.
\par Początkowo implementacja idei była bezpośrednia. 
Struktura sceny była płaska, a obiekty były od siebie niezależne.
Podejście obarczone było pewnymi negatywnymi konsekwencjami.
Najłatwiej dostrzec je posługując się przykładem.
W tym celu rozważony zostanie fragment krótkiego scenariusza.
\begin{quote} 
    \centering 
    \q{Człowiek porusza się po lesie zbierając jagody do koszyka}
\end{quote}
Pragnąc zrealizować animację za pomocą komputera należałoby przygotować kilka modeli - lasu, człowieka, koszyk oraz jagody.
Podział w ten sposób podyktowany jest chęcią sprawienia, aby modele były możliwe do użycia w innych konfiguracjach - sceneria pola zamiast lasu.
Dodatkowo, uczynienie ich niezależnymi od siebie sprawia, że modyfikacja ich położenia jest ułatwiona. 
Zakładając, że nie istnieje relacja posiadania pomiędzy obiektami ruch jagód, człowieka oraz koszyka byłby od siebie niezależny.
W celu zachowania immersji koniecznym byłoby nanoszenie korekt pozycji jagód i koszyka za każdym razem gdy pozycja człowieka ulegnie zmianie.
Jasnym staje się, że pomiędzy obiektami zachodzi relacja i ruch nadrzędnego elementu - człowieka - powinien mieć wpływ na podrzędne - koszyk, jagody.
W efekcie ruchu człowieka zmiana położenia koszyka, jak i znajdujących się w nim jagód powinna następować automatycznie.
Realizacja tego pomysłu dała początek hierarchicznym grafom sceny, które wynaleziono w latach 90-tych i wykorzystywane są do dzisiaj.
\begin{figure}[H]
    \begin{center}
        \includegraphics[width=15cm]{images/scene-graph.png}    
        \caption{Graf sceny przedstawiający hierarchię pomiędzy obiektami}
    \end{center}
    \label{fig:scene-graph}
\end{figure}
\section{Grafika akcelerowana sprzętowo}
Posiadając uprzednio zaaranżowaną scenę możliwe jest jej uwiecznienie.
Przekład modeli opisanych w trójwymiarowym układzie współrzędnych na dwuwymiarowy obraz nazywamy renderowaniem.
Początkowo algorytmy realizujące to zadanie w całości oparte były o wykorzystanie procesora CPU.
Charakter operacji matematycznych wykonywanych na poszczególnych modelach występujących w ramach kadru pokazywał jednak, że są one powtarzalne i nie występują między nimi zależności.
Naturalnym krokiem mającym na celu przyspieszenie operacji było stworzenie modułu, który za pomocą predefiniowanych operacji zrealizuje zadanie.
Urządzenia spełniające tę role nazywane są GPU \eng{graphics processing unit}.
Ze względu na fakt, że oddzielone są od procesora, koniczne było także zapewnienie drogi komunikacji pomiędzy programistami, a jednostką.
Pierwsze API \eng{application programming interface} realizowały stosunkowo szeroki zakres funkcjonalności.
Zadanie twórcy sprowadzało się do zapewnienia modeli i rozmieszczenia ich na scenie, kalkulacja wpływu światła na poszczególne obiekty lub zasób materiałów wpływających na charakterystykę pokrywających ich powierzchni spoczywał na module akceleracyjnym.
Choć z początku podejście wydawało się być wygodne bowiem programista nie musiał rozumieć aparatu matematycznego stanowiącego bazę dla technologii, to szybko okazało się, że wysoko poziomowy interfejs jest ograniczeniem.
Artyści nie byli w stanie realizować swoich wizji.
Akceleratory zapewniały możliwość tworzenie grafik tylko i wyłącznie w oparciu o pewien charakter.
Nie istniała możliwość zmiany algorytmu kalkulacji wpływu światła na obiekt czy choćby dodanie filtra na obraz.
W efekcie wiele produkcji z lat początków sprzętowej akceleracji jest podobna do siebie pod względem wyglądu.
Odpowiedzią na to zjawisko było wprowadzenie jawnego potoku renderowania, którego część kroków podlegała modyfikacji.
W szczególności niektóre z nich mogą być w pełni programowane przy użyciu specjalnego języka.

\section{Potok renderowania}
\begin{figure}[H]
    \begin{center}
        \includegraphics[width=15cm]{images/metal-rendering-pipeline.png}    
        \caption{Potok renderowania wykorzystywany w języku Metal, Apple Inc.}
    \end{center}
    \label{fig:metal-rendering-pipeline}
\end{figure}
\section{Współczesne algorytmy renderowania}
\section{Aparat matematyczny}
\section{Koncept biblioteki graficznej}
% Opisz jakie zadanie pełni biblioteka graficzna
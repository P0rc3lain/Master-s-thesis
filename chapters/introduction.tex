%
%  Copyright © 2022 Mateusz Stompór. All rights reserved.
%

\chapter{Grafika trójwymiarowa}
\section{Początki}

\begin{figure}[H]
    \begin{center}
        \includegraphics[width=6cm]{images/people/william-fetter.jpg}    
    \end{center}
    \caption{William Fetter w momencie gdy pracował dla Boeing Aircraft, 1963}
    \label{fig:william-fetter}
\end{figure}

\par % Wyjaśnienie terminu grafika komputerowa
Powszechnie uznaje się, że termin \q{grafika komputerowa} po raz pierwszy został użyty w 1960 roku przez Williama Fetera, ówczesnego pracownika firmy Boeing.
Pełniąc rolę menadżera w zespole zaawansowanych projektów graficznych otrzymał on zadanie wymyślenia nowego sposobu tworzenia rysunków technicznych za pomocą komputera.
W swoim założeniu miały przedstawiać one różne przedmioty opisane za pomocą punktów w przestrzeni trójwymiarowej z uwzględnieniem zniekształceń perspektywy.
Narzędzie miało nie tylko być w stanie stworzyć pojedynczy rysunek na papierze, ale także sekwencje przedstawiające animacje na taśmie filmowej.
Przekucie idei w działający prototyp zajęło kilka miesięcy. 
Zaprogramowane instrukcje interpretowane były przez autorski aparat matematyczny i przekładane na docelowy materiał za pomocą robotycznego ramienia.
W obrębie firmy rozwiązanie początkowo wykorzystywano do wizualizacji wnętrz kokpitów i pracy nad ich ergonomią.
Ważnym osiągnięciem będącym następstwem do generowania pojedynczych, niepowiązanych ze sobą scen było stworzenie serii rysunków przedstawiających animacje lądującego samolotu.
Pomimo wielu wad, jakimi między innymi były długi czas oczekiwania na wynik czy losowe błędy sprawiające, że rysunek musiał być ponownie wytworzony Fetter dostrzegł potencjał i zdecydował się przedstawić efekt swojej pracy szerszemu gronu odbiorców.
Swój pomysł pokazał światu posługując się sylwetką człowieka, która sięgając ręką do panelu sterowania wewnątrz samolotu ustanowiła początek trójwymiarowej grafiki komputerowej jaką znamy dzisiaj.
\par % Pokazanie przykładów i opowiedzenie do czego obecnie może być wykorzystywana
\begin{figure}[H]
    \begin{center}
        \includegraphics[width=6cm]{images/battlezone.jpg}    
        \caption{Battlezone, pierwsza produkcja wykorzystująca grafikę trójwymiarową, 1980}
    \end{center}
    \label{fig:battlezone}
\end{figure}
Trójwymiarowa grafika komputerowa, jak i wiele innych nowatorskich technologii niedługo po ujrzeniu światła dziennego stopniowo zaczęła przechodzić do masowego użytku.
Kamieniem milowym w osi czasu kreślącej rozwój tej dziedziny była publikacja gry Battlezone, która w 1980 roku pozwoliła milionom odbiorców doświadczyć po raz pierwszy animacji przestrzennych brył na płaszczyźnie telewizora.
Nie mniej ważne wydarzenie stanowiła premiera filmu \q{Toy story} z 1995 roku. 
Studio Pixar przewodzone w owym czasie przez Steve'a Jobsa wykonało pełnometrażową produkcję opartą w całości o wykorzystanie technik modelowania komputerowego i generowania cyfrowego obrazu trójwymiarowego.

\begin{figure}[H]
    \begin{center}
        \includegraphics[width=6cm]{images/toy-story.jpeg}    
        \caption{Klatka podglądu oraz finalna z filmu \q{Toy Story}, 1995}
    \end{center}
    \label{fig:toy-story}
\end{figure}

Zapisaniem się w historii i pionierstwem w dziedzinie projektowania w pełni cyfrowego powtórnie pochwalić może się firma Boeing.
Składający się z ponad 3 milionów części samolot serii 777 pokazał swoim pierwszym lotem w 1994 roku, że za pomocą komputerów nie tylko skraca się czas projektowania, ale także zmniejsza ilość pomyłek i zwiększa bezpieczeństwo~\cite{agile_at_boeing}.
Wymienione wydarzenia stanowiły przełom w swoich branżach.
Nie sposób jednak określić jednoznacznej przynależności trójwymiarowej grafiki komputerowej.
Niewątpliwymi filarami są napędzającymi rozwój są gry komputerowe, branża filmowa i przemysł projektowania wnętrz, szacowane na odpowiednio 220~\cite{games_market_share}, 80~\cite{movies_market_share} i 50~\cite{interior_design_market_share} miliardów dolarów w rocznym ujęciu globalnym.

\section{Organizacja obiektów}
Generowania dwuwymiarowych obrazów za pomocą komputera przypomina proces tworzenia filmów.
Nie dziwi więc fakt, że znaczna część pojęć używanych w odniesieniu do zawartości klatki czerpie właśnie z niego.
Nadrzędnym elementem zawierającym w sobie wszystkie pomniejsze jest scena.
W niej zorganizowani są aktorzy rozumiani jako postacie bezpośrednio mające wpływ na akcje, jak i pozostałe fragmenty scenerii stanowiące tło.
Za uwiecznienie trwających wydarzeń odpowiedzialne są kamery, które przedstawiają świat z różnych perspektyw.
W zależności od wizji twórczej może być ich różna ilość, zawsze jednak tylko jedna rozpatrywana jest jako aktywna, a więc przekazująca aktualny obraz.
Niezbędnym elementem, który konieczny jest, aby obserwator był w stanie dostrzec cokolwiek jest źródło światła.
Zaliczamy do nich wszystkie emitery, takie jak słońce, księżyc, ogień, czy żarówki elektryczne.
\par Początkowo implementacja idei była bezpośrednia. 
Struktura sceny była płaska, a obiekty były od siebie niezależne.
Podejście obarczone było pewnymi negatywnymi konsekwencjami.
Najłatwiej dostrzec je posługując się przykładem.
W tym celu rozważony zostanie fragment krótkiego scenariusza.
\begin{quote} 
    \centering 
    \q{Człowiek porusza się po lesie zbierając jagody do koszyka}
\end{quote}
Pragnąc zrealizować animację za pomocą komputera należałoby przygotować kilka modeli - las, człowieka, koszyk oraz jagody.
Zakładając, że nie istnieje relacja posiadania pomiędzy obiektami ruch jagód, człowieka oraz koszyka byłby od siebie niezależny.
W celu zachowania immersji koniecznym byłoby nanoszenie korekt pozycji jagód i koszyka za każdym razem gdy pozycja człowieka ulegnie zmianie.
Jasnym staje się, że pomiędzy obiektami zachodzi relacja i ruch nadrzędnego elementu - człowieka - powinien mieć wpływ na podrzędne - koszyk, jagody.
Realizacja tego pomysłu dała początek hierarchicznym grafom sceny, które wynaleziono w latach 90-tych i wykorzystywane są do dzisiaj.
\begin{figure}[H]
    \begin{center}
        \includegraphics[width=15cm]{images/scene-graph.png}    
        \caption{Graf sceny przedstawiający hierarchię pomiędzy obiektami}
    \end{center}
    \label{fig:scene-graph}
\end{figure}
\section{Potok renderowania}
\begin{figure}[H]
    \begin{center}
        \includegraphics[width=15cm]{images/metal-rendering-pipeline.png}    
        \caption{Potok renderowania wykorzystywany w języku Metal, Apple Inc.}
    \end{center}
    \label{fig:metal-rendering-pipeline}
\end{figure}
\section{Współczesne algorytmy renderowania}
\section{Aparat matematyczny}
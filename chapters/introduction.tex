\chapter{Grafika trójwymiarowa}
\section{Początki}

\begin{figure}[H]
    \begin{center}
        \includegraphics[width=6cm]{images/people/william-fetter.jpg}    
    \end{center}
    \caption{William Fetter w momencie gdy pracował dla Boeing Aircraft, 1963}
    \label{fig:william-fetter}
\end{figure}

\par % Wyjaśnienie terminu grafika komputerowa
Powszechnie uznaje się, że termin \q{grafika komputerowa} po raz pierwszy został użyty w 1960 roku przez Williama Fetera, ówczesnego pracownika firmy Boeing.
Pełniąc rolę menadżera w zespole zaawansowanych projektów graficznych otrzymał on zadanie wymyślenia nowego sposobu tworzenia rysunków technicznych za pomocą komputera.
W swoim założeniu miały przedstawiać one różne przedmioty opisane za pomocą punktów w przestrzeni trójwymiarowej z uwzględnieniem zniekształceń perspektywy.
Narzędzie miało nie tylko być w stanie stworzyć pojedynczy rysunek na papierze, ale także sekwencje przedstawiające animacje na taśmie filmowej.
Przekucie idei w działający prototyp zajęło kilka miesięcy. 
Zaprogramowane instrukcje interpretowane były przez autorski aparat matematyczny i przekładane na docelowy materiał za pomocą robotycznego ramienia.
W obrębie firmy rozwiązanie początkowo wykorzystywano do wizualizacji wnętrz kokpitów i pracy nad ich ergonomią.
Ważnym osiągnięciem będącym następstwem do generowania pojedynczych, niepowiązanych ze sobą scen było stworzenie serii rysunków przedstawiających animacje lądującego samolotu.
Pomimo wielu wad, jakimi między innymi były długi czas oczekiwania na wynik czy losowe błędy sprawiające, że rysunek musiał być ponownie wytworzony Fetter dostrzegł potencjał i zdecydował się przedstawić efekt swojej pracy szerszemu gronu odbiorców.
Swój pomysł pokazał światu posługując się sylwetką człowieka, który sięgając ręką do panelu sterowania wewnątrz samolotu ustanowił początek trójwymiarowej grafiki komputerowej jaką znamy dzisiaj.
\par % Przedstawienie konceptu grafiki komputerowej czasu rzeczywistego
W miarę upływu lat i postępującemu wzrostowi mocy obliczeniowej komputerów coraz krótszy stawał się czas oczekiwania na pojawienie się pożądanej grafiki na wybranym przez człowieka medium.
Osiągalnym wydawało się być generowanie kolejnych fragmentów animacji wystarczająco szybko, aby ludzkie oko postrzegało proces jako płynny, nieprzerwany.
Dziedziną, która wykorzystuje zjawisko jest grafika komputerowa czasu rzeczywistego.
Choć we wspólną definicje wpisuje się zarówno prezentacja responsywnego interfejsu użytkownika, jak i dedukowanie treści obrazu w oparciu o analizę za pomocą algorytmów to najczęściej pojęcie \q{grafiki komputerowej czasu rzeczywistego} odnosi się do zamiany obiektów opisanych w przestrzeni trójwymiarowej na dwuwymiarowe obrazy za pomocą karty graficznej.
Uznaje się, że metryką, którą można posłużyć się aby ocenić czy rozwiązanie spełnia założenia jest liczba klatek wygenerowanych w ciągu sekundy.
Klatka rozumiana powinna być jako pojedynczy obraz dwuwymiarowy przedstawiający punkt animacji, jako próg ilości w ciągu sekundy przyjmuje się zaś 30 klatek.
\par % Pokazanie przykładów i opowiedzenie do czego obecnie może być wykorzystywana
\begin{figure}[H]
    \begin{center}
        \includegraphics[width=6cm]{images/battlezone.jpg}    
        \caption{Battlezone, pierwsza produkcja wykorzystująca grafikę trójwymiarową, 1980}
    \end{center}
    \label{fig:battlezone}
\end{figure}
Trójwymiarowa grafika komputerowa, jak i wiele innych nowatorskich technologii niedługo po ujrzeniu światła dziennego stopniowo zaczęła przechodzić do masowego użytku.
Kamieniem milowym w osi czasu kreślącej rozwój tej dziedziny była publikacja gry Battlezone, która w 1980 roku pozwoliła milionom odbiorców doświadczyć po raz pierwszy animacji przestrzennych brył na płaszczyźnie telewizora.
Nie mniej ważne wydarzenie stanowiła premiera filmu \q{Toy story} z 1995 roku. 
Studio Pixar przewodzone w owym czasie przez Steve'a Jobsa wykonało pełnometrażową produkcję opartą w całości o wykorzystanie technik modelowania komputerowego i generowania cyfrowego obrazu trójwymiarowego.

\begin{figure}[H]
    \begin{center}
        \includegraphics[width=6cm]{images/toy-story.jpeg}    
        \caption{Klatka podglądu oraz finalna z filmu \q{Toy Story}, 1995}
    \end{center}
    \label{fig:toy-story}
\end{figure}

Zapisaniem się w historii i pionierstwem w dziedzinie projektowania w pełni cyfrowego powtórnie pochwalić może się firma Boeing.
Składający się z ponad 3 milionów części samolot serii 777 pokazał swoim pierwszym lotem w 1994 roku, że za pomocą komputerów nie tylko skraca się czas projektowania, ale także zmniejsza ilość pomyłek i zwiększa bezpieczeństwo~\cite{agile_at_boeing}.
Wymienione wydarzenia stanowiły przełom w swoich branżach.
Nie sposób jednak określić jednoznacznej przynależności trójwymiarowej grafiki komputerowej.
Niewątpliwymi filarami są napędzającymi rozwój są gry komputerowe, branża filmowa i przemysł projektowania wnętrz, szacowane na odpowiednio 220~\cite{games_market_share}, 80~\cite{movies_market_share} i 50~\cite{interior_design_market_share} miliardów dolarów w rocznym ujęciu globalnym.

\section{Techniki renderowania}

\section{Techniki cieniowania}
% Przedstaw trzy dostępne techniki - raytracing, scanline, raycasting

% Zasygnalizuj, że do dnia dzisiejszego wszystkie są wykorzystywane

\section{Organizacja obiektów}

\section{Potok renderowania}
\section{Współczesne algorytmy renderowania}
\section{Aparat matematyczny}
%
%  Copyright © 2022 Mateusz Stompór. All rights reserved.
%

\chapter{Ekosystem Apple}
\section{Relacja pomiędzy platformami}
% Mozna odwołać się do tego źródła -- https://www.quora.com/What-are-similarities-and-differences-between-iOS-and-macOS-development
% W skrócie serwisy i wiele frameworków jest po prostu cross platformowych
% Same koncept programowania ui są do siebie podobne -- controlery, widoki, nawet okna mają taką samą relacje.
% Co je bardzo różni to podjeście do projektowania ui -- w ios ekran jest stosunkowo mały, a więc interfejs musi być większy, w maku wygląda to zupełnie inaczej. W zależności od tego co chcem zrobić to bardzo przeszkadza lub wcale
\section{Środowisko deweloperskie}
\figh{images/xcode/swift-ui.png}
     {Edytor XCode na przykładzie creatora interfejsu w technologii SwiftUI}
     {fig:xcode-swiftui}
     {15cm}
\figh{images/xcode/templates.png}
     {Przykłady szablonów projektu w programie XCode}
     {fig:xcode-templates}
     {10cm}
\section{Przykładowy projekt}
\section{Formy bibliotek zewnętrznych}
\section{Dystrybucja}
% Napisać tutaj, że w celu dsystrybucji programu musimy posiadać opłacone konto certyfikat -- bo będzie nam potrzebny certyfikat do podpisania aplikacji, inaczej nie będziemy jej w stanie zbudować dla nikogo innego niż my sami 

% Napisać, że programy na maka są tworzone jako pliki .app, dla iphone .ipa 
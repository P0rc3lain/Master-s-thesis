\section*{Struktura pracy}
Praca podzielona jest na pięć rozdziałów. 
W idei ma to pomóc czytelnikowi płynnie przejść od nakreślenia podstaw, przez zrozumienia rynku, dostępne rozwiązania do ich analizy i refleksji nad możliwościami.
Pozwoli to lepiej oddać problem, który postawiono w pracy i cel będący reakcją na braki istniejące we współcześnie używanych produktach.
\par
Pierwszy rozdział pełni rolę wprowadzenia do grafiki komputerowej.
Przybliża pierwsze kroki i stopniową ewolucję dziedziny.
Charakteryzuje główne nurty widoczne w branży, pojawiające się na przestrzeni czasu.
Stara się także przekazać abstrakcje, na których operują biblioteki graficzne i wykorzystywane przez nie akcelatory sprzętowe.
Wyjaśnia przebieg potoku renderowania i etapów, które wchodzą w jego skład.
\par
Kolejny rozdział stanowi przejście do właściwiej części pracy.
Opisuje kolejno źródła użyte do zgromadzenia wiedzy.
Przedstawia obecne rozwiązania zebrane na podstawie przeglądu i sposób ich klasyfikacji, dokonując przy tym porównania.
Zagłębia się także w technologie dostępne w ekosystemie Apple, służące zarówno do tworzenia aplikacji, jak i używane do interakcji z procesorem graficznym.
Podjęte kroki pozwolą obnażyć braki występujących na rynku produktów, jednocześnie argumentując sens rozwijania pracy.
\par
Dalszy rozdział odpowiada za opis najistotniejszej części.
Przedstawiona zostanie w nim autorska wizja współczesnej biblioteki graficznej.
Omówienie dotknie najważniejszych aspektów z punktu widzenia użytkownika, takich jak forma dystrybucji, interfejs programistyczny, a także dokumentacja projektu.
W celu oddania pełnego obrazu wyjaśniona zostanie architektura w formie interakcji między komponentami, a także sam potok renderowania wykonywany na GPU. 
Naturalnie, zadaniem bibliotek jest rozwiązywanie pewnych problemów, więc pokazane zostaną możliwości funkcjonalne.
Wreszcie rozprawa podejmie tematykę jakości, optymalizację procesów i potrzebnej do niej automatyzacji.
\par
Czwarty rozdział demonstruje przykład użycia.
W tym celu stworzona została natywna aplikacja macOS będąca kompletną implementacją gry szachowej.
Stanowi to formę weryfikacji możliwości projektu zarówno pod kątem funkcjonalnym, wydajnościowym, jak i jakościowym.
\par 
Ostatni rozdział ma charakter podsumowujący.
Subiektywnie oceni czy owoc wykonanej pracy spełnił pierwotne założenia.
Przemyślenia w głównej mierze oparte będą o wnioski, które nasunęły się podczas rozwoju aplikacji demonstracyjnej, płynące z odczuć na podstawie bezpośredniego użycia.
Będzie to jednocześnie moment na krytyczne spojrzenie.
Ponownie rozważony zostanie dobór technologii, decyzji architektonicznych, by następnie wskazać wszelkie uchybienia i sugestie co do dalszego kierunku rozwoju.

%
%  Copyright © 2022 Mateusz Stompór. All rights reserved.
%

\section*{Struktura pracy}
Praca podzielona jest na pięć rozdziałów. 
W idei ma to pomóc czytelnikowi płynnie przejść od nakreślenia najważniejszych konceptów, przez zrozumienie rynku, dostępne rozwiązania, do ich analizy i refleksji nad możliwościami.
Pozwoli to lepiej oddać sedno postawionego problemu i cel będący reakcją na braki istniejące we współcześnie używanych produktach.
\par
Z punktu widzenia celu pracy pierwszy rozdział nie pełni istotnej roli.
Zdecydowano się go zawrzeć, ponieważ stanowi dopełnienie w stosunku do pozostałych.
Nada kontekst opisanym w poźniejszej części rozwiązaniom.
Jego zadaniem jest wprowadzenie do trójwymiarowej grafiki komputerowej.
Przybliża początkowe kroki i stopniową ewolucję dziedziny.
Charakteryzuje główne nurty widoczne w branży, pojawiające się na przestrzeni czasu.
Autor stara się w nim przekazać także abstrakcje, na których operują biblioteki graficzne i wykorzystywane przez nie akcelatory sprzętowe.
Wyjaśnia przebieg potoku renderowania i etapów, które wchodzą w jego skład.
\par
Kolejny rozdział jest przejściem do właściwiej części pracy.
Opisano w nim kolejno źródła użyte do zgromadzenia wiedzy.
Przedstawiono zabrane na podstawie przeglądu rozwiązania i sposób ich klasyfikacji, dokonując przy tym porównania.
W jego ramach autor zagłębia się także w technologie dostępne w ekosystemie Apple, służące zarówno do tworzenia aplikacji, jak i używane do interakcji z procesorem graficznym.
Podjęte kroki pozwolą obnażyć braki występujących na rynku produktów, jednocześnie argumentując sens rozwijania pracy.
\par
W dalszym rozdział zawarto opis najistotniejszej części.
Przedstawiona zostanie w nim autorska wizja współczesnej biblioteki graficznej.
Omówienie dotknie najważniejszych aspektów z punktu widzenia użytkownika, takich jak forma dystrybucji, interfejs programistyczny, a także dokumentacja projektu oraz kodu źródłowego.
W celu oddania pełnego obrazu wyjaśniona zostanie architektura w formie interakcji między komponentami, a także sam potok renderowania wykonywany na GPU. 
Po prezentacji możliwości funkcjonalnych podjęta zostanie tematyka jakości, optymalizacji procesów i koniecznej dla niej automatyzacji.
\par
Czwarty rozdział zawiera demonstrację przykładu użycia.
Natywna aplikacja macOS, będąca kompletną implementacją gry szachowej, użyta zostanie jako źródło do pozyskania przykładów rzeczywistego wykorzystania biblioteki.
Stanowi to formę weryfikacji możliwości pod kątem funkcjonalnym, wydajnościowym, jak i jakościowym.
\par 
Ostatni rozdział ma charakter podsumowujący.
Jego celem jest wskazanie odpowiedzi na pytanie czy spełniono pierwotnie przyjęte założenia.
Przemyślenia w głównej mierze oparte będą o wnioski, które nasunęły się autorowi podczas rozwoju aplikacji demonstracyjnej.
W formie refleksji oceniony zostanie dobór technologii, decyzji architektonicznych, by następnie wskazać wszelkie uchybienia i sugestie co do dalszego kierunku rozwoju.

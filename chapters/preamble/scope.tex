%
%  Copyright © 2022 Mateusz Stompór. All rights reserved.
%

\section*{Zakres pracy}
Istotną częścią składająca się na pracę jest analiza rynku.
Spośród wielu rozwiązań, mających za zadanie realizować podobne funkcjonalności wybrano najpopularniejsze, dokonując ich charakteryzacji.
Wraz z przeglądem dostępnej literatury oraz sugestii programistycznych kierowanych bezpośrednio przez producenta dostrzeżono niszę, która zyskać mogłaby dzięki dedykowanemu rozwiązaniu.
W oparciu o zebrane wnioski, bazując na nowinkach technicznych zaprojektowano nowe, autorskie narzędzie.
Następnie dokonano jego implementacji.
\par
Celem weryfikacji użyteczności zdecydowano się zaprogramować aplikację demo, w formie gry szachowej, opartą w pełni o stworzoną bibliotekę.
Zaimplementowaną logikę cechować będzie kompletność pod względem funkcjonalności.
Chęć wytworzenia nowego, zależnego od zbudowaniej biblioteki oprogramowania pozwoli uzyskać refleksje na temat osiągniętego wyniku.
Jednocześnie potencjalne braki ukierunkują dalsze prace.
\par
Ze względu na formę pracy magisterskiej i stosunkowo krótki czas rozwoju zakres przedsięwzięcia jest ograniczony.
Same decyzja o stworzeniu biblioteki implikuje mnogość zagadnień do pokrycia \longpause projektu interfejsu, funkcjonalności, ale także konieczność zapewnienia dokumentacji i jakości.
Podjęta zostanie próba spełnienia każdego ze wspomnianych aspektów na zaawansowanym poziomie.
Prawdopodobnym jest natomiast, że optymalna wydajność i wcielenie wszystkich wymagań spełnianych współcześnie przez biblioteki do generowania grafiki wiązało będzie się z koniecznością dalszego rozwoju.
Niemniej przypadek użycia, który podjęto się  spełnić, wymaga stosunkowo niedużego rozwiązania, co pozytywnie wpłynie na szansę wykonania postawionych celów.

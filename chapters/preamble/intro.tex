%
%  Copyright © 2022 Mateusz Stompór. All rights reserved.
%

Lata siedemdziesiąte dały początek trójwymiarowej grafice komputerowej, jaką znamy dzisiaj.
Wynalazek jeszcze w tym samym dziesięcioleciu ewoluował z prototypu do techniki będącej wykorzystywaną w konsumenckim oprogramowaniu.
Przez lata znajdowano kolejne zastosowania dla technologii, a wzrost mocy obliczeniowej dodatkowo polepszał jej przystępność.
Wśród branż charakteryzujących się największym udziałem wyróżnić można filmową, gier komputerowych, architektoniczną oraz wirtualnej rzeczywistości.
Każda z wymienionych gałęzi pochwalić może się co najmniej kilkoma produktami uwzględniającymi najnowsze nowinki, spełniając przy tym kryterium płynności na szerokiej gamie sprzętu.
W tym kontekście produkt rozumiany powinien być jako biblioteka, pozwalająca za pomocą wysokopoziomowego interfejsu na tworzenie animacji trójwymiarowych scen.
\par
Platformami stanowiącymi trzon uruchomieniowy dla technologii są Windows, Linux, macOS, a także mobilne Android oraz iOS.
W 2023 roku wśród ścisłej czołówki wymienić można te operate o ekosystem Apple, między innymi ze względu na rewolucję, jaką zapoczątkowały procesory z serii \textit{M} w 2020 roku.
Istniejąca tendencja motywuje do bliższego przyjrzenia się możliwościom platform składających się na ekosystem i zastanowienia nad przypadkami użycia, które są w stanie pokryć biblioteki na nie przeznaczone.
\par
Powierzchowne spojrzenie w kierunku produktów stworzonych z myślą o generowaniu grafiki wzbudzić może wrażenie, że ekosystem jest obfity w rozwiązania do tworzenia animacji trójwymiarowych.
Dokładniejsza analiza pozwala jednak wysnuć, że w środowisku brakuje natywnej biblioteki, która byłaby w stanie posłużyć do tworzenia gier mobilnych, codziennych \eng{casual game}, a także trójwymiarowych wstawek do aplikacji użytkowych.
Natywność w tym kontekście powinna odnosić się nie tylko do samej technologii, na której bazowano, ale także do podejścia architektonicznego zastosowanego w interfejsie biblioteki.
Twórcy, którzy mają potrzebę obsłużyć wspomniany przypadek, zmuszeni są do zaakceptowania wysokiego narzutu silników do gier. 
Obejmuje to konieczność wykorzystania odmiennego języka programowania, a także negatywnie wpływa na zarządzanie projektem przez wzgląd na potrzebę użycia mnogiej liczby środowisk programistycznych.
W przeciwnym razie skorzystać musieli będą z porzuconej przez producenta platformy alternatywy. 
Wpasowującej się wprawdzie w okazjonalne renderowanie, ale odbiegającej od współczesnych standardów, które sam wdraża.

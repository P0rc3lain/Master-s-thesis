%
%  Copyright © 2022 Mateusz Stompór. All rights reserved.
%

\section*{Cel i założenia pracy}
Za cel pracy przyjęto stworzenie natywnej biblioteki do generowania trójwymiarowej grafiki czasu rzeczywistego.
Technologia pomagać będzie w implementacji gry przeznaczonej do krótkiej rozgrywki oraz wzbogacać szeroko rozumiane aplikacje użytkowe o warstwę trójwymiarową.
Wykonana zostanie w sposób innowacyjny, wykorzystujący aktualne techniki z zakresu renderowania grafiki oraz projektowania architektury dla platformy.
Dzięki temu znacząco wyróżni się na tle alternatyw oferując wysoką przystępność.
Gama wspieranych urządzeń obejmie systemy Apple takie jak macOS, iOS oraz tvOS.
Rozwiązanie oparte zostanie o język Swift 5 oraz graficzny interfejs programistyczny GPU Metal 3.
Dystrybucja przybierze formę otwartą, wraz z dokumentacją opisującą założenia w sposobie interakcji.
\par
Renderowanie grafiki udostępnione zostanie za pomocą interfejsu pozwalającego na aranżacje sceny.
Użytkownik w zamyśle komponował będzie hierarchiczny, acykliczny graf skierowany, a elementy odpowiedzialne za węzły reprezentować będą odpowiednio siatki, transformacje, kamery i podlegać będą rozszerzaniu.
Do dyspozycji zostanie oddany także moduł umożliwiający animację szkieletową oraz bryły sztywnej.
Silnik obsługiwał będzie najpopularniejsze rodzaje świateł, takie jak punktowe i kierunkowe.
Model oświetlenia operował będzie na podstawach fizycznych i uwzględniał okluzję ambientową, a także rzucanie cieni przez obiekty blokujące promienie świetlne.
Finalnie, stworzony zostanie importer pozwalający na wygodne wczytywanie modeli oraz generowanie siatek w oparciu o mapy wysokości.
Logika renderowania wraz ze wszystkimi pośrednimi krokami stworzonymi w ramach potoku pozostanie niewidoczna, podlegać będzie jednak konfiguracji.

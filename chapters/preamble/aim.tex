%
%  Copyright © 2022 Mateusz Stompór. All rights reserved.
%

\section*{Cel i założenia projektu}
Za cel pracy przyjęto stworzenie natywnej biblioteki do generowania grafiki czasu rzeczywistego.
Projekt pomagał będzie w implementacji gry przeznaczonej do krótkiej rozgrywki oraz wzbogacał szeroko rozumiane aplikacje użytkowe o warstwę trójwymiarową.
Gama wspieranych urządzeń obejmowała będzie systemy Apple takie jak macOS, iOS oraz tvOS.
Rozwiązanie oparte zostanie o język Swift 5 oraz interfejs programistyczny GPU Metal 3.
Dystrybuowane zostanie w formie otwartej, wraz z dokumentacją opisującą założenia w sposobie interakcji.
Projekt charakteryzował będzie się utylizacją trendów, takich jak wsparcie dla programowania reaktywnego czy zorganizowanie interfejsu wokół podejścia zorientowanego na protokoły.
W założeniu pozwoli to nie tylko nadać charakter podobny do dostępnych bibliotek wykonujących odmienne funkcje, ale wpłynie pozytywnie na testowalność samej biblioteki, jak i rozwiązań ją wykorzystujących.
\par
Renderowanie grafiki udostępnione zostanie za pomocą interfejsu pozwalającego na aranżacje sceny.
Użytkownik w zamyśle komponował będzie hierarchiczny, acykliczny graf skierowany, a elementy odpowiedzialne za węzły reprezentować będą odpowiednio siatki, transformacje, kamery i podlegać będą rozszerzaniu.
Do dyspozycji zostanie oddany także moduł umożliwiający animację szkieletową oraz bryły sztywnej.
Silnik obsługiwał będzie najpopularniejsze rodzaje świateł takie jak punktowe o pełnym oraz ograniczonym kącie świecenia, oraz światło kierunkowe.
Model oświetlenia operował będzie na podstawach fizycznych i uwzględniał okluzję ambientową, a także rzucanie cieni przez obiekty blokujące promienie świetlne.
Finalnie, stworzony zostanie importer pozwalający na wygodne wczytywanie modeli oraz generowanie siatek w oparciu o mapy wysokości.
Logika renderowania wraz ze wszystkimi pośrednimi krokami stworzonymi w ramach potoku pozostanie niewidoczna, podlegać będzie jednak konfiguracji.
\par
Na potrzeby pracy stworzona zostanie aplikacja demo, stanowiąca weryfikację używalności biblioteki.
Przybierze ona formę trójwymiarowej gry szachowej.
Zaimplementowana logika będzie kompletna pod względem funkcjonalności.
Będzie to równocześnie moment na refleksje na temat wyników, które udało się uzyskać i szansa na wskazanie kierunku dalszego rozwoju.

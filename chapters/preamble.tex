%
%  Copyright © 2022 Mateusz Stompór. All rights reserved.
%

\uchapter{Wstęp}
Lata siedemdziesiąte dały początek trójwymiarowej grafice komputerowej, jaką znamy dzisiaj.
Wynalazek jeszcze w tym samym dziesięcioleciu ewoluował z prototypu do techniki będącej wykorzystywaną w konsumenckim oprogramowaniu.
Przez lata znajdowano kolejne zastosowania dla technologii, a wzrost mocy obliczeniowej dodatkowo polepszył przystępność.
Branże charakteryzujące się największym udziałem to kolejno filmowa, gier komputerowych, architektoniczna oraz wirtualnej rzeczywistości.
Każda z wymienionych gałęzi pochwalić może się co najmniej kilkoma produktami uwzględniającymi najnowsze nowinki spełniając przy tym kryterium płynności na szerokiej gamie sprzętu.
W tym kontekście produkt rozumiany powinien być jako biblioteka, pozwalająca za pomocą wysokopoziomowego interfejsu na tworzenie obrazów trójwymiarowych scen.
\par
Platformami stanowiącymi trzon uruchomieniowy dla technologii są Windows, Linux, macOS, a także mobilne Android oraz iOS.
Największym wzrostem popularności wśród użytkowników zadeklarować mogą się te operujące w ekosystemie Apple, między innymi ze względu na rewolucję jaką przyniosły procesory z serii \textit{M} w 2020 roku.
Istniejąca tendencja motywuje do bliższego przyjrzenia się możliwością platform skłądających się na ekosystem i zastanowienia nad przypadkami użycia, które są w stanie pokryć biblioteki na nie dostępne.
\par
Powierzchowne spojrzenie na dostępne produkty stworzone z myślą o generowaniu grafiki sugerować może, że ekosystem jest obfity w rozwiązania.
Dokładniejsza analiza pozwala jednak wysnuć, że w środowisku brakuje natywnej biblioteki, która byłaby w stanie posłużyć do tworzenia prostych gier, a także trójwymiarowych wstawek do aplikacji użytkowych.
Natywność w tym kontekście powinna odnosić się nie tylko do samej technologii, która posłużyła do stworzenia, ale także podejścia architektonicznego do interfejsu biblioteki.
Twórcy, którzy pokryć pragną wspomniany przypadek zmuszeni są zaakceptowania wysokiego narzutu rozwiązania  - obejmuje to konieczność wykorzystania odmiennego języka programowania, a także zaakceptowania znacznie bardziej skompilkowanego zarządzania projektem.
W przeciwnym razie zmuszeni są oni skorzystać z porzuconych przez producenta alternatyw, odbiegających od współczesnych standardów do których przyzwyczajeni są pod względem projektu interfejsu i jego podatności na testowanie.
\section*{Cel i założenia projektu}
% Celem pracy jest stworzenie biblioteki graficznej na platformy


% Napisz, że to jest biblioteka czasu rzeczywistego na środowisko apple
% Napisz czym biblioteka będzie się charakteryzowała - będzie podejmowała trendy obecne w apple, to znaczy protocol oriented programming,
% W idei ma to sprawić, że pozostała część aplikacji będzie testowalna, a same powiązania będą luźne
% dodatkowo, będzie miała interfejs który ułatwiał będzie programowanie reaktywne
% na jej potrzeby przygotowana zostanie aplikacja demo prezentująca działanie i udowadniająca zasadność
% W założeniach będzie miała też spełniać standardy współczesnych proejktów - ma dokumentację, pipeline'y, unit testy
% Napisz o motywacji - scenekit został porzucowny, arkit nie realizuje zadania, potrzeba rozwiązać ten problem
% dodatkowo nawet jeśli chcieć korzystać z scenekita to nie da się go rozwinąć
% \section*{Założenia projektu}
% Tu napisać, że to projekt natywny
% Napisać, że operuje na eksoystemie apple z wyłaczeniem watchos - co wyjaśnione będzie w dalszej części pracy.
% Napisać, że będzie to open source i wykorzystywać będzie swift + metal
% Napisać, że w idei będzie wykorzystywało graf sceny do operowania na tym co jest renderowane, a reszta będzie niewidoczna dla użytkownika
% żeby dobrze wpasować się obecne trendy oparte będzie to o programowanie reaktywne 
\section*{Zakres pracy}
% Napisać, że z uwagi na formę i ilość zaangażowanych osób zakres projektu jest dość ograniczony.
% Decyzja o stworzeniu biblioteki implikuje, że zakres będzie stosunkowo duży - należy zaprojektować interfejs, funkcjonalności, a także pomyśleć nad dokumentają oraz jakością.
% Samo to implikuje, że jest dużo kwestii do rozważenia - z tego względu podjęta będzie próba, żeby wszystko stało na wysokim poziomie, ale może nie być state of the art.
% Napisz, że zostanie przygotowana biblioteka, a poza nim aplikacja demo w formie gry, która będzie miała za zadanie sprawdzić w prawdziwym życiu jak biblioteka sobie radzi
% Natomiast rynek, który praca stara się wypełnić wymaga dość małego rozwiązania, co dobrze wpasowuje się w ograniczenia
\section*{Struktura pracy}
Praca podzielona jest jest na pięć rozdziałów. 
W idei ma to pomóc czytelnikowi płynnie przejść od zrozumienia rynku, przez dostępne rozwiązania do ich analizy i refleksji nad możliwościami.
Pozwoli to lepiej oddać problem, który postawiono w pracy i cel będący reakcją na braki istniejące w współcześnie używanych produktach.
\par
Pierwszy rozdział pełni rolę wprowadzenia do grafiki komputerowej.
Przybliża on nieco jak wyglądała ewolucja dziedziny.
Charakteryzuje główne nurty widoczne w branży, pojawiające się na przestrzeni czasu.
Stara się także przekazać abstrakcje na których operują biblioteki graficzne i wykorzystywane przez nie akcelatory sprzętowe.
Wyjaśnia przebieg potoku renderowania i etapów, które wchodzą w jego skład.
\par
Kolejny rozdział stanowi przejście do właściwiej części pracy.
Opisuje kolejno źródła, które użyto do zgromadzenia wiedzy.
Przedstawia obecne rozwiązania zebrane na podstawie przeglądu i sposób ich klasyfikacji, dokonując przy tym porównania.
Zagłebia się także w technologie dostępne w ekosystemie Apple, służące zarówno do tworzenia aplikacji, jak i używane do interakcji z procesorem graficznym.
Podjęte kroki pozwolą obnażą braki występujących na rynku produktów, jednocześnie argumentując sens rozwijania pracy.
\par
Dalszy rozdział odpowiada za opis najistotniejszej części pracy.
Przedstawiona zostanie w nim autorska wizja współczesnej biblioteki graficznej.
Omówienie dotknie najważniejsze aspekty z punktu widzenia użytkownika takie jak forma dystrybucji, interfejs programistyczny, a także dokumentację projektu.
W celu oddania pełnego obrazu wyjaśniona zostanie architektura w formie interakcji między komponentami, a także sam potok renderowania wykonywany na GPU. 
Naturalnie, zadaniem bibliotek jest rozwiązywanie pewnych problemów, więc pokazane zostaną możliwości funkcjonalne.
Wreszcie rozprawa podejmię tematykę jakości, optymalizację procesów i potrzebnej do niej automatyzacji.
\par
Czwarty rozdział demonstruje przykład użycia.
W tym celu stworzona została natywna aplikacja macOS będąca kompletną implementacją gry szachowej.
Stanowi to formę weryfikacji możliwości projektu zarówno pod kątem funkcjonalnym, wydajnościowym jak i jakościowym.
Przedstawione zostaną także wnioski, które nasunęły się podczas rozwoju.
\par 
Ostatni rozdział podsumowuje pracę.
Ma za zadanie subiektywnie ocenić czy postawione zadanie spełniło pierwotne założenia.
Jest jednocześnie momentem na krytyczne spojrzenie w kierunku wykonanych prac.
Ponownie rozważony zostanie dobór technologii, decyzji architektonicznych by następnie wskazać wszelkie uchybienia i sugestie co do dalszego kierunku rozwoju.

%
%  Copyright © 2022 Mateusz Stompór. All rights reserved.
%

\subsection{C++}
Decyzję o stworzeniu rozszerzenia do języka C, Bjarne Stroustrup \longpause twórca C++ \longpause motywował chęcią otworzenia się na paradygmat programowania obiektowego.
Jego dzieło zachowywało wszystkie dotychczasowe korzyści wynikające z użycia niskopoziomowego języka, ale sprawiało, że programiści byli w stanie otworzyć się na projektowanie elastycznej architektury.
Pomimo upłynięcia blisko 40 lat od publikacji pierwszej wersji standardu C++ dowiódł swojej wartości i do dzisiaj jest jednym z najpopularniejszych języków programowania.
\par
Największą siłą jest mnogość platform na których jest obecny.
Począwszy od telefonów komórkowych, samochodów, kończąc na branży lotniczej i medycznej.
Zastosowania są równie zróżnicowane jak docelowe urządzenia.
Użytkownicy \longpause w postaci firm, ale także i osób prywatnych \longpause wskazują, że w głównej mierze chodzi o niezawodność.
Przez lata kompilatory uzyskały stabilność, a tysiące użytkowników aktywnie uczestniczących w procesie rozwoju pozwalają zachować pewność, nawet w sytuacjach gdzie na szali stoi życie ludzkie.
\par
Długa obecność na rynku niesie za sobą dodatkową wartość.
Pula bibliotek bogata jest w sprawdzone rozwiązania, pokrywające wiele dziedzin.
Ponadto, kompatybilność z C sprawia, że szerokie już grono bibliotek powiększane jest o te stworzone z myślą o pierwowzorze.
\par
Nie jest regułą, że C++ stanowi natywny język dla platformy.
Wręcz to rzadkie zjawisko.
Często programy C++ odpowiedzialne są za wykonywanie tylko i wyłącznie zadań intensywnych obliczeniowo lub ukierunkowanych na niską latencję.
Natomiast same wywołania i odpowiedzi dokonywane są za pośrednictwem języka operującego na wyższym poziomie abstrakcji \longpause takim jak C\#, Java czy Swift.
W razie potrzeby wspomniane interfejsowania może zachodzić także w drugą stronę \longpause z C++ do języka maszynowego.
Możliwość wkomponowania C++ w inne technologie niewątpliwie pozwoliła uniknąć wyparcia z rynku.
\par
Konceptem stojącym za stworzeniem języka C była chęć uzyskania technologii będącej w stanie zastąpić język maszynowy, zachowując przy tym możliwie jak najniższy poziom abstrakcji.
C++ podtrzymuje to podejście.
Dzięki temu mają one niewielki narzut w stosunku do języka maszynowego.
Często, przez wzgląd na wbudowane w kompilator algorytmy do optymalizacji kod C++ jest lepszy niż równoważny odpowiednik napisany przez człowieka, bezpośrednio w formie zrozumiałej dla komputera.
\lstinputlisting[language=C++, caption=Przykład kodu w języku C++]
                {code/code_comparison/example.cpp}

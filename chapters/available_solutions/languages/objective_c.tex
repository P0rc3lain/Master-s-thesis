%
%  Copyright © 2022 Mateusz Stompór. All rights reserved.
%

\subsection{Objective-C}
W przeciwieństwie do C++ Objective-C nie zyskało licznego grona odbiorców.
Głównie za sprawą faktu, że prawa własności intelektualnej przynależały do Apple.
Technologia jest szeroko-stosowana na ich własnych platformach, ale wybrakowana i niemalże nieobecna na innych.
Z punktu widzenia twórców aplikacji użycie i wybór Objective-C często podyktowany był koniecznością, a nie rozważną decyzją podpartą argumentami.
\par
Objective-C charakteryzuje się gorszą wydajnością w porównaniu z C++.
Wynika to z faktu zastosowania podejścia zaczerpniętego ze SmallTalka \longpause wywołanie funkcji wymaga przesyłania wiadomości do obiektu jej z nazwą w postaci łańcucha znakowego i listy argumentów.
\par
Dodatkowo, Objective-C zniechęca swoją składnią.
Sygnatury metod przypominają nieco język naturalny.
Każdy parametr opatrzony jest stosowną etykietą.
Programista w teorii powinien być w stanie lepiej zrozumieć działanie algorytmu.
Skutkuje to jednak długimi wywołaniami metod, a konieczność zamykania każdego z nich w nawiasy kwadratowe sprawia, że niekiedy sekwencja zabiera wiele linii w pliku źródłowym.
\par
Wymienione czynniki nie były jedynymi, które wpłynęły negatywnie na przystępność.
W przeciwieństwie do C czy C++ Objective-C nie posiadał jawnie publikowanych standardów, pomimo faktu iż był rozwijany i podlegał wersjonowaniu.
O nowych funkcjach użytkownicy dowiadywali się wraz z dystrybucją nowszych wersji kompilatora.
Nie pomagał fakt, że dokumentacja na stronie Apple opisująca język opóźniona była o wiele miesięcy, dostępna tylko w jednym języku i napisana technicznie, bez przykładów.

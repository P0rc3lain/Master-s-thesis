%
%  Copyright © 2022 Mateusz Stompór. All rights reserved.
%

\subsection{Objective-C}
Bjarne Stroustrup nie był jedynym, który dostrzegał braki w języku C i pragnął porzucić podejście proceduralne.
We wczesnych latach osiemdziesiątych Tom Love and Brad Cox opracowali własną odpowiedź na ukłon w stronę obiektowości.
Ich dzieło \longpause Objective-C \longpause w pierwszej wersji ukazało się w 1984 roku.
Wybrane zostało przez firmę NeXT jako technologia służąca do stworzenia ich autorskiego systemu operacyjnego NeXTSTEP.
Był on podstawą, użytą przez Apple do rozwijania własnego systemu \longpause znanego dzisiaj pod nazwą macOS \longpause za sprawą przejęcia firmy NeXT w 1996 roku.
\par
W przeciwieństwie do C++ Objective-C nie zyskało licznego grona odbiorców.
Głównie za sprawą faktu, że prawa własności intelektualnej przynależały do Apple.
Technologia jest szeroko-stosowana na ich własnych platformach, ale wybrakowana i niemalże nieobecna na innych.
Z punktu widzenia twórców aplikacji użycie i wybór Objective-C często podyktowany był koniecznością, a nie rozważną decyzją podpartą argumentami.
\par
Objective-C charakteryzuje się gorszą wydajnością w porównaniu z C++.
Wynika to z faktu zastosowania podejścia zaczerpniętego ze SmallTalka \longpause wywołanie funkcji wymaga przesyłania wiadomości do obiektu jej z nazwą w postaci łańcucha znakowego i listy argumentów.
C++ osiąga ten cel wykorzystując tablice funkcji wirtualnych, które do wywołania metody wymagają wykonania mniejszej liczby operacji.
\par
Dodatkowo, Objective-C zniechęca swoją składnią, niespotykaną w żadnym innym języku.
Sygnatury metod przypominają nieco język naturalny.
W przeciwieństwie do C czy C++ każdy parametr opatrzony jest stosowną etykietą.
Programista w teorii powinien być w stanie lepiej zrozumieć działanie algorytmu, jednak podejście takie ma także negatywne skutki.
Przede wszystkim wywołania metod są długie, a konieczność zamykania każdego z nich w nawiasy kwadratowe sprawia, że niekiedy sekwencja zabiera wiele linii w pliku źródłowym.

% Przydałoby się więcej konkretów o Objective-C
% Automatic garbage collector
% Categories
% Można napisać, że Objective-C służy do interfejsowania z ios
% Good support for binary frameworks
% NSObject is part of foundation

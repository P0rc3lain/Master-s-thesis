\section{Wnioski}
Przegląd dostępnych rozwiązań pozwolił zbadać rynek i dostrzec jego mocne strony.
Jednocześnie wskazano niedoskonałości i braki, które motywują do odpowiedzi na istniejące problemy.
Jako wymagający uwagi wskazać można w którym twórca rozważa stworzenie prostej gry lub wizualizacji obiektu 3D wkomponowanej w aplikację użytkową.
Obecne produkty choć niewątpliwie są w stanie dostarczyć obraz wysokiej jakości to narzut w postaci osobnego języka programowania, który należy zastosować, ograniczonego wsparcia dla ekosystemu Apple, czy przestarzały interfejs motywują do użycia autorskiego rozwiązania.
\par
Zdaniem autora odpowiedzią na wspomniane niedogodności byłoby stworzenie biblioteki graficznej generującej grafikę trójwymiarową, ukierunkowaną na wspomniany scenariusz.
W zamyśle powinna ona oparta być o konwencje, którymi obecnie kierują się architekci projektujący nowe biblioteki na platformę.
Obejmuje to między innymi interfejs oparty o luźne powiązania, wsparcie dla programowania reaktywnego, adekwatne nazewnictwo czy rodzaj dystrybucji.
\par
Jednocześnie zachowana powinna być przystępność.
Koniecznym do oddania użytkownikom jest odpowiednia dokumentacja i dostęp za pośrednictwem języków najczęściej stosowanych w ekosystemie.
Wsparte powinny być również wszystkie platformy dostępne, tak aby twórcy mogli wykorzystać produkt zarówno w przypadku urządzeń stacjonarnych, jak i mobilnych.
\par
Oprogramowanie powinno być również dystrybuowane w formie open source.
Pozwoli to zareagować aktywnym użytkownikom na wypadek braku wsparcia ze strony twórcy.
Dodatkowo, sprawi, że kierunek rozwoju wspomagany będzie przez społeczność i umożliwi jej włączenie się do pracy w przypadku napotkanych błędów.
\par
Produkt powinien być oparty o nowoczesne API graficzne - Metal.
Może wygenerować to początkowo nieco większy narzut programistyczny.
Zagwarantuje jednak wyższą wydajność i uniewrażliwi na wypadek usunięcia alternatywnego, starszego API.
% Powinienem wspomnieć, że architektura opara będzie o protocol programing i w głównej mierze będę czerpał z książki game engine architecture
% Jako referencja do obsługi API programistycznego skorzystam z Metal by tutorials
% Sam model oświetlenia będzie zaczerpnięty z disneya
% Samo programowanie języka to od apple
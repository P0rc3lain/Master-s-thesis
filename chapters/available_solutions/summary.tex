\section{Wnioski}
Wykonany na potrzeby rozdziału przegląd pozwolił zbadać rynek i dostrzec jego mocne strony.
Jednocześnie wskazano niedoskonałości i braki.
Stwierdzono, że przypadek użycia w którym twórca rozważa stworzenie prostej gry lub wizualizacji obiektu 3D wkomponowanej w aplikację użytkową mógłby zostać obsłużony lepiej.
Obecne produkty choć niewątpliwie są w stanie dostarczyć obraz wysokiej jakości to narzut w postaci osobnego języka programowania, który należy zastosować, ograniczonego wsparcia dla ekosystemu Apple, czy przestarzałego interfejsu, kłopotliwego w testowaniu motywują do zaproponowania autorskiego rozwiązania.
\par
Odpowiedzią na wspomniane niedogodności byłoby stworzenie biblioteki graficznej, generującej grafikę trójwymiarową, ukierunkowaną na niewielkie produkcje.
W zamyśle powinna ona oparta być o konwencje, którymi obecnie kierują się architekci projektujący nowe biblioteki na potrzeby ekosystemu.
Obejmuje to szereg założeń do których między innymi zaliczyć można interfejs oparty o luźne powiązania, wsparcie dla programowania reaktywnego, adekwatne nazewnictwo czy rodzaj dystrybucji.
\par
Jednocześnie zachowana powinna być przystępność.
Koniecznym do oddania użytkownikom jest odpowiednia dokumentacja i dostęp za pośrednictwem języków najczęściej stosowanych w ekosystemie.
Wsparte powinny być również wszystkie dostępne platformy Apple, tak aby twórcy mogli wykorzystać produkt zarówno w przypadku urządzeń stacjonarnych, jak i mobilnych.
\par
Oprogramowanie powinno być również dystrybuowane w formie open source.
Pozwoli to zareagować aktywnym użytkownikom na wypadek braku wsparcia ze strony twórcy.
Dodatkowo, sprawi, że kierunek rozwoju wspomagany będzie przez społeczność a nawet umożliwi jej aktywny udział.
\par
Produkt powinien być oparty o nowoczesne API graficzne \longpause Metal.
Może wygenerować to początkowo nieco większy narzut programistyczny.
Zagwarantuje jednak wyższą wydajność i uniewrażliwi na wypadek usunięcia alternatywnego, starszego API.
% Powinienem wspomnieć, że architektura opara będzie o protocol programing i w głównej mierze będę czerpał z książki game engine architecture
% Jako referencja do obsługi API programistycznego skorzystam z Metal by tutorials
% Sam model oświetlenia będzie zaczerpnięty z disneya
% Samo programowanie języka to od apple
% Napisać, że na podstawie źródeł może wykreować sobie własny obraz na bibliotekę

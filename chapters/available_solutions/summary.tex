%
%  Copyright © 2022 Mateusz Stompór. All rights reserved.
%

\section{Wnioski}
Wykonany na potrzeby rozdziału przegląd pozwolił zbadać rynek i dostrzec jego mocne strony, jednocześnie wskazując niedoskonałości i braki.
Stwierdzić można, że przypadek użycia w którym twórca rozważa stworzenie gry przeznaczonej do krótkiej rozgrywki lub wizualizacji obiektu 3D wkomponowanej w aplikację użytkową mógłby zostać obsłużony lepiej.
Obecne produkty choć niewątpliwie dostarczają obraz wysokiej jakości to odbywa się to kosztem znacznego narzutu.
\par
Pełnoprawne silniki najlepiej realizują scenariusz stworzenia gry.
Problematyczne jest, że same w sobie są ekosystemami.
Posiadają własne IDE, a platforma docelowa jest tylko szczegółem z punktu widzenia programisty.
Z tego powodu nie sprawdzą się dobrze w sytuacji, gdy produkt planuje wykonać osoba posiadająca doświadczenie pochodzące wyłącznie z ekosystemu Apple.
Decyzja o wykorzystaniu silnika wiązać będzie się z koniecznością nauki całkowicie nowej technologii.
\par
Filament, jako przykład multiplatformowej biblioteki charakteryzuje się doskonałą jakością.
Największą wadą jest wymóg zaprogramowania całej warstwy graficznej w oparciu o język C++.
Dodatkowo, rozwiązanie stworzone jest z myślą o profesjonalistach w dziedzinie.
Wykorzystuje manualne zarządzanie pamięcią i może okazać się przytłaczające w momencie potrzeby integracji niewielkiej sceny trójwymiarowej.
\par
SceneKit stworzony przez Apple z nastawieniem na obsługę ogólnego przypadku generowania grafiki trójwymiarowej jest najlepszym wyborem.
Jego architektura jednakże ma 10 lat i przez wzgląd na brak jej modernizacji nie spełnia wymagań dzisiejszych standardów.
Sama biblioteka, pomimo stałego rozwoju technik renderowania nie jest aktualizowana.
\par
ARKit podejmujący tematykę generowania grafiki wolny jest od bolączek SceneKit.
Jego przeznaczenie jednak silnie ukierunkowane jest w stronę rozszerzonej rzeczywistości.
Wykorzystanie biblioteki do wizualizacji czy gry casualowej jest niemożliwe.
\par
Odpowiedzią na wspomniane niedogodności byłoby stworzenie biblioteki graficznej, generującej grafikę trójwymiarową, ukierunkowaną na niewielkie produkcje.
W zamyśle powinna spełniać szereg kryteriów, które pozwolą wyróżnić ją na tle konkurencji.
\par
Przyjęty język programowania powinien odpowiadać trendom.
Chęć ścisłej integracji z nowoczesnymi aplikacjami sprawia, że jedynym wyborem jest Swift.
\par
Koniecznym do oddania użytkownikom jest odpowiednia dokumentacja. 
Wymagać należy kompletności pod względem merytorycznym i bycia możliwie spójną w warstwie UI/UX z natywnymi bibliotekami. 
\par
Wsparcie obejmować musi wszystkie dostępne platformy Apple, tak aby twórcy mogli wykorzystać produkt zarówno w przypadku urządzeń stacjonarnych, jak i mobilnych.
\par
Oprogramowanie powinno być dystrybuowane w formie open source.
Sprawi to, że kierunek rozwoju wspomagany będzie przez społeczność i umożliwi jej aktywny udział na wypadek napotkania problemu.
\par
Interfejs programistyczny powinien projektowany być w oparciu o programowanie zorientowane protokołowo.
Dodatkowo, uwzględnić należy wsparcie dla reaktywnego podejścia.
\par
Produkt oprzeć należy o nowoczesne API \longpause Metal.
Zagwarantuje wysoką wydajność i uniewrażliwi na wypadek usunięcia alternatywnego, starszego OpenGL.
Dodatkowo, rozbudowane narzędzia do profilowania oraz debugowania wpłynie pozytywnie na efektywność pracy.
% Powinienem wspomnieć, że w głównej mierze będę czerpał z książki game engine architecture
% Jako referencja do obsługi API programistycznego skorzystam z Metal by tutorials
% Sam model oświetlenia będzie zaczerpnięty z disneya
% Samo programowanie języka to od apple
% Napisać, że na podstawie źródeł może wykreować sobie własny obraz na bibliotekę

%
%  Copyright © 2022 Mateusz Stompór. All rights reserved.
%

\section{Materiały źródłowe}
Mając na uwadze współczesne trendy \longpause potęgowanie informacji na skutek globalizacji i faktu, iż każdy jest w stanie stworzyć publikacje \longpause nie sam dostęp do zasobów stanowi problem, a wartościowość źródeł.
Dokonując przeglądu dostępnej literatury wyraźnie dostrzec można szczególną popularność książek wśród odbiorców.
Równie dużym zainteresowaniem i wkładem w rozwój dziedziny pochwalić mogą się publikacje pracowników firm z branży filmowej oraz gier.
Nie są to jednak jedyne kanały dostępne dla współczesnego odbiorcy.
Ostatnie piętnastolecie zaowocowało intensywnym rozwojem platform strumieniowych.
Możliwość ta nie tylko pozwoliła zaangażować się twórcom niezależnym, ale włączyła także instytucje, takie jak uczelnie.
Zainteresowani mogą bezpłatnie skorzystać z nagrań cyklów wykładów, które niegdyś zarezerwowane były tylko dla nielicznych.
\par
Źródła dostępne w formie książek podzielić można na kilka głównych kategorii \longpause jest to literatura poświęcona odpowiednio: matematyce w świecie grafiki oraz implementacji algorytmów, śledzeniu promieni, a także projektowaniu i budowie silników do gier.
Nie można zapomnieć o narzędziach koniecznych są do rozwoju technologii \longpause w tym kontekście odnosi się to do języków programowania, API programistycznych oraz środowisk, takich jak systemy macOS, iOS, Windows, Linux.
\par
Pierwszą grupę, rozważającą aparat matematyczny oraz algorytmy wykorzystywane w generowaniu grafiki, reprezentują takie tytuły jak \textit{3D Math Primer for Graphics and Game Development}~\cite{math_f_dunn_i_parberry}, \textit{Real-Time Rendering}~\cite{real_time_rendering}.
Dostępne są one od wielu lat, posiadają szereg wydań i nieprzerwanie górują w wynikach popularności na platformach aukcyjnych, takich jak eBay czy Amazon.
Choć statystyki sprzedaży nie muszą wiązać się z jakością samego dzieła to wartość wspomnianych tytułów podparta jest mnogimi referencjami w bibliografiach szanowanych publikacji naukowych.
Obie pozycje skierowane są do osób posiadających niewielkie zaplecze inżynieryjne i stworzone przy założeniu, że czytelnik dopiero rozpoczyna zgłębianie tematyki.
Oba z dzieł przybliżają nomenklaturę używaną w branży, koncepcje przyjęte powszechnie, opisują aparat matematyczny, by w końcu przejść do właściwej części jaką jest renderowanie grafiki.
Mając szerszy ogląd nie sposób przeoczyć fakt, że pomimo ciągłej pracy ze strony autorów nie jest to odpowiednia literatura dla odbiorców poszukujących najświeższych nowinek.
Bezsprzecznie zagwarantować mogą silne podstawy i nakreślić idee, jednak chcąc stworzyć technologie aktualną należy odwołać się także do alternatyw.
\par
Warto pamiętać również o literaturze poruszającej kwestie śledzenia promieni. 
Można przytoczyć tutaj między innymi \textit{Ray Tracing Gems}~\cite{ray_tracing_gems} oraz \textit{An Introduction to Ray Tracing}~\cite{ray_tracing_introduction}.
Opisywany w nich zakres jest znacznie szerszy niż współcześnie można wykorzystać w silniku czasu rzeczywistego, natomiast zawarta wiedza konieczna jest do uzyskania rozwiązania hybrydowego.
W szczególności tyczy się to algorytmów generowania odbić światła, rzucania cienia, okluzji ambientowej czy pozornie niezwiązana obsługa interakcji pomiędzy użytkownikiem, a elementami sceny.
\par
Zrozumienie zasad generowania grafiki nie jest jednak wystarczające do stworzenia biblioteki graficznej, ani bardziej obszernego wcielenia \longpause silnika gry.
W tym celu należy zapoznać się z architekturą pozycji z branży rozrywkowej\cite{game_programming_patterns}.
Na rynku dostępne są produkcje stworzone wprost przez pracowników takich studiów jak \textit{Naughty Dog}\cite{game_engine_architecture} czy \textit{Electronic Arts}\cite{game_programming_patterns}, opisujące problemy jakie twórcy mogą doświadczyć wraz z propozycjami ich rozwiązania.
Stanowi to nieodłączny suplement, ponieważ biblioteki graficzne czy też silniki gry mają za zadanie wyabstrahować skomplikowane techniki renderowania i zapewnić użytkownikowi środowisko w którym będzie w stanie realizować wizję artystyczną, bez konieczności rozumienia detali implementacji.
W kwestii interfejsu programistyczny trudno wytypować konkretne tytuły mogące pomóc w procesie projektowania.
Odwołać należy się do ogólnie przyjętych dobrych praktyk programistycznych, co w tym kontekście oznaczać może przejrzysty interfejs i wysoką hermetyzację.
\par 
Kolejnym ważnym elementem są źródła pozwalające przekuć idee związaniem z generowaniem grafiki w program komputerowy.
Przez wzgląd na platformę, oczekiwanie wysokiej wydajności ogranicza wybór do języków Swift, Objective-C oraz C++.
W przypadku C++ oraz Swift liczyć można na literaturę wprost od autorów \cite{cpp_bjorne}\cite{swift_docs}.
Objective-C jest wciaż wspierany i używany powszechnie, jednak nie jest już rozwijany, a zbiór pozycji opisujących jego działanie jest niewielki. 
Pomimo tego pewne z nich zdecydowanie można uznać za solidne \cite{objective_c_kochan}. 
\par
Największy wkład we współczesny rozwój grafiki komputerowej mają publikacje firm zaangażowanych bezpośrednio w dziedzinę z przyczyn komercyjnych.
Organizacje takie jak Crytek, Electronic Arts, Epic Games czy Ubisoft chętnie dzielą się informacjami związanymi z technikami renderowani.
Przykładami mogą być modele oświetlenia w silnikach \textit{Unreal Engine}~\cite{real_shading_ue_4} czy \textit{Frostbite}~\cite{moving_frostbite_to_pbr}.
Nie brakuje także przełomowych algorytmów, takich jak powszechnie używane \textit{SSAO}~\cite{finding_next_gen_cryengine2} stworzone przez Crytek czy {mgła wolumetryczna}~\cite{volumetric_fog} przez Ubisoft.
Niezaprzeczalnie jednak w procesie tym przewodzą giganty takie jak Pixar czy Disney.
Nie tylko opisują oni jakie metody stosują, ale także tworzą sprawozdania podsumowujące prace związane z konkretnymi produkcjami i upubliczniają własne modele scen do pomiaru wydajności.
Wiedza z branży filmowej nie przekłada się bezpośrednio na ogólnie pojęte renderowanie czasu rzeczywistego, jednak w obliczu układów tworzonych współcześnie \longpause wyposażonych w moduły akceleracji śledzenia promieni \longpause podejście to zyskuje na użyteczności.
\par
W grupie źródeł opartej o dystrybucję wideo umieścić można cykle wykładów z uczelni \textit{UC Davis}~\cite{uc_davis_computer_graphics}, \textit{MIT}~\cite{mit_computer_graphics} oraz \textit{Indian Institute of Technology Delhi}~\cite{iiotd_computer_graphics}. 
Są to kursy poruszające ogólną tematykę generowania obrazu.
Najaktualniejszy spośród nich należy do MIT \longpause wykłady publikowane są corocznie.
Zagadnienia są omówione szczegółowo, a w przypadku gdy wykraczają poza przewidziany materiał prelegent informuje z jakiego zakresu należy wiedzę tę rozszerzyć.
Odbiorca nie posiądzie kompletnej wiedzy, ale będzie wiedział jakie kierunki powinien zgłębić.
Pozostałe dwa kursy pomimo upływu wielu lat od publikacji nadal są przydatne.
W szczególności \textit{Ken Joy} w ramach pracy w UC Davis opisuje dokładnie algorytmy wykorzystywane w sprzętowej akceleracji grafiki komputerowej.
Nie jest to wiedza niezbędna dla osoby chcącej zbudować silnik, ale pozwala zrozumieć w pełniej skali jak wygląda przebieg potoku renderowania od strony algorytmicznej.
\par
Równie użyteczne są wystąpienia pojedynczych osób \longpause pracowników z branży gier \longpause opowiadających o podejściu, jakie należy stosować w celu uzyskania aplikacji o optymalnej wydajności.
Przez wzgląd na formę pojedynczego wykładu, nieprzekraczającego godziny czasu, są one zwięzłe i jedynie nakreślają kierunek. 
Z racji, iż branża gier potrzebuje najwyższej klasy wydajności to wzorce projektowe i podejście do programowania aplikacji niejednokrotnie budzą kontrowersje.
Podpierając opinie przykładem można tutaj odwołać się do dwóch nagrań z konwencji \textit{CppCon} górujących w listach popularności platformy YouTube.
Mowa o \eng{Data Oriented Design} przedstawione przez Mike'a Actona~\cite{mike_acton_dod} oraz Stoyana Nikoleva~\cite{stoyan_nikolov_dod}.
\par 
W przypadku platform Apple źródłem, które nie może być pominięte są wydarzenia WWDC~\cite{wwdc_conferences}.
Ich siłą jest możliwość zapoznania się z nowinkami technologicznymi udostępnionymi przez twórców dla użytkowników sprzętu.
Dodatkowo, często stanowią też okazję do skorygowania podejścia używanego przez programistów w celu zapewnienia optymalnej wydajności.
Przybierając przy tym formę zwięzłej prezentacji 20-60 minut.
Niewątpliwym atutem jest dostarczenie w pełni funkcjonalnego kodu źródłowego do samodzielnej analizy.
\par 
Ostatnimi, wartymi do przywołania źródłami są kanały na platformie YouTube twórców niezależnych i pracowników korporacji zajmujących się produkcją gier.
Pierwsi z nich są otwarci na komentarz dotyczący stosowanych technologii w branży, opisując niekiedy krok po kroku jak wyglądać powinna logika działania aplikacji.
Drudzy, w formie dialogu z widzami, chętnie przeprowadzają przegląd kodu lub też rozwijają wątki związane z tworzeniem silników.
Siła tej formy przekazu wynika z interaktywności.
Filmy same w sobie nie posiadają ograniczenia czasowego, nagrywający nie ponosi kosztów związanych z produkcją, a sami widzowie mogą zadawać pytania, które adresowane mogą być w krótkim czasie.
Wśród wielu twórców przywołać można kanały takie jak ThinMatrix~\cite{yt_thin_matrix} czy TheCherno~\cite{yt_the_cherno} ponieważ są one nad wyraz rozpoznawalne wśród entuzjastów grafiki komputerowej.
\par
Podsumowując, globalizacja, otwartość uczelni, a także korporacji sprawia, że dostęp do wartościowych informacji nie stanowi żadnego problemu.
Dziedzina obfituje w wysokiej jakości źródła i wydaje się, że stworzenie rozwiązania bazującego na współczesnych trendach nie powinno stanowić wyzwania.
Problemem będzie natomiast nakład pracy potrzebny do uzyskania dzieła kompletnego pod względem funkcjonalnym oraz sam dobór technologii.

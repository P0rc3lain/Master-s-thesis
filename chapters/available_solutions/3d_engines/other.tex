%
%  Copyright © 2022 Mateusz Stompór. All rights reserved.
%

\subsection{Pozostałe silniki}
% TODO: Jeśli nie mam zamiaru opisywać unreal engine i godot to powinienem tutaj o nich wspomnieć
W celu dokonania rzetelnej oceny rynku dokonano także przeglądu rozwiązań open source udostępnionych za pomocą witryn internetowych github.com oraz gitlab.com.
Kryterium, którego spełnienia oczekiwano była dostępność na wszystkie platformy Apple i możliwość uruchomienia biblioteki na najnowszych systemach operacyjnych.
W rezultacie wyszukiwania znaleziono około dziesięciu repozytoriów, niestety stan żadnego z nich nie pozwolił określić go jako stabilne rozwiązanie do zastosowania produkcyjnego.
Do będących w stosunkowo zaawansowanej formie rozwoju zaliczyć można Satin~\cite{satin_project} oraz SwiftVVD~\cite{swiftvvd_project}.
% TODO: Powinienem pokusić się o napisanie czegoś więcej
\par
Warto wspomnieć, że poza wymienionymi przykładami bibliotek oraz silników istnieje wiele innych alternatyw.
Jako najpopularniejsze można przytoczyć tutaj jmonkey, urho3d, OGRE 3D, Amazon Lumberyard czy panda 3d.
Wszystkie z nich oparte są o podejście open-source.
Były to rozwiązania niegdyś popularne lub stale zwiększające swoje zasięgi, jednak ze względu na koncentrację pracy wokół całego ekosystemu Apple wyłączono je z dalszej analizy.
Głównie za sprawą skierowania jedynie na komputery stacjonarne lub wybrakowane wsparcie spowodowane zamknięciem się firmy Apple na API inne niż Metal.
\par
Sytuacja podobnie wygląda w przypadku silników na potrzeby gier AAA, takich jak CryEngine, Frostbite, RAGE czy Naughty Dog Game Engine.
Wszystkie z nich cieszą się niemałą popularnością i sukcesem w świecie gier.
Niestety są one skierowane tylko i wyłącznie na urządzenia o wysokiej mocy obliczeniowej.
Dodatkowo, część podlega opłatom licencyjnym, zaś inne nie mogą być w ogóle użyte z uwagi na fakt, że ich źródła są niepubliczne, a pliki binarne dystrybuowane są jedynie dla firm posiadających prawa autorskie.
\par
Zdecydowano się na pominięcie opisu popularnych silników Unreal Engine i Godot.
Ich charakterystyka jest podobna do Unity i nie wniosłaby nic do pracy.

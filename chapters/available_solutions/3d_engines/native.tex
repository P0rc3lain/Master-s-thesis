%
%  Copyright © 2022 Mateusz Stompór. All rights reserved.
%

\subsection{Biblioteki natywne}
Alternatywą do multiplatformowych rozwiązań są biblioteki natywne.
Włączyć do tej grupy można produkty dostępne tylko i wyłącznie na pojedynczym systemie lub te obecne na wielu, lecz posiadające wszystkie warianty zbudowane od podstaw.
Pomimo, że wsparcie każdego środowiska z osobna może postrzegane być jako nieuzasadniona powtarzalność to przynosi ona także korzyści.
\par
Jedną z najważniejszych zalet jest możliwość osiągnięcia optymalnej wydajności.
Kod źródłowy dostosowany pod konkretną platformę nie musi posiadać szeregu dodatkowych warstw abstrakcji.
\par
Natywność to również większa intuicyjność wynikającą z możliwości tworzenia interfejsu dostosowanego do przyjętych konwencji.
Pojęcie jest ogólne, składa się na nie nazewnictwo, schemat komunikacji czy hermetyzacja.
\par
Aplikacja spójna technologicznie zyskuje w sytuacji analizy potencjalnych problemów.
W przypadku otwartego kodu źródłowego przyczynić może się to do wzrostu liczby kontrybucji ze strony społeczności.
%
%  Copyright © 2022 Mateusz Stompór. All rights reserved.
%

\subsubsection{Filament}
Android, którego rozwój prowadzi Google opiera się o nieco odmienne podejście na dostarczanie dodatkowych funkcjonalności niż to, które przyjęto w Apple.
Producent aktywnie rozwija system, dostarcza komponenty w oparciu o które można budować aplikacje, ale długi czas nie zapewniał bibliotek ułatwiających pracę w wyspecjalizowanych dziedzinach.
W przypadku chęci stworzenia gry dwu lub trzy wymiarowej sugerowano skorzystanie wprost z API graficznego i własnego rozwiązania.
Zmianę zainicjował wzrost popularności techniki AR po 2014 roku, kiedy urządzenia mobilne były w stanie zaspokoić potrzeby wydajnościowe.
Wiedziano, że bez odgórnego wsparcia wiele aplikacji ominie platformę lub nawet doprowadzi do migracji użytkowników na konkurencyjny iOS.
W 2018 Google wydał ARCore.
Bibliotekę do obsługi AR, a wraz z nią silnik graficzny Filament, stworzony na potrzeby projektu.
\par
Nadrzędnym celem jaki rozszerzona rzeczywistość powinna spełniać jest zapewnienie immersji.
Oznacza to realistyczne odwzorowanie modelu oświetlenia przy zachowaniu podejścia czasu rzeczywistego.
Elementy wzbogacające obraz powinny wtapiać się w resztę otoczenia.
Filament stworzony został dokładnie z myślą o opisanym scenariuszu.
\par
Oferowany jest jako silnik do renderowania.
Dodatkowo, aby ułatwić proces importu modeli posiada moduł do wczytywania.
Są to główne, ale i jedyne funkcjonalności.
\par
Projekt wspiera szereg systemów, jednakże największy nacisk położony jest na Android.
Pomimo iż stworzony został w C++ integracja z Kotlinem \longpause natywnym językiem platformy \longpause stoi na wysokim poziomie.
Użytkownik jest w stanie bezpośrednio wykonywać zapytania do API, bez pośrednich interfejsów.
Sam projekt publicznych części biblioteki pokrywa się z schematami powszechnie przyjętymi na platformie.
\par
Dystrybucja prowadzona jest w oparciu o podejście open source.
Użytkownicy mogą nieodpłatnie wykorzystać Filament do celów komercyjnych i wprowadzać w nim dalsze poprawki.
Dziwi, że Google wprawdzie umiejscowiło projekt pod skrzydłami swojej organizacji, ale zaznacza, że nie jest on oficjalnie wspierany.
\par
Podejście do renderowania oparte o fizyczne odwzorowanie zachowania światła stało sie powszechne.
Zupełnie wyjątkowy jest jednak sposób w jaki zaimplementowano je w bibliotece.
Zadbano, aby każdy konfigurowalny parametr miał fizyczne znaczenie, a dodatkowo oparty był o jednostki SI.
Przykładowo, nie ma możliwości nadania kierunkowemu światłu imitującemu słońce wartości na podstawie czynników koloru czerwonego, zielonego oraz niebieskiego.
W tym wypadku należy posłużyć się temperaturą barwową w Kelvinach oraz natężeniem strumienia świetlnego w lumenach.
Rozwiązanie jest ściśle ukierunkowane na realizm i nie pozwala kreować dowolnego charakteru.
\par
Dokumentacja projektu oparta jest o trzy główne filary.
Pierwszy to szczegółowy opis modelu oświetlenia.
Jego jakość jest wybitna. 
Każdy aspekt renderowania ujęty jest w osobnej sekcji, wyjaśniony słowie wraz przykładami kodu i ilustracjami.
Przytłaczający może wydawać się stopień włożonego wysiłku, który poświęcono by przedstawić zjawiska fizyczne przy pomocy aparatu matematycznego.
Starano oddać się kompletny proces, bez pomijania kroków pośrednich.
Drugi z filarów to funkcjonalne przykładowy użycia.
Jest ich kilka, w formie projektów początkowych dla każdej z platform.
Nie budzą zastrzeżeń, bez problemu można je zmodyfikować i wykorzystać do swoich potrzeb.
Zawodzi dokumentacja kodu oraz interfejsu.
Nie udostępniono użytkownikom jej za pośrednictwem internetu, uzyskać można ją tylko przy budowie aplikacji.
W głównej mierze ogranicza się ona do jednozdaniowych opisów.
\par
Największą wadą z punktu widzenia użytkownika iOS jest wsparcie dla platformy.
Wprawdzie Filament zdolny jest do pracy zarówno na OpenGL, jak i Metal, natomiast tylko do rozszerzenia obsługi o wspomniane API wydaje się ograniczać włożony nakład pracy.
Nie posiada ona dedykowanego interfejsu dla platformy Apple i wszystkie jej funkcjonalności wymagają użycia C++.
Wobec tego, aby skorzystać z biblioteki koniecznym jest oprzeć część aplikacji o Objective-C i zastosować ewentualne interfejsowanie do Swift.
\par
Przystępności nie poprawia też sposób zarządzania pamięcią.
Cały silnik graficzny oparty jest na modelu manualnym, bez udziału inteligentnych wskaźników.
Dotyczy to zarówno zasobów utrzymywanych w pamięci RAM procesora, jak i GPU.
Przenoszenie zasobów na kartę graficzną i zwalnianie jej przestrzeni adresowej wymaga dodatkowych wywołań funkcji.
Nieco burzy to powszechne konwencje C++, które sugerują użycie podejścia RAII.
Decyzja o takim projekcie jest zrozumiała z punktu widzenia osiągnięcia optymalnej wydajności i wykorzystania pamięci.
Zdradza jednak pośrednio grupę odbiorców.
Filament jest silnikiem graficznym zrobionym przez grafików dla grafików.
Przeciętny programista ekosystemu Apple, który chciałby tylko skorzystać z funkcjonalności może mieć trudności w efektywnym wykorzystaniu.
\begin{figure}[H]
    \begin{center}
        \includegraphics[width=7cm]{images/engines/filament/helmet.jpeg}
    \end{center}
    \caption{Przykład klatki wygenerowanej przez silnik Filament}
    \label{fig:filament-helmet}
\end{figure}
% Otrzymuje się osobno binarkę oraz pliki nagłówkowe, nie ma pojedynczego pliku
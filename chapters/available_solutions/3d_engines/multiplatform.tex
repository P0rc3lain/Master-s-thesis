%
%  Copyright © 2022 Mateusz Stompór. All rights reserved.
%

\subsection{Biblioteki multiplatformowe}
Jedną z dwóch możliwości dystrybucji oprogramowania jest forma multiplatformowa.
W założeniu polega ona na stworzeniu jednego rozwiązania, dostępnego na co najmniej dwóch platformach.
Warto podkreślić, że chodzi tutaj o architekturę w której istnieje część wspólna \longpause w rozpatrywanym znaczeniu nie należy uwzględniać oprogramowania realizującego identyczne funkcje, ale za pomocą ponownej, natywnej implementacji.
\par
Bez względu na przeznaczenie wysokopoziomowy projekt zachowuję pewną prawidłowość i ma warstwowy charakter.
Obszar współdzielony tworzony jest w technologii dostępnej na każdej ze wspieranych platform.
Poniżej niego znajduje się strefa specyficzna dla systemu, mająca uwzględniać ewentualne różnice w interakcji z jego zasobami.
Opcjonalnie, ponad nimi usytuowany jest natywny interfejs, poprawiający przystępność \longpause programista systemu Android oczekiwał będzie API w formie biblioteki JAVA lub Kotlin, zaś twórca aplikacji iOS powinien otrzymać dostęp za pośrednictwem języków Swift czy Objective-C.
\fig{images/multiplatform-architecture.png}
    {Demonstracja idei oprogramowania mulitplatformowego}
    {fig:multi-platform}
    {6cm}
\par
Główną motywacją stojącą za budowaniem oprogramowania w sposób multiplatformowy jest chęć zwiększenia grona odbiorców.
Dla końcowego konsumenta szerokie wsparcie to niewątpliwa zaleta, natomiast narzuca to na twórców dodatkową prace związaną z koniecznością uwzględnienia wpływu poszczególnych zmian na wszystkie dostępne warianty.
\par
Wprowadzenie generalizacji wiąże się jednakże z narzutem wydajnościowym.
Każdy istniejący poziom abstrakcji przekłada się na większą ilość wywołań pomiędzy obiektami wewnątrz biblioteki.
Możliwe minimalizowanie zjawiska jest kluczowe, każda strata płynności wynikająca z wzmożonego nakładu obliczeń procesora implikować będzie konieczność obniżenia jakości grafiki celem utrzymania pożądanej prędkości animacji. 
\par
Kolejnym aspektem, który należy wziąć pod uwagę jest sposób integracji z natywnym środowiskiem.
Oznacza to określenie czy interfejs biblioteki dla końcowej platformy uwzględnia powszechnie przyjęte konwencje.
W idealnym przypadku użytkownik powinien być w stanie domyślić się klucza komunikacji na podstawie posiadanych doświadczeń, bez konieczności wspierania się dokumentacją. 
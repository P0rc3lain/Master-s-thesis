%
%  Copyright © 2022 Mateusz Stompór. All rights reserved.
%

\subsubsection{SceneKit}
\figh{images/engines/scene-kit/editor.png}
     {Edytor treści w silniku SceneKit}
     {fig:scenekit-editor}
     {15cm}
\figh{images/engines/scene-kit/example-game.jpg}
     {Przykładowa gra w silniku SceneKit}
     {fig:scenekit-example-game}
     {15cm}
Debiut rozwiązania nastąpił w 2012 roku. 
Za projekt i wykonanie odpowiedzialna jest firma Apple, której urządzenia stanowią jedyną docelową grupę odbiorców oprogramowania.
Początkowo wsparcie ograniczało się do platformy OS X, rozszerzenie dostępności na urządzenia mobilne nadeszło w 2014 roku, a na przestrzeni lat objęło również watchOS oraz tvOS.
\par
Pragnąc zaklasyfikować SceneKit najtrafniej byłoby określić, iż jest to biblioteka graficzna realizująca dodatkowo pewne funkcjonalności silnika do gier.
Obejmuje to możliwość kontroli audio oraz integracja modelu fizycznego, będącego w stanie symulować wzajemny wpływ obiektów.
\par
W przeciwieństwie do wiodących rozwiązań skierowanych do branży gier SceneKit nie posiada własnego IDE, nie pozwala na programowanie wizualne, ani zaawansowane skryptowanie.
Jedynym udogodnieniem jest zintegrowany w środowisko XCode edytor scen.
Użytkownik za jego pomocą może pozycjonować obiekty, zmieniać właściwości ich materiałów, a także dowolnie modyfikować ich hierarchię.
Funkcjonalności nie pozwolą na stworzenie gry, ale ułatwią proces aranżacji sceny.
\par
Myśląc o grupie docelowej wskazać można deweloperów tworzących gry typu indie \longpause w niewielkiej grupie, z małym budżetem.
Ponadto, SceneKit sprawdza się w roli generatora grafiki trójwymiarowej w aplikacjach innych niż gry wideo.
Z uwagi na fakt, że jest to technologia natywna łatwo wkomponować można widoki z animowanymi scenami w aplikacje pełniącą inną funkcje niż rozrywkowa.
\par
Narzut użycia biblioteki na wielkość paczki jest niewielki.
W pesymistycznym przypadku wzrost wagi wynieść może 4 MB, jednak potencjalne usunięcie nieużywanych symboli przełoży się na znaczną redukcje tej wartości.
\par
Nie bez przyczyny wspomniano o dacie wydania biblioteki.
Choć jej interfejs ewoluował i rozrastał się w miarę upływu lat to założenia pozostały niezmienione.
2015 był rokiem kiedy podczas konferencji WWDC Apple zaprezentowało koncepcje, które sugerowały programistom wykorzystującym język Swift podejście oparte o komunikacje zorganizowaną wokół protokołów.
W głównej mierze uogólnić można je na zestaw sugestii, które sprawią, że kod będzie reużywalny, testowalny oraz przejrzysty.
Producent stopniowo wcielał je w swoje biblioteki, podczas uaktualniania języka Swift.
Niestety reforma nie dotknęła SceneKit, który współcześnie postrzegany zaczyna być jako przestarzały.
\par
2020 rok był ostatnim kiedy wprowadzono usprawnienia do produktu.
Od tamtego czasu użytkownicy nie otrzymali żadnych poprawek, ani stanowiska Apple wyjaśniającego przyszłość projektu.
Sytuację komplikuje fakt, że kod źródłowy jest zamknięty, więc nie istnieje szansa rozwoju na własną rękę.
\par
Pomimo, że Apple jest gigantem, ma ogromne dochody i przez wielu programistów traktowane jest jako miejsce w którym jakość jest najważniejsza, to nadal w pewnych kwestiach można poczuć się zawiedzionym.
Jedną z takich chwil jest spojrzenie na dokumentację SceneKit.
Przyznać należy, że w sieci dostępnych jest wiele projektów demonstrujących działanie.
Natomiast pozytywny obraz przyćmiewają wszechobecne lakoniczne opisy odnoszące się do zawartości biblioteki.
Niejednokrotnie klasy oraz ich zmienne posiadają wyjaśnienia w formie pojedynczych zdań, nie nadających im szerszego kontekstu.
Odnieść można wrażenie, że jedyną drogą do zrozumienia pewnej puli z zaimplementowanych mechanizmów będzie wysnucie własnych wniosków empirycznie.

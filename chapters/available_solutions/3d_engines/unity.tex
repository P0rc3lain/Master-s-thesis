%
%  Copyright © 2022 Mateusz Stompór. All rights reserved.
%

\subsubsection{Unity}
Najpopularniejszym silnikiem grafiki trójwymiarowej na platformie iOS jest Unity.
Stworzony został z myślą o grach wideo.
Dostępny jest na większości platform, zarówno mobilnych, jak i stacjonarnych, ale w głównej mierze przeznaczony jest do mniejszych produktów.
\par
Rozwiązanie należy zaliczyć do płatnych, choć rozwój aplikacji wykorzystującej technologię można rozpocząć bezpłatnie.
Użytkownik, który jednak poważnie myśli o wydaniu komercyjnie rentownej aplikacji dostrzeże mnogość ograniczeń.
Do najpoważniejszych zaliczyć można brak możliwości zmiany ekranu ładowania widocznego podczas startu na urządzeniu czy niemożność dostępu do kodu silnika.
\par
Z drugiej strony Unity jest kompletne \longpause pozwala na tworzenie w technologiach 2D, 3D, VR oraz AR.
Posiada zintegrowane moduły do obsługi fizyki, dźwięku, kontrolerów, sieci, analityki użycia, a także reklam.
\par
Choć rozwijany jest w języku C++ to użytkownik dokonuje interakcji za pomocą C\#.
W założeniach chodzi o to, aby silnik skompilować tylko raz, a każda modyfikacja była natychmiastowo odzwierciedlana pod względem przyrostowego czasu budowy.
Mono \longpause framework wykorzystywany przez Unity w tym celu \longpause ma jednak pewne mankamenty.
Przede wszystkim użytkownicy narzekają na okazjonalne przycięcia w grach wynikające ze specyfiki zarządzania pamięcią.
Zastanawiać może także sama obecność C\#.
Unity deklaruje, że najlepiej odnajduje się na mobilnych platformach, ale język skryptowy nie pokrywa się z żadną wiodącą technologią ze świata smartfonów.
Na obronę jednak przyznać należy, że grę w unity można stworzyć bez znajomości platformy docelowej.
\par
Dodatkowo, Unity jest rozszerzalne. 
Dokonać można tego za pomocą pluginów występujących w dwóch formach.
Jest to odpowiednio plugin natywny dla zadanego systemu operacyjnego, albo zgeneralizowany oferujący funkcjonalność dla wielu platform.
\par
Pomimo, że Unity jest stale rozwijane to grafika wydaje się odstawać od konkurencji.
Podkreślić należy, że potok renderowania silnika jest konfigurowalny, więc jeśli jakość dla odbiorcy jest niezadowalający istnieje możliwość dokonania korekcji we własnym zakresie.
Otwiera to także możliwość stworzenia zupełnie unikalnego stylu graficznego, choć zaakceptować należy wysoki poziom wymaganej wiedzy do osiągnięcia tego celu.
\par
Sama dokumentacja stoi na wysokim poziomie.
Obfita jest w liczne przykłady użycia i zawiera sugestie co do sposobu implementacji pewnych funkcjonalności w sposób zgodny z wizją autorów.
\par
Pierwotnie cała interakcja warstwy gry oraz silnika obsługiwana była za pomocą skryptów C\#.
Na przestrzeni lat rozwinięto jednak interfejs graficzny środowiska, a także dodano możliwość konfigurowania przebiegu rozgrywki za pomocą programowania wizualnego.
W założeniu podobne jest to nieco do schematów blokowych używanych do prezentacji algorytmów.
Koncept miał za zadanie otworzyć silnik na szersze grono odbiorców i zaoferować możliwość tworzenia gier osobom niebędącym zaznajomionym z programowaniem.
Pozornie pomysł może wydawać się oszczędzać czas i ułatwiać rozwój, mimo to grono osób nadal niechętnie korzysta z rozwiązania.
W głównej mierze podyktowane jest to niską przejrzystością zmian dokonanych przez użytkowników w systemach kontroli wersji.
Równie często przywoływany jest także argument wskazujący na trudność w odnajdywaniu się w logice stworzonej przy pomocy programowania wizualnego.
Równoważna implementacja w formie skryptu określana jest jako łatwiejsza do zrozumienia.
\par
Za minus można uznać narzut w kwestii wagi aplikacji.
Szablon pustego dokumentu wiąże się z koniecznością posiadania około 20 MB wolnej przestrzeni.
Dla porównania natywny, pusty dokument iOS to zaledwie 1 MB.
Wydawać może się to nieistotne, jednak z uwagi na fakt, że aplikacje mobilne często pobierane są za pomocą transmisji komórkowej to naturalnie waga powinna być ograniczona do minimum.
\par
Pewną charakterystyką większości silników do gier, włącznie z Unity, jest zapewnianie użytkownikowi własnego środowiska programistycznego.
Dodać należy, że nie jest ono alternatywą dla preferowanego edytora.
Potrafi znacznie więcej niż asekurować w pisaniu skryptów.
Za jego pomocą importować można zasoby takie jak modele czy muzyka i określać zależności pomiędzy nimi a obiektami używanymi w algorytmach skryptowych. 
Oprogramowanie jest także kluczowe w momencie, kiedy przychodzi czas wygenerowania paczek dla platform docelowych. 
Tak jak wcześniej wspomniano \longpause twórca nie musi znać nawet podstaw programowania każdego ze wspieranych systemów.
Silnik gry jest w stanie za niego wykonać wszystkie pośrednie kroki i dostarczyć gotowy plik binarny, będący finalnym produktem, gotowym do uruchomienia.
W przypadku platformy iOS czy macOS oznacza to, że dokumenty w formacie natywnym dla XCode są ulotne i generowane na nowo przy każdym eksporcie z Unity.
\par
Opisany powyżej sposób rozwijania aplikacji sprawia, że niebywale trudno wykorzystać jest silniki graficzne tylko w wycinku funkcjonalności.
Dedykowane są one do obsłużenia całości aplikacji od interfejsu, po warstwę sieciową aż po interakcje z komponentami platformy jak kamera czy żyroskop.
O ile w przypadku produkcji gier podejście nie budzi zastrzeżeń, o tyle niebywale nierozsądne byłoby opierać całą aplikację użytkową o silnik tylko w celu skorzystania z jego możliwości renderowania.
\figh{images/engines/unity/example-game.png}
     {Przykładowa gra stworzona w silniku Unity}
     {fig::unity-example-game}
     {10cm}

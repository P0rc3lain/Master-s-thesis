%
%  Copyright © 2022 Mateusz Stompór. All rights reserved.
%

\subsubsection{ARKit}
Wydanie ARKit było odpowiedzią Apple na rosnące zainteresowanie branży technologią rozszerzonej rzeczywistości.
Idea polega na wzbogacaniu obrazu przechwytywanego z kamery urządzenia o dodatkowe elementy 2D oraz 3D.
Choć część funkcjonalności wydaje się być współdzielona z SceneKit to firma deklaruje, iż rozwiązania nie są w żadnym stopniu powiązane ze sobą.
Biblioteka jest aktywnie rozwijana, na bieżąco publikowane są poprawki, rozszerzenia funkcjonalności i ulepszenia modelu oświetlenia.
Pomimo, że jej przeznaczenie nie jest ukierunkowane na zgeneralizowaną obsługę trójwymiaru, to rozsądnym wydaje się przyjrzeć jej możliwościom pod kątem tego scenariusza.
\par
Interfejs ARKit zaskakuje w pozytywny sposób.
Dostosowany jest do współczesnych konwencji firmy, dzięki czemu dobrze integruje się z resztą systemu.
Zauważyć można odejście od złożonych hierarchii dziedziczenia na rzecz kompozycji i tworzenia luźnych powiązań za pomocą interfejsów.
\par
Podobnie jednak jak w przypadku innych rozwiązań Apple ARKit nie jest otwarty.
Nie istnieje możliwość wglądu do kodu, ani jego modyfikacji, technologia może być jednak w nieodpłatny sposób wykorzystana z uwzględnieniem przeznaczenia do celów komercyjnych.
\figh{images/engines/ar-kit/example-app.jpg}
     {Rzeczywistość rozszerzona na przykładzie aplikacji mobilnej}
     {fig:arkit-example-app}
     {\linewidth}
\par
Inteligentne suplementowanie obrazu obarczone jest jednak niemałym kosztem.
Każda klatka analizowana jest pod kątem semantyki, a ewentualny ruch wpływający na generowane obiekty rozszerzonej rzeczywistości kompensowany.
Mimo budzącego podziw modelu oświetlenia wydajność jest tylko poprawna
\par
Zaskakiwać może, że ARKit obsługuje jedynie iOS, na pozostałych platformach nie występuje lub tak jak przypadku technologii \textit{mac catalyst} wywołania są ignorowane.
Z jednej strony to zrozumiałe, ponieważ tylko telefony i tablety posiadają stosowne kamery spełniające wymagania.
Z drugiej wyklucza to możliwość wykorzystania ARKit do użycia w celu wygenerowania scen 3D bez AR.

\subsubsection{ARKit}
Wydanie ARKit było odpowiedzią na rosnące zainteresowanie branży technologią rozszerzonej rzeczywistości.
Idea polega na wzbogacaniu obrazu przechwytywanego z kamery urządzenia o dodatkowe elementy 2D oraz 3D.
Podobnie jak w przypadku SceneKit jest to autorska biblioteka Apple.
Choć część funkcjonalności wydaje się być współdzielona jest z SceneKit to firma deklaruje, iż rozwiązania nie są w żadnym stopniu powiązane ze sobą.
\par
ARKit poprawia wiele problemów, które obecne są w SceneKit.
Przede wszystkim jest ona aktywnie rozwijana, na bieżąco publikowane są poprawki, rozszerzenia funkcjonalności i ulepszenia modelu oświetlenia.
\par
Interfejs ARKit jest kolejnym czynnikiem zaskakującym pozytywnie.
Dostosowany jest do współczesnych konwencji firmy i dobrze integruje się z resztą systemu.
Tak jak wspomniano wcześniej interakcje oparte o luźne powiązania pozytywnie wpływają na testowalność kodu wykorzystującego bibliotekę.
\par
Podobnie jednak jak w przypadku innych rozwiązań Apple ARKit nie jest otwarty.
Nie istnieje możliwość wglądu do kodu, ani jego modyfikacji, technologia może być jednak w nieodpłatny sposób wykorzystana z uwzględnieniem przeznaczenia do celów komercyjnych.
\begin{figure}
    \begin{center}
        \includegraphics[width=15cm]{images/engines/ar-kit/example-app.jpg}
    \end{center}
    \caption{Rzeczywistość rozszerzona na przykładzie aplikacji mobilnej}
    \label{fig:arkit-example-app}
\end{figure}
\par
Inteligentne suplementowanie obrazu obarczone jest jednak niemałym kosztem.
Każda klatka analizowana jest pod kątem semantyki, a ewentualny ruch wpływający na generowane obiekty rozszerzonej rzeczywistości kompensowany.
Spodziewać należy się, że mimo budzącego podziw modelu oświetlenia wydajność będzie przeciętna.
\par
Zaskakiwać może, że ARKit obsługuje jedynie iOS, na pozostałych platformach nie występuje lub tak jak przypadku technologii \textit{mac catalyst} wywołania są ignorowane.
Z jednej strony to zrozumiałe, ponieważ tylko telefony i tablety posiadają stosowne kamery spełniające wymagania.
Z drugiej wyklucza to możliwość wykorzystania ARKit do użycia w celu wygenerowania prostych scen 3D bez AR.

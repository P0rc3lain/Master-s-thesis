\section*{Wprowadzenie}
Rozumiejąc główne koncepcje stojące za generowaniem grafiki trójwymiarowej możliwe jest dalsze zgłębienie gałęzi informatyki odnoszącej się do tego zagadnienia.
Na przestrzeni lat powstał niemały zasób literatury poruszający tematykę.
Rozwiązania - silniki - będące implementacją idei rozwinięto do tego stopnia, że same w sobie nie są już komponentami aplikacji, ale wyewoluowały do samodzielnych, pełnoprawnych produktów.
Choć realizowane cele wydają się być identyczne - tworzenie obrazu na podstawie opisu sceny - to podejścia, które stosują różnią się znacząco od siebie.
Podobnie odmienne są wykorzystywane technologie.
Dotyczy to zarówno programowania samej aplikacji, jak i komunikacji z GPU.
Celem niniejszego rozdziału będzie przegląd rozwiązań dostępnych na rynku, zestawienie ich ze sobą i analiza możliwości platform ekosystemu Apple.
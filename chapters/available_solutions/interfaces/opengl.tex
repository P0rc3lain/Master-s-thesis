%
%  Copyright © 2022 Mateusz Stompór. All rights reserved.
%

\subsection{OpenGL}
Najnowsza wersja dostępna na stacjonarnej platformie to opublikowane w 2010 roku OpenGL 4.1.
W przypadku urządzeń mobilnych jest to OpenGL ES 3.0, które pomimo zbieżności wielu zbieżności pozostaje podzbiorem stacjonarnego odpowiednika.
W 2022 roku nadal istnieje możliwość stworzenia aplikacji opartej o OpenGL, choć rozwiązanie uznane jest jako przestarzałe i nierekomendowane od 2018 roku.
\par
Brak wsparcia nie musi stanowić o słabości produktu jednak w przypadku OpenGL największą wadą jest podejście do komunikacji z GPU.
W przeciwieństwie do nowoczesnych API, które synchronizują zadania CPU oraz GPU w zminimalizowany sposób OpenGL dokonuje tego znacznie częściej.
Każdorazowe wywołanie funkcji oznacza wykonanie zapytania do GPU.
\par
Pewną niedogodność stanowi także fakt, iż OpenGL oparty był o język C.
Z tego względu wszystkie jego wywołania mają proceduralny charakter i powodują dysonans w obiektowym kodzie stworzonym w Objective-C czy Swift.
\par
Korzystania nie ułatwia także fakt, że zasoby uzyskane z GPU bez względu na przeznaczenie i typ przekazywane są do użytkownika za pomocą zmiennej całkowitej.
Na programiście spoczywa konieczność stworzenia abstrakcji wokół pozyskanego numeru.
\par
Kolejną niewątpliwą wadą rozwiązania jest sama jego architektura.
OpenGL zaprojektowane jest jako globalna maszyna stanów.
Choć nie jest to jawnie zabronione, podejście sprawia, że nie istnieje efektywna możliwość komunikacji z API w sposób wielowątkowy.
W teorii dwa lub więcej wątków jest w stanie pracować współbieżnie za pomocą różnych kontekstów, ale operacje w sekcji krytycznej są nieefektywne, ponieważ każdorazowa zmiana stanu w GPU jest kosztowna czasowo.
\par
Lata dostępności na rynku wpłynęły na ilość źródeł wyjaśniających działanie API.
Opublikowano dziesiątki książek i tutoriali przedstawiających na przykładach idee programowania grafiki.
Pomimo, że branża odchodzi od OpenGL to społeczność stale wskazuje je jako najbardziej przystępne z pośród wszystkich dostępnych.
\par
Platformy Apple zawsze znacząco opóźnione były pod względem wspieranej wersji OpenGL w stosunku do standardu.
Obecna posiada kilka istotnych braków.
Przykładami jest niemożliwość skorzystania z programów obliczeniowych i tesalacyjnych.
% Napisać, że opengl jest goły i dużo trzeba robić na własną rękę - tak jak kompilacja condiitonali
% Kompilacja jest runtime, co jest słabe
% Napisać, że dużo trzeba zrobić samemu - debugowanie, errory są runtime
% To jest nie do końca prawda - można pracować na wielu wątkach, ale każdy musi mieć swój kontekst.
% Zmiana kontekstu jest kosztowna czasowo
%
%  Copyright © 2022 Mateusz Stompór. All rights reserved.
%

\section{Interfejsy programistyczne}
Ostatnim aspektem wymagającym uwagi podczas analizy bibliotek wykorzystujących akcelerację sprzętową jest interfejs dostępu do zasobów karty graficznej.
Interakcja podzielona może być na dwa obszary.
Pierwszy to faktyczne wykorzystanie mocy obliczeniowej do wykonania spersonalizowanych programów \longpause shaderów lub coraz bardziej popularnych obliczeń na tensorach czy strukturach akceleracji dla śledzenia promieni.
W tym celu posłużyć należy się także wyspecjalizowanym językiem.
Pozostała część API odpowiedzialna jest za umożliwienie użytkownikowi dostarczenie ów programu, a także wszystkich zasobów potrzebnych do jego działania w formie buforów oraz tekstur.
Ta część oddana jest za pośrednictwem biblioteki do języka wysokiego poziomu.
\par
Współczesne API w miarę upływu czasu wydają się zbliżać do siebie coraz bardziej.
Wynika to bezpośrednio ze wspólnego trendu wraz z którym podążają dostępne rozwiązania.
Wnikliwe porównanie obrazujące skalę podobieństw odnaleźć można między innymi jako publikację na łamach blogu \textit{alain.xyz}~\cite{alain_modern_graphics_comparison}.
Wspomnianym trendem tak \longpause jak wyjaśniono we wprowadzeniu \longpause jest chęć oddania możliwie niskopoziomowo sposobu, w który operują jednostki graficzne.
Asynchroniczna natura wykonywania operacji na GPU i chęć minimalizacji ilości synchronizacji pomiędzy GPU i CPU sprawia, że wszystkie nowoczesne API kolejkują zadania i wysyłają je partiach.
\par
Obecnie do dyspozycji programistów Apple oddana jest możliwość skorzystania z dwóch API.
Są to OpenGL, będący ogólnym standardem wspieranym przez wiele platform oraz, Metal, zaprojektowany jako autorskie rozwiązanie Apple.
Pomimo popularności na innych platformach najświeższy, otwarty standard graficzny opublikowany przez grupę Khronos \longpause Vulkan \longpause pozostaje niedostępny.

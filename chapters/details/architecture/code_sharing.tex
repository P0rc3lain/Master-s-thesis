%
%  Copyright © 2022 Mateusz Stompór. All rights reserved.
%

\subsection{Współdzielenie kodu}
Programy cieniujące, podobnie jak kod wykonywany na CPU używają struktur w celu zgrupowania typów prostych w spójną abstrakcję.
Powszechną praktyką w przypadku Metal jest dwukrotna deklaracja i definicja w takich przypadkach. 
Raz dla części Swift, raz dla Metal.
Zdecydowano się jednak wykorzystać inny pomysł, który mógłby być mniej podatny na pomyłki i zredukować ilość zduplikowanego kodu.
Podstawowa definicja struktury stworzona jest w kodzie C, kompatybilnym z Metal.
Dzięki temu może być wykorzystana w programach cieniujących.
W następnym kroku nagłówki C grupowane są w moduły, które kompilator XCode udostępnia do kodu Swift.
Dzięki temu za pomocą pojedynczego importu modułu dane opisane w kodzie Swift mogą być w łatwy sposób przekazywane do GPU, a pojedyncza definicja sprawia, że ustrzec można się przed błędami w przypadku braku propagacji zmian.
Dodatkowo Swift choć wewnętrznie zapewnia to standard nie daje gwarancji w kwestii uporządkowania pól dla struktury, co jest kluczowe w celu właściwego odczytania danych.
\lstinputlisting[language=C, caption=Definicja struktury w języku C]{code/code_sharing/Camera.h}
\lstinputlisting[language=Swift, caption=Deklaracja modułu Swift składającego się z kodu C]{code/code_sharing/module.map}
\lstinputlisting[language=Swift, caption=Użycie modułu w kodzie Swift]{code/code_sharing/PNSceneDescription.swift}
%
%  Copyright © 2022 Mateusz Stompór. All rights reserved.
%

\subsection{Graf sceny}
Graf zaimplementowany został w oparciu o drzewo.
Pomiędzy węzłami występuje relacja rodzic-potomek i wymagane jest, aby dany węzeł występował w strukturze dokładnie jeden raz.
Oznacza to, że dwa różne węzły nie mogą posiadać tego samego potomka.
Dozwolona jest dowolna ilość potomków.
\par
Każdy z węzłów zbudowany jest z trojga informacji.
Referencji do rodzica, listy dzieci oraz odnośnika do samych danych.
Korzeń drzewa nie posiada rodzica, jego referencja będzie pusta.
\par
Struktura danych jest generyczna.
Może posłużyć do przechowania informacji dowolnego typu.
Drzewo posiada zaimplementowane najpopularniejsze operacje, takie jak wyszukiwanie, wstawianie elementu, usuwanie.
Ze względu na separacje zachowania od samych danych wykonywane jest to za pomocą pochodnych interfejsu \textit{PNNodeInteractor}.
\figh{images/pnengine/scene/node.png}
     {Definicja węzła w grafie sceny}
     {fig:scene_node}
     {5.4cm}
% TODO: Napisać po co w ogóle ta hierarchia została wprowadzona
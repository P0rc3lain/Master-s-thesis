%
%  Copyright © 2022 Mateusz Stompór. All rights reserved.
%

\subsection{Interakcja}
Poza jednostkami stacjonarnymi biblioteka przeznaczona jest do współpracy z urządzeniami przenośnymi wyposażonymi w ekrany dotykowe.
Z tego powodu należało zadbać o wykrywanie interakcji pomiędzy użytkownikiem, a modelami osadzonymi na scenie.
\figh{images/raycasting.jpg}
     {Ilustracja idei ray-castingu}
     {fig:ray-casting}
     {8cm}
Do dyspozycji oddano klasę realizującą interfejs \textit{PNScreenInteractor}.
Przeszukuje ona graf sceny na podstawie współrzędnych punktu wybranego na ekranie.
W przypadku powodzenia zwraca obiekt będący najbliżej kamery.
\lstinputlisting[language=Swift, caption=Węzeł kamery]{code/PNScreenInteractor.swift}
W tym celu wykorzystano technikę rzucania promieni.
Promień osadzany jest w punkcie początkowym kamery z uwzględnieniem jej transformacji, a następnie wykonywany jest algorytm przecinania półprostej z ramkami ograniczającymi modeli.
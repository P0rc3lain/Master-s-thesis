%
%  Copyright © 2022 Mateusz Stompór. All rights reserved.
%

\subsection{Właściwości węzła}
Dane przechowywane w grafie biblioteki oparte muszą być o interfejs \textit{PNSceneNode}.
Zawiera on w sobie deklarację kilku istotnych dla silnika rzeczy.
Dotyczy to pozycji, otoczki zewnętrznej obiektu oraz jego identyfikatora.
Węzeł posiada informacja nie tylko odnoszące się do lokalnej pozycji i otoczki dla obecnego obiektu, ale także wartości w relacji do układu współrzędnych świata.
Dzięki temu użytkownik wygodnie porównywać może pozycje obiektów między sobą, a ponieważ wartość jest przechowywana w węźle nie trzeba jej każdorazowo obliczać.
\par
Oprócz właściwości węzeł sceny posiada także zachowanie.
Może on implementować metodę \textit{update()}, która wykonywana jest przy generowaniu każdej klatki.
Najistotniejszą jednak jest \textit{write()} służąca do zamiany hierarchicznej struktury drzewa na płaską, wydajną podczas renderowania.
Obiekt musi być w stanie zakodować siebie wewnątrz struktury \textit{PNSceneDescription}, opisanej w dalszej części pracy.
\lstinputlisting[language=Swift, caption=Interfejs bazowy dla obiektów w węzłach grafu sceny]{code/PNSceneNode.swift}
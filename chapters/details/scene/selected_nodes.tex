%
%  Copyright © 2022 Mateusz Stompór. All rights reserved.
%

\subsection{Wybrane węzły}
Użytkownik jest w stanie na własną rękę implementować interfejs \textit{PNSceneNode} w celu personalizacji węzłów, jednak nie jest to konieczne do skorzystania z biblioteki.
W celu wygenerowania obrazu i aranżacji sceny skorzystać można z szeregu uprzednio stworzonych węzłów, które rozróżnić można na kilka typów.
Zdecydowano się stworzyć wiele niewielkich, skoncentrowanych typów.
Dzięki temu wyspecjalizowane są one w kierunku pełnionej przez siebie funkcjonalności i łatwiej można je analizować niż w przypadku alternatywnego podejścia, gdzie węzeł jednocześnie mógłby pełnić wiele ról.
\subsubsection{Transformacja}
Podstawowym typem są węzły transformacji.
Ich zadaniem jest grupowanie obiektów.
Dzięki temu umożliwiają modyfikację pozycji wszystkich potomnych węzłów za pomocą pojedynczej zmiany.
Możemy do nich zaliczyć interfejs \textit{PNSceneNode} w podstawowej formie oraz jego pochodną \textit{PNAnimatedSceneNode}.
\lstinputlisting[language=Swift, caption=Węzeł obsługujący animacje]{code/PNAnimatedNode.swift}
Główną różnicą występującą pomiędzy nimi jest źródło używane do próbkowania pozycji.
Choć interfejs tego nie narzuca to w przypadku \textit{PNSceneNode} zakłada się, że pozycja określana jest na podstawie zmiennych modyfikowanych przez użytkownika.
\textit{PNAnimatedNode} ma w tym celu używać animatora, który pozycje uzyskuje z szeregu animacji kluczowych dla zadanego punktu w czasie.
\figh{images/pnengine/scene/transform_node.png}
     {Hierarchia węzłów służących do reprezentacji transformacji}
     {fig:scene_transform}
     {10cm}
\subsubsection{Kamera}
Kolejnym istotnym elementem sceny są węzły kamery.
Służą one do przedstawienia perspektywy z której scena zostanie wyrenderowana.
Rozwijają one koncepcje wdrożone podczas prezentacji węzłów transformacji.
Pozycja kamery oparta może być o zmienną reprezentującą konkretną, modyfikowalną wartość lub być próbkowana z klatek kluczowych.
\figh{images/pnengine/scene/camera.png}
     {Hierarchia węzłów służących do reprezentacji kamery}
     {fig:scene_camera}
     {8cm}
\par
Sam interfejs węzła w stosunku rozszerza podstawowy interfejs o dodanie referencji do kamery.
Istnieje szansa, że osadzona na scenie zostanie więcej niż jedna kamera.
W takiej sytuacji wykorzystywana zostaje także zmienna \textit{priority} na podstawie której kamery są sortowane.
Do renderowania wybrana zostanie ta z najwyższym priorytetem.
\lstinputlisting[language=Swift, caption=Interfejs węzła kamery]{code/PNCameraNode.swift}
\subsubsection{Model}
\figh{images/pnengine/scene/mesh.png}
     {Hierarchia węzłów służących do reprezentacji siatek}
     {fig:scene_mesh}
     {\linewidth}
Do dyspozycji oddano także węzły służące do reprezentacji siatek.
W podstawowej formie \textit{PNMeshNode} może posłużyć do aranżacji statycznej siatki, która dodatkowo obsłuży animacje kluczowe w rozszerzeniu \textit{PNAnimatedMeshNode}.
\lstinputlisting[language=Swift, caption=Interfejs węzeła siatki]{code/PNMeshNode.swift}
Biblioteka wspiera także animację szkieletową dla siatek.
\textit{PNRiggedMeshNode} jest pochodną interfejsu \textit{PNMeshNode} i wzbogaca deklarację o odnośnik do szkieletu.
Podobnie jak w innych przypadkach animację wzbogacić można o technikę animacje na podstawie klatek kluczowych.
\lstinputlisting[language=Swift, caption=Interfejs węzła siatki obsługującej animację szkieletową]{code/PNRiggedMeshNode.swift}
\subsubsection{Światła}
Aby obraz w ogóle mógł powstać konieczne są źródła światła, która sprawią, że kamera zarejestruje kształty modeli.
\lstinputlisting[language=Swift, caption=Interfejs węzła światła nadającego wpływ komponentu ambientowego]{code/PNAmbientLightNode.swift}
Poza światłem kierunkowym definiowanym poza grafem sceny wykorzystać można źródła ambientowe, punktowe dookólne oraz skierowane.
\lstinputlisting[language=Swift, caption=Interfejs węzła światła punktowego\, kierunkowego]{code/PNSpotLightNode.swift}
We wszystkich przypadkach interfejsy są zwięzłe i oczekują jedynie referencji do wybranego typu światła.
\lstinputlisting[language=Swift, caption=Interfejs węzła światła punktowego\, dookólnego]{code/PNOmniLightNode.swift}
\figh{images/pnengine/scene/lights.png}
     {Hierarchia węzłów służących do reprezentacji świateł}
     {fig:scene_lights}
     {10cm}
\subsubsection{Efekty cząsteczkowe}
Ostatnim typem węzła, który może zostać użyty w grafie jest węzeł efektów cząsteczkowych.
Służy on do osadzenia emitera na scenie.
\figh{images/pnengine/scene/particle.png}
     {Hierarchia węzłów służących do reprezentacji efektów cząsteczkowych}
     {fig:scene_particle}
     {4cm}
\lstinputlisting[language=Swift, caption=Interfejs węzła służącego do obsługi efektów cząsteczkowych]{code/PNParticleNode.swift}
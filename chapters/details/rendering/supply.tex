\subsection{Opis klatki}
Wygenerowania każdej klatki wymaga dwóch składowych.
Opisu sceny przypadającego na daną chwilę w formie efektywnej do procesowania przez procesor.
Dodatkowo konieczne jest aby informacje które wymagane są do renderowania przeniesione zostały na GPU.
Dane te każdorazowo przygotowywane są przed generowaniem klatki, a następnie przekazywane do klasy \textit{PNPipeline}.
Są one dystrybuowane dalej, aż do poszczególnych klas implementujących interfejsy \textit{PNComputeJob} lub \textit{PNRenderJob}.
\par
Agregatorem danych dla klatki jest klasa \textit{PNFrameSupply}.
Jej instancje tworzone są na początku potoku i niszczone wraz z jego zakończeniem.
Zawiera ona w sobie referencję do sceny opartej o płaską, niehierarchiczną strukturę, masek wskazujących widoczność obiektu oraz buforów GPU z parametrami obiektów w odniesieniu do sceny.
\lstinputlisting[language=Swift, caption=Struktura będąca informacją na bazie której klatka jest renderowana]{code/PNFrameSupply.swift}
\par
\textit{PNRenderMask} zawiera w sobie informacje na temat widzialności wszystkich obiektów znajdujących się na scenie z punktu widzenia aktywnej kamery oraz źródeł światła rzucających cienie.
Zdecydowano się na pozostawienie w scenie obiektów, które są niewidoczne dla kamery właśnie przez wzgląd na fakt, że mogą być one nadal w zasięgu źródeł światła i rzucać cień w polu widzenia kamery.
Przed zleceniem wygenerowania modelu do shadera wartość maski jest weryfikowana, jeśli obiekt jest niewidoczny algorytm przechodzi do analizy kolejnego.
\lstinputlisting[language=Swift, caption=Struktura zawierająca maski dla renderowanych obiektów]{code/PNRenderMask.swift}
\par
Bufory, które osadzone są w przestrzeni adresowej GPU zasadniczo podzielić można na dwie kategorie.
Aktualizowane jednokrotnie podczas wczytywania sceny, takie jak siatki czy tekstury opisujące materiały.
Alternatywnie, bufory mogą aktualizowane być dla każdej klatki, w momencie gdy zawierają pozycje czy przechowują kroki pośrednie renderowania klatki.
\textit{PNBufferStore} zawiera w sobie wszystkie bufory, które aktualne są tylko dla pojedynczej klatki.
Są to dane dotyczące parametrów źródeł światła, pozycji wszystkich encji czy palet transformacji dla modeli ze szkieletem.
Istotne jest, że bufory tworzone są tylko raz i dynamicznie dostosowują się do rozmiaru danych, które przechowują.
Instancja klasy reużywana jest dla kolejnych klatek.
W przypadku renderowania wielowątkowego istnieją dwie równolegle używane instancje.
Jedna jest używana aktywnie do renderowania klatki, podczas gdy druga podlega aktualizacji.
Po zakończeniu renderowania klatki cykl odwraca się.
\lstinputlisting[language=Swift, caption=Interfejs skupiający w sobie niezmienniki dla danej klatki]{code/PNBufferStore.swift}
\par
Zlecanie renderowania oparte jest o informacje zawarte w \textit{PNSceneDescription}.
Struktura agreguje w sobie dwa typy informacji.
Pierwszą jest hierarchia obiektów w formie drzewa opartego o tablicę.
Encje przechowywane w niej mają uprzednio obliczone pozycje we współrzędnych świata.
Dane które przechowywane są w węzłach stanowią jedynie odnośniki do pozostałych właściwości z klasy.
Pozostałe pola, które przechowują wartości na temat źródeł światła, siatek czy generatorów efektów cząsteczkowych podzielone zostały, aby zapewnić szybką iterację oraz uniewrażliwić na przypadek kilkukrotnej iteracji przez obiekty.
Przykładowo renderując efekty cząsteczkowe informacja na temat siatek czy ich szkieletów jest zbędna.
Klasa odpowiedzialna za przekazanie danych do shadera może bezpośrednio odneść się do generatorów.
Podobnie sprawa wygląda w przypadku nakładania efektów oświetlenia.
Możliwa jest natychmiastowa iteracja przez wszystkie rodzaje źródeł światła z pominięciem siatek, które już zostały wyrenderowane do klatki pośredniej.
\lstinputlisting[language=Swift, caption=Struktura sceny w formie efektywnej do renderowania]{code/PNSceneDescription.swift}

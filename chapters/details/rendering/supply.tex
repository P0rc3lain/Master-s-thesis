%
%  Copyright © 2022 Mateusz Stompór. All rights reserved.
%

\subsection{Opis klatki}
Wygenerowanie każdej klatki wymaga opisu sceny przypadającego na konkretną chwilę, w formie efektywnej do procesowania przez CPU.
Oprócz tego konieczne jest aby informacje użyte do renderowania przeniesione zostały na GPU.
Dane te każdorazowo przygotowywane są przed generowaniem klatki, a następnie przekazywane do klasy \textit{PNPipeline}.
Dystrybucja wgłąb przebiega aż do poszczególnych klas implementujących interfejsy \textit{PNComputeJob} lub \textit{PNRenderJob}.
\par
Agregatorem danych dla klatki jest klasa \textit{PNFrameSupply}.
Jej instancje tworzone są na początku potoku i niszczone wraz z jego zakończeniem.
Zawiera ona w sobie referencję do sceny opartej o płaską, niehierarchiczną strukturę, masek określających widoczność obiektu oraz buforów GPU z parametrami obiektów w odniesieniu do sceny.
\code{Swift}
     {Struktura będąca informacją na bazie której klatka jest renderowana}
     {code/gpu_supply/PNFrameSupply.swift}
\par
\textit{PNRenderMask} posiada informacje na temat widzialności wszystkich obiektów znajdujących się na scenie.
Perspektywa obejmuje punkt widzenia aktywnej kamery oraz źródeł światła rzucających cienie.
Zdecydowano się na pozostawienie w scenie obiektów niewidocznych dla kamery przez wzgląd na fakt, że mogą być one nadal w zasięgu źródeł światła i rzucać cień w polu widzenia kamery.
Przed zleceniem wygenerowania modelu do shadera wartość maski jest weryfikowana.
Jeżeli obiekt jest niewidoczny algorytm przechodzi do analizy kolejnego.
\code{Swift}
     {Struktura zawierająca maski dla renderowanych obiektów}
     {code/gpu_supply/PNRenderMask.swift}
\par
Bufory osadzone w przestrzeni adresowej GPU zasadniczo podzielić można na dwie kategorie.
Aktualizowane jednokrotnie podczas wczytywania sceny, takie jak siatki czy tekstury opisujące materiały i te odświeżane dla każdej klatki.
\textit{PNBufferStore} posiada wszystkie bufory aktualne tylko dla pojedynczej.
Są to dane dotyczące parametrów źródeł światła, pozycji wszystkich encji czy palet transformacji dla modeli ze szkieletem.
Istotne jest, że bufory tworzone są tylko raz i dynamicznie dostosowują się do rozmiaru danych.
Instancja klasy reużywana jest dla kolejnych klatek.
W przypadku trybu wielowątkowego istnieją dwie instancje.
Jedna jest używana aktywnie do renderowania klatki, podczas gdy druga podlega aktualizacji.
Po zakończeniu działania potoku cykl odwraca się.
\code{Swift}
     {Interfejs skupiający w sobie niezmienniki dla danej klatki}
     {code/gpu_supply/PNBufferStore.swift}
\par
Zlecanie renderowania oparte jest o informacje zawarte w \textit{PNSceneDescription}.
Struktura agreguje w sobie dwa typy informacji.
\par
Pierwszym jest hierarchia obiektów w formie drzewa opartego na tablicy.
Encje mają uprzednio obliczone pozycje we współrzędnych świata.
Dane w węzłach stanowią jedynie odnośniki do pozostałych właściwości z klasy.
\par
Pozostałe pola posiadają wartości na temat źródeł światła, siatek czy generatorów efektów cząsteczkowych.
Podzielone zostały, aby zapewnić szybkie przeglądanie oraz uniewrażliwić na przypadek zbędnie powtórzonej iteracji przez obiekty.
Przykładowo renderując efekty cząsteczkowe informacja na temat siatek czy ich szkieletów jest zbędna.
Klasa odpowiedzialna za przekazanie danych do shadera może bezpośrednio odnieść się do generatorów.
Podobnie sprawa wygląda w przypadku nakładania efektów oświetlenia.
Możliwa jest natychmiastowa iteracja przez wszystkie rodzaje źródeł światła z pominięciem siatek, które już zostały wyrenderowane do klatki pośredniej.
\code{Swift}
     {Struktura sceny w formie efektywnej do renderowania}
     {code/gpu_supply/PNSceneDescription.swift}

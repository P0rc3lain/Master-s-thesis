\section{Renderowanie}
\begin{figure}[H]
    \begin{center}
        \includegraphics[width=15cm]{images/pnengine/toy_drummer.jpg}
    \end{center}
    \caption{Klatka prezentująca wyrenderowany model zabawki}
    \label{fig:toy_drummer}
\end{figure}
\begin{figure}[H]
    \begin{center}
        \includegraphics[width=15cm]{images/pnengine/sponza_scene.jpg}
    \end{center}
    \caption{Scena z popularnego modelu firmy Crytek \longpause Sponza}
    \label{fig:sponza}
\end{figure}
\begin{figure}[H]
    \begin{center}
        \includegraphics[width=15cm]{images/pnengine/bloom.jpg}
    \end{center}
    \caption{Przykład efektu bloom na podstawie klatki stworzonej przez bibliotekę}
    \label{fig:bloom}
\end{figure}
    \subsection{Ogólne podejście}
    \subsection{Potok renderowania}
    \subsection{Zawarte techniki}
        \subsubsection{PBR}
        \subsubsection{Cienie}
        \subsubsection{Mapowanie Normalnych}
        \subsubsection{Przeźroczystość}
        \subsubsection{Efekty cząsteczkowe}
        \subsubsection{Bloom}
        \subsubsection{Environment mapping}
        \subsubsection{SSAO}
        \subsubsection{Postprocessing}
\subsection{Ogólne podejście}
Koncepcyjnie proces generowania klatki obrazu oparto w głównej mierze o podejście odroczonego renderowania.
Oznacza to, że finalny obraz powstaje poprzez zastosowanie wielu pośrednich klatek, które finalnie składane są jedną całość.
Decyzja ta podytkowana była przez względy wydajnościowe.
Zdarza się, że pośrednie klatki używane są także przez więcej niż jeden program cieniujący, co dodatkowo zwiększa korzyści płynące z wyboru techniki.
\par
Powyższe podejście nie dotyczy jednak wszystkich przypadków, które mogą wystąpić.
W szczególności modele obiektów charakteryzujących się przezroczystością lub efekty cząsteczkowe oparte są o forward rendering.
Są one nanoszone stosunkowo późno w potoku, na gotową, ocieniowaną klatkę obrazu.

%
%  Copyright © 2022 Mateusz Stompór. All rights reserved.
%

\subsection{Ogólne podejście}
Koncepcyjnie proces generowania klatki obrazu oparto w głównej mierze o podejście odroczonego renderowania.
Oznacza to, że finalny obraz powstaje poprzez zastosowanie wielu pośrednich klatek składanych w jedną całość.
Decyzja ta podyktowana była przez względy wydajnościowe, będące najbardziej widoczne podczas cieniowania.
Jednak zdarza się, że pośrednie klatki używane są przez więcej niż jeden program cieniujący, co dodatkowo zwiększa korzyści płynące z wyboru techniki.
\par
Powyższe podejście nie dotyczy jednak wszystkich przypadków, które mogą wystąpić.
W szczególności modele obiektów charakteryzujących się przezroczystością lub efekty cząsteczkowe oparte są o tradycyjny potok \longpause forward rendering.
Są one nanoszone stosunkowo późno, na gotową, ocieniowaną klatkę obrazu.

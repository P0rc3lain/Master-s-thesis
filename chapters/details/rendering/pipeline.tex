\subsection{Potok renderowania}
Niniejsza sekcja rozłoży na części pierwsze przykładową klatkę obrazu wygenerowaną przez biblioteką.
Omówione zostaną krok po kroku etapy, które złożyły się na jej stworzenie.
\par
Scena testowa, którą przygotowano zawiera popularny model zamku stworzony przez Crytek \longpause Sponza.
Na dziedzińcu modelu unosi się model samolotu, który posiada niebieski, jednoklity odcień i charakteryzuje przezroczystością.
Poniżej samolotu znajduje się źródło efektów cząsteczkowych symulująca dym.
Scena oświetlona jest przy użyciu pojedynczego źródła światła, kierunkowego, umiejscowionego prostopadle do podłogi dziedzińca.
Kamera umiejscowiona jest w taki sposób aby zawrzeć w polu widzenia wszystkie modele i obejmuje także niebo.
\par
Wybrór sceny oraz osadzenia obeiktów motywowany był chęcią pokazania możliwie dużej liczby efektów.
Przedstawiony potok będzie kompletny pod względem kroków, które mogą w nim wystąpić.
Należy pamiętać jednak, że nie przedstawione zostaną wszystkie możliwości silnika przez wzgląd na obszerną ilość możliwych konfiguracji.
\figh{images/pnengine/rendering/final.png}
    {Finalna klatka obrazu}
    {fig:pipeline_final}
    {\linewidth}
Potok renderowania oparty został o dwa rodzaje programów cieniujących.
W przypadku renderowania siatek używane są programy cieniujące korzystające z tradycyjnego potoku GPU \longpause wykorzystujące programy cieniujące wierzhcołków i fragmentów.
Natomiast jeśli nie zachodzi potrzeba generowania siatki, a procedura wciąż odbywa się na GPU, np w przypadku efektów postprocesowych używane są programy obliczeniowe.
\par
W pierwszym kroku renderowaniu podlegają siatki.
Dotyczy to zarówno tych wyposażonych w szkielet, jak i nie.
Konieczne jest aby materiały, których używają modele nie były przezroczyste.
Te, które są naniesione będą w późniejszym czasie.
\par
Na klatkę obrazu wynikowego składa się cztery tekstury.
Trzy z nich posiada cztery kanały, pozostała jedna pojedynczy podzielony jednak na dwa.
Pierwsza z nich przechowuje informacje o kolorze oraz chropowatości.
Druga zawiera wektory normalne oraz współczynnik metaliczności.
Kolejna pozycja w układzie współrzędnych ekranu oraz odbijalność.
Ostatnia przechowuje głębię oraz maskę szblonu.
\par
Generowanie klatki nie wiąże się z cieniowaniem, zostanie ono wykonane w późniejszym czasie.
Skorzystanie z tego pośredniego kroku sprawia, że oszczędzana jest moc obliczeniowa dla każdego fragmentu, który potencjalnie mógł zostać ocieniowany, ale zostałby przesłoniony przez inny.
Obecna technika daje pewność, że cienowaniu podlegać będą tylko fragmenty widoczne na finalnej ramce.
\figh{images/pnengine/rendering/gbuffer/all.jpg}
    {Zawartość tekstur składających się na GBuffer}
    {fig:pipeline_gbuffer}
    {\linewidth}
\figh{images/pnengine/rendering/ssao/ssao-combine.jpg}
    {Klatki pośrednie SSAO, pierwotna (lewa) oraz rozmazana (prawa)}
    {fig:pipeline_ssao}
    {\linewidth}
\figh{images/pnengine/rendering/lights/directional.png}
    {Bufor cieni dla źródło kierunkowego}
    {fig:pipeline_shadow_map}
    {\linewidth}
\figh{images/pnengine/rendering/lighting/lights-combined.jpg}
    {Klatki poddane wpływowi źródeł światła, amibentowe (lewa), kierunkowe (prawa)}
    {fig:pipeline_lights}
    {\linewidth}
\figh{images/pnengine/rendering/combine-env-part-trans.jpg}
    {Klatki mapowania środowiska, efektów cząsteczkowych oraz obiektów przezroczystych}
    {fig:pipeline_env_part_trans}
    {\linewidth}
\figh{images/pnengine/rendering/bloom-combine.jpg}
    {Klatki pośrednie Bloom, najjaśniejsze fragmenty (lewa), rozmazane (prawa)}
    {fig:pipeline_bloom}
    {\linewidth}
\figh{images/pnengine/rendering/pipeline.png}
    {Graf renderowania}
    {fig:pipeline_graph}
    {5cm}

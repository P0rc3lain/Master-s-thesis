\subsection{Przebieg przekazywania zadań}
\figh{images/pnengine/all-stages.png}
     {Hierarchia klas odpowiedzialnych za renderowanie}
     {fig:hierarchy_rendering_jobs}
     {\linewidth}
Zgodnie z informacjami, które zawarte były w sekcji nakreślającej przepływ danych klasa potoku podzielona jest na mniejsze fragmenty.
Są nimi etapy, które współdzielą między sobą tekstury, nakładając kolejne warstwy efektów lub tworząc klatki pośrednie.
Etapy nie zawierają jednak logiki odpowiedzialnej za interakcję z programami cieniującymi.
Służą one jedynie enkapsulacji i spłaszczenia logiki do formy prostych wyjść i wejść w formie tesktur.
\par
Komponenty odpowiedzialnymi za kolejkowanie zleceń są klasy z przyrostkiem \textit{job}.
Istnieje bezpośrednie relacja pomiędzy programem cieniującym a pochodnymi \textit{PNRenderJob} lub \textit{PNComputeJob}.
Klasy te rozumieją jakich danych oczekuje program cieniujący oraz które obiekty kwalifikują się do renderowania przez niego.

%
%  Copyright © 2022 Mateusz Stompór. All rights reserved.
%

\subsection{Siatka} 
Siatka definiuje zbiór wierzchołków w przestrzeni trójwymiarowej.
Konieczność efektywnego zarządzania pamięcią sprawia, że współdzielone są one pomiędzy poszczególnymi powierzchniami.
Są one reprezentowane za pomocą struktury \textit{PNPieceDescription}.
Zawarta jest tam informacja na temat materiału, który powinien zostać wykorzystany do renderowania wraz z detalami powierzchni opisanymi w \textit{PNSubmesh}.
Wewnątrz przechowywany jest bufor, który specyfikuje w jaki sposób łączyć wierzchołki przechowywane bezpośrednio w \textit{PNMesh}, aby nadać powierzchni pożądany kształt.
Siatka zawiera także swoją otoczkę, którą rozpinają przechowywane w niej wierzchołki.
Efektywne renderowanie wymaga aby niewidoczne wierzchołki usuwane zostały podczas wykonywania potoku.
W tym celu wykorzystuje się informacje na temat kierunku w który zwrócony jest trójkąt \longpause do lub od kamery.
Sposób filtrowania dla siatki skonfigurowany może być przy użyciu struktury \textit{PNCulling}.
\figh{images/pnengine/scene/mesh-details.png}
     {Budowa siatki}
     {fig:mesh-details}
     {10cm}

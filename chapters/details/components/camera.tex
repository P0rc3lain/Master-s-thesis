\subsection{Kamera}
Typ Kamery jest protokołem.
Wewnątrz niego najistotniejszą właściwością opisującą kamerę jest macierz projekcji.
Wykorzystywana jest ona podczas renderowania obrazu.
Reguluje zasięg widoczności oraz kąt, który obejmuje pole widzenia.
Ściśle powiązana z macierzą projekcji jest otoczka.
Używana jest w procesie filtrowania niewidzialnych obiektów.
Ostatnim parametrem charakteryzującym kamerę jest zmienna definiująca czy jest ona włączona.
Przydatna jest w momencie gdy na scenie umieszczonych jest mnoga ilość kamer i zachodzi potrzeba dynamicznej zmiany z jednej na inną.
Biblioteka posiada dwie implementacje interfejsu \longpause kamerę perspektywiczną (\textit{PNPerspectiveCamera}) oraz ortograficzną (\textit{PNOrthographicCamera}).
\lstinputlisting[language=Swift, caption=Protokół kamery]{code/PNCamera.swift}
%
%  Copyright © 2022 Mateusz Stompór. All rights reserved.
%

\subsection{Ramka ograniczająca}
Wszystkie spośród typów obiektów możliwych do osadzenia na scenie scharakteryzowane mogą być w oparciu o obszar na który mają wpływ.
W przypadku siatek jest to odpowiednio wierzchołek osiągający punkt minimalny oraz maksymalny, dla źródła światła przestrzeń na który oddziałuje.
Kamerę opisuje zasięg pola widzenia.
Z tego względu zdecydowano się nadać właściwość każdemu z tych obiektów dodatkową, integralną właściwość \longpause otoczkę zewnętrzną.
Właściwość ta  definiowana jest za pomocą dwóch punktów \longpause minimalnego oraz maksymalnego w przestrzeni trójwymiarowej, które przechowywane są w formie homogenicznego wektora, dla ułatwienia obliczeń z macierzami transformacji.
W przypadku biblioteki zdecydowano się wykorzystać ramki wyrównane do osi \longpause \textit{AABB}.
\lstinputlisting[language=Swift, caption=Struktura ramki ograniczającej]{code/PNBoundingBox.swift}
Na ramce przeprowadzić można szereg operacji.
Jej przechowywanie dla każdego z obiektów daje dwie podstawowe korzyści.
Możliwe jest efektywne stwierdzanie interakcji z zadanym obiektem na scenie za pomocą przecięcia z promieniem.
Zachodzi to na przykład w momencie gdy użytkownik kliknie na punkt ekranu.
Dodatkowo, dzięki wykorzystaniu algorytmu przecięcia ramek można odfiltrować obiekty niebędące widzialne na scenie.
W tym celu dla każdej klatki przecinana jest ramka kamery ze wszystkimi obiektami.
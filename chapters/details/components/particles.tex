%
%  Copyright © 2022 Mateusz Stompór. All rights reserved.
%

\subsection{Efekty cząsteczkowe}
Procesowanie efektów cząsteczkowych zlokalizowane jest na CPU.
Decyzja została umotywowana chęcią zachowania przystępności.
Znacznie łatwiej z punktu widzenia użytkownika zaimplementować kilka interfejsów niż samodzielnie stworzyć program obliczeniowy do wykonania na GPU.
\figh{images/pnengine/scene/particle-details.png}
     {Hierarchia komponentu efektów cząsteczkowych}
     {fig:particle-details}
     {10cm}
\par
Każda cząsteczka opisana jest za pomocą struktury \textit{PNParticle}.
Zdefiniowano tam między innymi pozycję, prędkość oraz kierunek podążania.
Cząsteczki różnią się wzajemnie maksymalnym czasem życia, początkową pozycją czy limitem prędkości.
\lstinputlisting[language=Swift, caption=Struktura cząsteczki]
                {code/particles/PNParticle.swift}
\par
Zasady doboru parametrów inicjacji zdefiniowane są za pomocą \textit{PNEmissionRules}.
Opisane są one w formie przedziałów liczbowych w przypadku wartości jednowymiarowych lub obszarów przestrzeni dla wektorów.
\lstinputlisting[language=Swift, caption=Zasady wytwarzania cząsteczek]
                {code/particles/PNEmissionRules.swift}
\par
Specyfikowane restrykcje wykorzystywane są przez pochodne interfejsu \textit{PNEmitter}, które w oparciu o określone ograniczenia tworzą nowe cząsteczki.
\lstinputlisting[language=Swift, caption=Emitter cząsteczek]
                {code/particles/PNEmitter.swift}
\par
Klasa kontrolera odpowiada za zachowanie cząsteczek.
Dotyczy to ich konfigurowania podczas tworzenia oraz momentu w którym ma to nastąpić.
Dodatkowo, kontroler odpowiedzialny jest za aktualizacje aktualnych parametrów cząsteczek.
\lstinputlisting[language=Swift, caption=Kontroler zachowania cząsteczek]
                {code/particles/PNParticleController.swift}
\par
Z punktu widzenia silnika renderowania najistotniejszy jest interfejs \textit{PNRenderableParticlesProvider}.
Za jego pomocą zwracana jest gotowa lista cząsteczek oraz tekstura zawierająca atlas z poszczególnymi stanami pomiędzy którymi animacja powinna być przeprowadzona.
Konieczne jest również uzyskanie dostępu do zasad tworzenia cząsteczek dzięki czemu można estymować maksymalny obszar który obejmą one na scenie.
\lstinputlisting[language=Swift, caption=Interfejs do pozyskiwania renderowalnych cząsteczek]
                {code/particles/PNRenderableParticlesProvider.swift}
\par
Spoiwem łączącym wszystkie elementy w całość jest \textit{PNParticleGenerator}.
Jest to zgeneralizowana pod kątem zastosowania klasa, którą użytkownik może wypełnić własnym emiterem, zasadami, kontrolerem oraz atlasem.
Będzie ona dla każdej klatki koordynować aktualizacje cząsteczek i przesyłać dane na ich temat do bufora GPU.
Generator poza implementacją interfejsu \textit{PNRenderableParticlesProvider} jest pochodną \textit{PNTask}.
Istotne jest aby umieścić go w kolejce zadań biblioteki by ta wzbudzała go co klatkę kiedy obraz ma zostać wygenerowany.
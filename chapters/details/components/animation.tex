%
%  Copyright © 2022 Mateusz Stompór. All rights reserved.
%

\subsection{Techniki animacji}
Biblioteka wspiera dwie techniki nadające obiektom płynny ruch.
Są to animacje bryły sztywnej oraz szkieletowe.
Mogą być one łączone ze sobą.
\subsubsection{Animacja bryły sztywnej} 
Przypadek użycia tego rodzaju animacji jest szeroki.
W przypadku grafu sceny skorzystać można z niego dla wszystkich rodzajów obiektów.
\figh{images/pnengine/animation/keyframe.jpg}
     {Animacja na podstawie klatek kluczowych}
     {fig:animation-keyframe}
     {6cm}
\par
Technika oparta jest o użycie klatek kluczowych.
Użytkownik definiuje pozycje obiektu i wiąże je z punktem na osi czasu.
W przypadku konieczności skorzystania z pośredniego stanu pozycja jest interpolowana na podstawie sąsiednich.
Do definicji klatek kluczowych w bibliotece służy struktura \textit{PNKeyframeAnimation}.
\code{Swift}
     {Struktura przechowująca animację poklatkową}
     {code/animation/PNKeyframeAnimation_short.swift}
\par
Koordynacja procesu generowania aktualnej pozycji wykonywana jest za pośrednictwem interfejsu \textit{PNAnimator}.
Z punktu widzenia zachowania jest on odpowiedzialny za pojedyncze zadanie.
Na podstawie klatek kluczowych skali, rotacji oraz transformacji zamkniętej w \textit{PNAnimatedCoordinateSpace} musi wyznaczyć finalną pozycje w formie macierzy transformacji.
Do dyspozycji użytkownika oddana została klasa \textit{PNIAnimator} implementująca interfejs.
Może on wypełnić konieczne parametry, dzięki czemu skorzysta z gotowego koordynatora procesu.
\code{Swift}
     {Interfejs służący do obsługi animacji}
     {code/animation/PNAnimator.swift}
\par
Wyznaczanie klatki animator rozpoczyna od sprawdzenia aktualnego czasu na chronometrze.
Na jego podstawie odpytuje instancję implementującą \textit{PNSampleProvider} o wskazanie klatek będących najbliżej punktu.
Dzięki modyfikacji jego zachowania możliwy jest do uzyskania szereg efektów takich jak ruch wsteczny animacji, zapętlanie czy animacja poklatkowa.
Klatki kluczowe przekazywane są następnie do interpolatora.
Wyznacza on pozycję pośrednią pomiędzy nimi.
Biblioteka wspiera jedynie interpolację liniową jednak możliwe jest rozszerzenie funkcjonalności we własnym zakresie.
W ostatnim kroku dane na temat skali składane są w macierz jako iloczyn \textit{TRS} lub \textit{RTS} w zależności od potrzeb użytkownika.
% \begin{figure}
%     \[ M_{orientation} = M_{translation} * M_{rotation} * M_{scale} \]
%     \[ M_{orientation} = M_{rotation} * M_{translation} * M_{scale} \]
%     \caption{Dwie możliwości składania transformacji}
% \end{figure}
\code{Swift}
     {Interfejs samplera}
     {code/animation/PNSampleProvider.swift}
\subsubsection{Animacja szkieletowa}
Modele siatek mogą wzbogacone zostać o dodatkowy rodzaj animacji \longpause animację szkieletową.
Wymaga ona od artysty przygotowania modelu pod tym kątem, konieczne jest zdefiniowanie szkieletu dla danej siatki.
Dodatkowo należy określić wpływ każdej z kości na poszczególne fragmenty modelu.
\code{Swift}
     {Definicja szkieletu modelu}
     {code/PNSkeleton.swift}
\par
Animacja szkieletowa wprowadza pewne rozróżnienie do modelu.
Dzieli go na dwie składowe \longpause skórę oraz szkielet.
W odróżnieniu od animacji bryły sztywnej wierzchołki poddane animacji szkieletowej zmieniają pozycję względem siebie, poddawane są deformacji.
Każdy z nich poza standardowymi parametrami posiada dodatkowo listę kości, które wpływają na jego pozycję \longpause silnik obsługuje do czterech \longpause oraz wagi wpływu każdej z nich.
\par
Obszerny opis techniki znaleźć można między innymi w \textit{Game Engine Architecture}~\cite{game_engine_architecture}.
Wspomnieć warto, że szkielet posiada tablicę składającą się z lokalnych transformacji dla każdej z kości.
Dodatkowo, równoliczna tablica przeznaczona jest do określenia indeksu rodzica dla każdej z transformacji.
Składanie transformacji pozwoli uzyskać macierze pozycji spoczynkowej zmieniające układ współrzędnych z układu modelu do kości.
\par
Pod pewnymi względami animacja szkieletowa bardzo przypomina animację bryły sztywnej.
Poszczególne pozycje opisane są przy pomocy klatek kluczowych, a ich próbkowanie i interpolacja przebiega w ten sam sposób.
Lista zawierająca aktualne transformacje kości przekazywana jest do programu cieniującego gdzie obliczana jest finalna pozycja wierzchołków.
\figh{images/pnengine/animation/skeleton.png}
     {Idea animacji szkieletowej}
     {fig:animation-skeletal}
     {7cm}

\subsection{Techniki animacji}
Biblioteka wspiera dwie techniki wspierające nadanie obiektom płynnego ruchu.
Są to animacje bryły sztywnej oraz animacje szkieletowe.
Mogą być one łączone ze sobą.
\subsubsection{Animacja bryły sztywnej} 
Przypadek użycia tego rodzaju animacji jest szeroki.
W przypadku grafu sceny skorzystać można z niego dla wszystkich rodzajów obiektów.
\figh{images/pnengine/animation/keyframe.jpg}
    {Animacja na podstawie klatek kluczowych}
    {fig:animation_keyframe}
    {10cm}
\par
Technika oparta jest o użycie klatek kluczowych.
Użytkownik definiuje pozycje obiektu i powiązuje je z punktem na osi czasu.
W przypadku konieczności skorzystania z pośredniego punktu pozycja jest interpolowana na podstawie sąsiednich.
Do definicji klatek kluczowych w bibliotece służy struktury \textit{PNKeyframeAnimation}.
\lstinputlisting[language=Swift, caption=Interfejs służący do obsługi animacji]{code/PNKeyframeAnimation_short.swift}
\par
Koordynacja procesu generowania aktualnej pozycji wykonywana jest za pośrednictwem interfejsu \textit{PNAnimator}.
Z punktu widzenia zachownia jest on odpowiedzialny za pojedyncze zadanie.
Na podstawie klatek kluczowych skali, rotacji oraz transformacji zamkniętej w \textit{PNAnimatedCoordinateSpace} musi wyznaczyć finalną pozycje w formie macierzy transformacji.
Do dyspozycji użytkownika oddana została klasa \textit{PNIAnimator} implementująca interfejs.
Może on wykorzystać ją wypełniając interesujące go parametry dzięki czemu skorzysta z gotowego koordynatora procesu.
\lstinputlisting[language=Swift, caption=Interfejs służący do obsługi animacji]{code/PNAnimator.swift}
\par
Wyznaczanie klatki animator rozpoczyna od sprawdzenia aktualnego czasu na chronometrze.
Na jego podstawie odpytuje on instancję implementującą \textit{PNSampleProvider} o wskazanie klatek będących najbliżej punktu.
Dzięki modyfikacji jego zachowania możliwy jest do uzyskania szereg efektów takich jak ruch wsteczny animacji, zapętlanie czy animacja poklatkowa.
Klatki kluczowe przekazywane są następnie do interpolatora.
Wyznacza ona pozycję pośrednią pomiędzy nimi.
Biblioteka wspiera jedynie interpolację liniową jednak możliwe jest rozszerzenie funkcjonalności we własnym zakresie.
W ostatnim kroku dane na temat skali składane są w macierz jako iloczyn \textit{TRS} lub textit{RTS} w zależności od potrzeb użytkownika.
\begin{figure}
    \[ M_{result} = M_{translation} * M_{rotation} * M_{scale} \]
    \[ M_{result} = M_{rotation} * M_{translation} * M_{scale} \]
    \caption{Dwie możliwości składania transformacji}
\end{figure}

\lstinputlisting[language=Swift, caption=Interfejs samplera]{code/PNSampleProvider.swift}
\subsubsection{Animacja szkieletowa}
\figh{images/pnengine/animation/skeleton.png}
    {Idea animacji szkieletowej}
    {fig:animation_skeletal}
    {10cm}
\clearpage
%
%  Copyright © 2022 Mateusz Stompór. All rights reserved.
%

\subsection{Światła}
Zgodnie z tym co wyjaśniono w poprzedniej sekcji biblioteka wspiera trzy podstawowe źródła światła.
Są tą źródła kierunkowe (imitujące słońce), punktowe (żarówki) oraz punktowe, kierunkowe (latarki, reflektory samochodów).
Każde z nich opisane jest przy pomocy wektora reprezentującego kolor w formacie RGB.
W celu udogodnienia procesu generowania wartości biblioteka pozwala także na wytworzenie koloru na podstawie temperatury barwowej światła białego podanego w Kelvinach.
Moc źródeł opisuje parametr intensywności.
Użytkownik może skonfigurować czy dane źródło rzuca cienie.
Wpływa to znacząco na wydajność generowanego obrazu, natomiast zwiększa jego realizm.
W przypadku źródeł kierunkowych możliwa jest także modyfikacja ich orientacji.
\lstinputlisting[language=Swift, caption=Protokół światła kierunkowego]{code/lights/PNDirectionalLight.swift}
\par
Nieco odmienne przeznaczenie, choć nadal przynależące wykorzystaniem w stosunku do pozostałych świateł ma światło ambientowe.
Podobnie jak w przypadku pozostałych źródeł charakteryzuje się je za pomocą barwy oraz intensywności.
Działa ono jednak obszarowo w sposób stały \longpause wpływ istnieje lub nie, ale nie zmienia się on wraz ze zmianą odległości.
Źródło ambientowe ma sprawić, że ekran nie będzie czarny w momencie gdy w danym miejscu sceny nie znajduje się żadne źródło.
\lstinputlisting[language=Swift, caption=Interfejs źródła ambientowego]{code/lights/PNAmbientLight.swift}
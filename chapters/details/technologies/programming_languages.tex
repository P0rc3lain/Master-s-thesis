\subsection{Języki programowania}
Kod źródłowy biblioteki graficznej składa się z ponad 10 tysięcy linii kodu.
Obsługa grafiki, przez wzgląd na konieczność użycia programów cieniujących, narzuca konieczność skorzystania z więcej niż jednego języka programowania.
W projekcie wykorzystano trzy \longpause Swift, C oraz MSL.
\par
Większość biblioteki \longpause 80\% \longpause oparta jest o język Swift w wersji piątej.
Stworzona jest w nim cała zawartość przeznaczona do wykonywania na CPU.
Za jego pośrednictwem udostępniony jest interfejs zewnętrzny.
\par
Shadery, będące odpowiedzialne za 10\% udział, wykorzystują Metal Shading Language.
W zależności o zastosowania są to programy cieniujące dla wierzchołków, fragmentów i obliczeniowe.
Jest to część poza zasięgiem użytkownika.
\par
GPU daję możliwość skorzystania z buforów danych oraz tekstur.
Muszą one dowiązane być przy użyciu indeksu, współdzielonego pomiędzy kartą graficzną a procesorem.
Z tego względu, by uniknąć pomyłek, w bibliotece znajduje się pewna ilość plików nagłówkowych zawierająca stałe w języku C.
Zdefiniowano tam indeksy dowiązań dla poszczególnych programów.
Wyeksportowane są one do języka Swift za pomocą modułu.
MSL kompatybilny jest z C co zapewnia pomost pomiędzy obiema technologiami.
Podobny mechanizm zastosowano dla struktur danych współdzielonych pomiędzy GPU oraz CPU.
Stanowi to drugi, ostatni pomocniczy moduł.
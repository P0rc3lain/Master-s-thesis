\subsection{Języki programowania}
W głównej mierze biblioteka oparta jest o język Swift w wersji piątej.
Stworzona jest w nim cała zawartość przeznaczona do wykonywania na CPU.
Programy cieniujące z kolei wykorzystują Metal Shading Language.
GPU daję możliwość skorzystania w nich z buforów danych oraz tekstur.
Jednak muszą one dowiązane być przy użyciu indeksu, współdzielonego pomiędzy kartą graficzną a procesorem.
Z tego względu w celu uniknięcia pomyłek w bibliotece znajduje się pewna ilość plików nagłówkowych zawierająca stałe w języku C.
Zdefiniowano tam indeksy dowiązań dla poszczególnych programów.
Wyeksportowane są one do języka Swift za pomocą modułu.
MSL kompatybilny jest z C co zapewnia pomost pomiędzy obiema technologiami.
Podobny mechanizm zastosowano dla struktur danych współdzielonych pomiędzy GPU oraz CPU.
Stanowi to drugi, ostatni pomocniczy moduł.
%
%  Copyright © 2022 Mateusz Stompór. All rights reserved.
%

\subsection{Biblioteki zewnętrzne}
Wykonanie biblioteki wiązało się z koniecznością użycia szeregu rozwiązań autorstwa osób trzecich.
Wpłynęły one na prędkość rozwoju, jednak zachowały pomocniczy charakter i nie realizowały celu zbieżnego z tym postawionym w pracy.
Starano się, aby możliwie duża ilość pochodziła wprost od producenta.
Decyzję motywowano chęcią zapewnienia bezpieczeństwa, stabilności i wsparcia na wypadek wewnętrznego błędu.
Wszystkie są także integralną częścią systemu operacyjnego, więc nie narzuca to konieczności korzystania z zewnętrznych źródeł w celu ich uzyskania.
\par
Na listę kompletnych rozwiązań składają się:
\paragraph*{simd}\longpause skorzystano ze struktur wspomagających algebrę liniową takich jak kwaterniony, wektory, macierze i najczęstsze operacje na nich wykonywane.
\paragraph*{AppKit, UIKit}\longpause użyte zostały w celu zapewnienia użytkownikowi widoku skonfigurowanego uprzednio na potrzeby generowania grafiki za pośrednictwem silnika.
\paragraph*{Metal}\longpause moduł udostępnił zbiór funkcji do interakcji z kartą graficzną.
Za jego pomocą kolejkowane są zadanie do wykonania na GPU i pośredniczy w uzyskiwaniu zasobów.
\paragraph*{Combine}\longpause funkcjonalność przeznaczona jest do programowania reaktywnego. 
Skorzystano z niej na potrzeby odświeżania danych w miejscach gdzie zachodziły zależności pomiędzy stanami obiektów, a także by oddać użytkownikowi możliwość obserwowania zmian z poziomu interfejsu zewnętrznego.
\paragraph*{MetalKit}\longpause biblioteka Metal tworzona była przybierając możliwie jak najmniejszą formę. 
Część funkcji niebędących koniecznymi do sprawowania jej działania przeniesiona została do osobnego modułu gromadzącego dodatkowe narzędzia.
Użyto między innymi metod to wczytywania tekstur z plików binarnych czy uprzednio przygotowanego widoku, mogącego być bezpośrednio osadzonym w aplikacji.
\paragraph*{ModelIO}\longpause zakres pracy nie objął obsługi poszczególnych formatów plików zawierających dane na temat sceny.
Użyto gotowego rozwiązania opublikowanego przez producenta. 
Obsługuje ono szeroki zakres wspieranych typów, między innymi \textit{fbx}, \textit{obj}, \textit{usdz}, \textit{dae}.
Działa na zasadzie fasady i konwertuje binarną reprezentację na jednolity format, dostępny z poziomu języka programowania.
\paragraph*{CoreGraphics}\longpause biblioteka zapewnia stosunkowo niskopoziomowy dostęp do operowania na grafice dwuwymiarowej.
Jej użycie ograniczało się do wykorzystania struktur koniecznych do wyznaczenia obszaru renderowania czy punktu, z którym użytkownik dokonał interakcji za pomocą wskaźnika lub dotyku.
\paragraph*{MetalPerformanceShaders}\longpause Apple opracowało w formie gotowych programów cieniujących listę najczęściej używanych efektów graficznych.
Zdecydowano się nie duplikować istniejącej funkcjonalności i posłużyć nimi do wsparcia efektów postprocesowych.
\paragraph*{Swift Standard Library}\longpause poza szeregiem modułów przygotowanych przez producenta specjalnie dla ekosystemu Apple skorzystano także z biblioteki standardowej języka Swift.
Wewnątrz niej znajdują się definicje typów podstawowych jak liczby całkowite czy zmiennoprzecinkowe a także struktury takie jak tablica czy słownik.
\paragraph*{Foundation}\longpause funkcje systemowe nie wchodzą w skład funkcjonalności dostępnych wraz z językiem. 
Obsługa daty, wyrażeń regularnych czy dostępu do plików wewnątrz archiwum biblioteki to wybrane przykłady z których skorzystano.
\paragraph*{XCTest}\longpause część kodu weryfikowana jest pod kątem poprawności za pomocą przypadków testowych.
Walidację wyników oparto o funkcje zaimplementowane w stworzonym pod tym kątem produkcie.
Dodatkowo, integracja z IDE XCode zapewniła wgląd do procentowego pokrycia kodu źródłowego przez testy.
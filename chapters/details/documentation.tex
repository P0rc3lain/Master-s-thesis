\section{Dokumentacja}
Projekt posiada dokumentację występującą w dwóch formach.
Pierwszą z nich jest opis repozytorium przy użyciu pliku \textit{README}, drugi aspekt, zebrany w formie paczki, dotyczy zaś architektury i interfejsu.
\par
\textit{README} przygotowany jest z myślą o pierwszym kontakcie z potencjalnym odbiorcą.
Nie zakłada on profilu jaki osoba posiada.
W zamyśle krótko podsumowuje kluczowe cechy.
Traktowany może być jako wizytówka projektu.
Opisuje przeznaczenie, funkcjonalności, wymagania oraz proces budowy.
Wyjaśnia w jaki sposób uzyskać dostęp do dokumentacji interfejsu biblioteki w celu wykonania integracji.
Zadbano by dostarczyć kilka klatek w formie grafik poglądowych.
\par
Referencja kodu stworzona została w oparciu o powszechnie stosowany mechanizm.
Archiwum ze stronami dokumentacji generowane jest na podstawie specjalnych komentarzy dołączonych do źródeł.
Zawierają one opisy parametrów wejściowych, wartości zwracanych, a także samego celu stojącego za daną metodą czy klasą.
Objęły one publiczne komponenty, a dzięki dołączeniu dodatkowych plików poruszyły także zagadnienie architektury.
\lstinputlisting[language=Swift, caption=Dokumentacja kodu w języku Swift]{code/docc.swift}
\par
W podstawowym przypadku symbole, które udokumentowano pogrupowane są w zależności od typu - na klasy, interfejsy, struktury, itd.
Zdecydowano jednak wprowadzić się podział na podstawie ich semantyki.
Rozróżnienie występuje w zależności od funkcjonalności do jakiej dany symbol należy.
Struktura dokumentacji przypomina nieco strukturę projektu i wśród wielu innych wymienić można takich zagadnienia jak: zarządzanie sceną, elementy sceny, interfejs graficzny.
Starano się aby możliwie w przypadku problemu symbol, który będzie analizowany przez użytkownika zestawiony był z pokrewnymi sobie.
\par
Wykorzystanym narzędziem, które posłużyło do generowania dokumentacji było DocC.
Jest ono natywne dla platform Apple. 
Stworzone zostało przez producenta i dostarczane jest w ramach środowiska programistycznego XCode.
Narzędzie potrafi wytworzyć paczkę \textit{docarchive}, która następnie możne podlegać dystrybucji i być zaimportowana w innym środowisku XCode.
\par
Największą zaletą DocC jest fakt, że tworzy ono interfejs graficzny spójny z natywnymi bibliotekami systemowymi.
Dzięki temu konsumpcja treści jest bardziej przystępna.
Programiści tworzący natywne aplikacje nie będą mieli problemu, aby odnaleźć się podczas nawigacji.
\par
Dystrybucja dokumentacji przy użyciu pojedynczego pliku \textit{docarchive} może wydawać się wygodna, ma jednak pewne wady.
Wymusza na użytkowniku pobranie paczki, posiadanie środowiska XCode, a także import archiwum.
Z tego względu zdecydowano się na dodatkową formę publikacji.
W tym celu skorzystano z możliwości generowania plików HTML w narzędziu DocC.
Następnie przy użyciu funkcjonalności platformy GitHub \longpause Pages \longpause osadzono dokumentacje w formie strony internetowej.
Dzięki temu dostęp do najnowszej wersji referencji jest natychmiastowy, a w razie braku łączności sieciowej do dyspozycji pozostaje skorzystanie z archiwum \textit{docarchive}.
\begin{figure}[H]
    \begin{center}
        \includegraphics[width=15cm]{images/docs/camera_docs.png}
    \end{center}
    \caption{Dokumentacja na przykładzie klasy kamery}
    \label{fig:camera_docs}
\end{figure}

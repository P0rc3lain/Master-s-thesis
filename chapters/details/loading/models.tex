%
%  Copyright © 2022 Mateusz Stompór. All rights reserved.
%

\subsection{Modele}
Wczytywanie modeli z pliku wykonane może być za pomocą \textit{PNSceneLoader}.
Oddaje on kompletną scenę przepisaną na format biblioteki, zdolną do bycia przekazaną do instancji \textit{PNEngine} i wyświetloną na ekranie.
\lstinputlisting[language=Swift, caption=Interfejs loadera sceny]
                {code/loader/PNSceneLoader.swift}
Zdarza się, że siatki reprezentujące teren nie są zapisywane jako zbiór wierzchołków.
Znaczną oszczędność pamięciową uzyskuje się wykorzystując w tym celu teksturę, przedstawiającą obraz dwuwymiarowy w odcieniach szarości.
Interpretacja skali jest dowolna, jedak przyjęło się. że wartości jasne przedstawiają wysokie punkty, zaś ciemne niskie.
W ten sposób teren może być modelowany w narzędziach 2D za pomocą pędzli.
\textit{PNISceneLoader} jest implementacją, która pozwala wczytywać obrazy jako siatki.
Zadbano o wygenerowanie wektorów normalnych na podstawie kształtu siatki oraz ramki ograniczającej.
Podobnie jak w przypadku tekstur, oprzeć można się zarówno o plik przynależący do bundle, jak i natywną dla platformy klasę reprezentującą obraz.
\lstinputlisting[language=Swift, caption=Interfejs loadera terenu]
                {code/loader/PNTerrainLoader.swift}
\figh{images/heightmap.jpg}
     {Generowanie siatki na podstawie tekstury}
     {fig:height-map}
     {8cm}

%
%  Copyright © 2022 Mateusz Stompór. All rights reserved.
%

\section{Wczytywanie danych zewnętrznych}
Istnieją dwa sposoby za pomocą których scenę wypełnić można zawartością.
Pierwszym z nich jest programowe wygenerowanie siatek, ich materiałów, kamer i ustanowienie między nimi hierarchii.
Alternatywnie posłużyć można się wyspecjalizowanym programem do modelowania, takim jak \textit{Blender} i wykreować scenę za pomocą metod opierających się o interakcję z interfejsem wizualnym.
Zachodzi wówczas jednak potrzeba na późniejszy import zawartości.
Biblioteka oddaje taką możliwość.
Umożliwia wczytywanie scen w najpopularniejszych formatach, takich jak .abc, .usd, .usda, .usdc, .usdz, .ply, .obj oraz .stl.
Jest w stanie także przetwarzać natywne dla frameworka aplikacji platformy obrazy.
\subsection{Tekstury}
W przypadku tekstur udogodnienie polega na możliwości zamiany obrazów przechowywanych za pomocą typów \textit{NSImage} czy \textit{UIImage} na bufor tekstury w Metal.
\textit{PNKitImage} jest aliasem, który w zależności od platformy na której uruchamiany jest kod zamieniany jest na właściwą dla siebie klasę.
Ponadto bufor tesktury stworzony może być bezpośrednio w oparciu o nazwę pliku przechowującego dane w bundle aplikacji.
\lstinputlisting[language=Swift, caption=Interfejs loadera tekstur]{code/loader/PNTextureLoader.swift}
\subsection{Modele}
Wczytywanie modeli z pliku wykonane może być za pomocą \textit{PNSceneLoader}.
Oddaje on kompletną scenę przepisaną na format biblioteki.
W takiej formie może być ona przekazana do instancji \textit{PNEngine} i wyświetlona na ekranie.
\lstinputlisting[language=Swift, caption=Interfejs loadera sceny]{code/loader/PNSceneLoader.swift}
Bardzo często zdarza się, że siatki reprezentujące teren nie są zapisywane jako model siatki trójkątów.
Wykorzystać w tym celu można teksturę, przedstawiającą obraz dwuwymiarowy w odcieniach szarości.
W zależności od interpretacji wartości białe mogą przedstawiać wysokie punkty, zaś ciemne niskie.
W ten sposób teren może być modelowany w narzędziach 2D za pomocą pędzli.
\textit{PNISceneLoader} jest implementacją, która pozwala wczytywać obrazy jako siatki.
Zadbano o wygenerowanie wektorów normalnych na podstawie kształtu siatki oraz jej otoczki.
Podobnie jak w przypadku loadera tekstur oprzeć można się zarówno o plik przynależący do bundle, jak i natywną dla platformy klasę reprezentującą obraz.
\lstinputlisting[language=Swift, caption=Interfejs loadera terenu]{code/loader/PNTerrainLoader.swift}
\figh{images/heightmap.jpg}
     {Generowanie siatki na podstawie tekstury}
     {fig:heightmap}
     {10cm}

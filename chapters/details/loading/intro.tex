%
%  Copyright © 2022 Mateusz Stompór. All rights reserved.
%

\section{Wczytywanie danych zewnętrznych}
Istnieją dwa sposoby za pomocą których scenę wypełnić można zawartością.
Pierwszym z nich jest programowe wygenerowanie siatek, ich materiałów, kamer i ustanowienie między nimi hierarchii.
Alternatywnie posłużyć można się wyspecjalizowanym programem do modelowania, takim jak \textit{Blender} i wykreować scenę za pomocą metod opierających się o interakcję z interfejsem wizualnym.
Zachodzi wówczas jednak potrzeba na późniejszy import zawartości.
Biblioteka oddaje taką możliwość.
Umożliwia wczytywanie scen w najpopularniejszych formatach, takich jak \textit{.abc}, \textit{.usd}, \textit{.usda}, \textit{.usdc}, \textit{.usdz},\textit{.ply}, \textit{.obj} oraz \textit{.stl}.

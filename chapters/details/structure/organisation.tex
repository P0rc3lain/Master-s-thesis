\subsection{Organizacja źródeł}
Rozwijanie wielu funkcjonalności projektu wiązało się z eksperymentowaniem.
Kod modyfikowano w celu wykonania prototypu i zweryfikowania możliwości.
Wpływało to przejściowo na stabilność, niekiedy zmiany ostatecznie zostawały wycofywane.
Manualna obsługa scenariusza była pracochłonna.
Narzędziem pomagającym w tej sytuacji był system kontroli wersji.
Zdecydowano skorzystać się z wiodącego na rynku - git.
Dzięki niemu zadbano o stabilność głównej gałęzi rozwoju.
Dodatkowo, funkcjonalności budowane w pobocznych gałęziach, pozostawały tam do czasu stabilizacji.
Wcielano je do głównego strumienia projektu po upewnieniu się, że nie powodują regresji.
\par
Na potrzeby pracy wykonano dwa projekty \longpause Silnik graficzny oraz grę szachową.
Starano się, aby możliwie uniezależnić je od siebie.
Z tego względu stworzona na ich potrzeby dwa osobne repozytoria.
Biblioteka graficzna jest niezależna, natomiast gra szachowa, która posiada na niej zależność używa funkcjonalności submodułów w celu pobrania źródeł.
Dzięki temu uniknięto duplikacji i zadbano, aby repozytoria były możliwie niewielkie.
\par
Komplementując zarządzanie wersją zdecydowano się na integrację z platformą GitHub.
Za jej pośrednictwem utworzono organizację do której podpięto repozytoria z projektami~\cite{porcelain_project}.
Dzięki temu zapewniono bezpieczeństwo na wypadek utraty danych z lokalnego dysku.
Otworzyło to także szereg możliwości zarządzania zadaniami, automatyzacji i zapewnienia jakości.
Aspekty te, podobnie jak gra szachowa omówione zostaną w dalszej części pracy.
Sekcja koncentrowała będzie się na stworzonej bibliotece graficznej.
\subsubsection{Hierarchia}
\listdirs{[Engine[Core]
                 [MetalBinding]
                 [Shaders]]
          [EngineTests[Engine][Extensions]]
          [...]
}
\linebreak
Repozytorium projektu posiada dwa główne katalogi.
\textit{Engine} zawiera w sobie właściwy kod projektu.
Funkcjonalności operate o język Swift przynależą do \textit{Core}.
\textit{MetalBinding} stanowi pomost pomiędzy programami cieniującymi, a wysokopoziomowym Swift.
Wewnątrz niego zorganizowane są pliki nagłówkowe języka C.
\textit{Shaders} z kolei jest miejscem w którym zgromadzono programy cieniujące.
Wewnątrz \textit{EngineTests} znajduje się katalog z dokładnym odwzorowaniem plików biblioteki silnika z przypadkami testowymi, a dodatkowo rozszerzenie funkcjonalności modułu do testowania.

\listdirs{[Core[UI]
               [Aliases]
               [Scene]
               [Animation]
               [Loaders]
               [Extensions]
               [Configuration]
               [Rendering]
               [...]]
}

Rdzeń projektu agreguje wiele komponentów.
Wśród nich wyróżnić można logiczny podział na części odpowiedzialne za interfejs użytkownika czy służące do konfiguracji projektu.
Najistotniejsze jednak są te odpowiedzialne za zarządzanie sceną, animacją i renderowaniem.
Ponadto chcąc zapewnić możliwie wysoką czytelność skorzystano z szeregu alternatywnych odwołań do typów.
Użyto także możliwości języka Swift, która pozwala rozszerzać istniejące funkcjonalności.
W zależności od modułu, do którego się odnosiły umieszczono je w odpowiednich podkatalogach.
% \section{Interfejs programistyczny}
%     \subsection{Komponenty publiczne}
%     \subsection{Testowalność}
\section{Konfiguracja}
Poza aranżacją własnej sceny istnieje dodatkowy sposób za pomocą którego użytkownik może dopasować bibliotekę do swoich potrzeb.
Dzieje się to za sprawą komponentu konfiguracji.
Odnosi się to zarówno do jakości generowanego obrazu, jak i jego stylistyki na którą składa się wpływ poszczególnych efektów.
\lstinputlisting[language=Swift, caption=Idea konfiguracji]{code/PNDefaults.swift}
\par
Zadbano by powiązane ze sobą opcji zostały odpowiednio zgrupowane.
Struktura \textit{PNDefaults} zawiera w sobie wiele definicji pomniejszych struktur dla logicznie różnych aspektów silnika.
Przykładowo konfiguracja modelu oświetlenia dla źródeł dookólnych zamknięta jest w strukturę \textit{PNOmniLighting}, a kierunkowych w \textit{PNDirectionalLighting}.
Zagnieżdżenie wewnątrz \textit{PNDefaults} sprawia, że pomimo iż wszystkie struktury są publiczne to nie dostęp do nich jest ograniczony.
Dzięki temu łatwiej powiązać je z pełnioną funkcją.
\par
Modyfikacje mają charakter globalny i dokonywane są za pośrednictwem struktury \textit{PNDefaults}.
Wśród wielu innych parametrów dostosować można między innymi rozdzielczość generowanych cieni, moc okluzji ambientowej czy zarządzać efektami postprocesowymi, jak bloom czy winietowanie.
Zmiany nie są odświeżane dynamiczne, biblioteka próbkuje je każdorazowo przy tworzeniu instancji \textit{PNEngine}.
Chęć wprowadzenia modyfikacji wiązać będzie się z koniecznością przeładowania sceny.
\lstinputlisting[language=Swift, caption=Konfiguracja silnika w kodzie klienckim]{code/PNDefaults_example.swift}
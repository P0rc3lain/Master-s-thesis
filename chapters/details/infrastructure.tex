%
%  Copyright © 2022 Mateusz Stompór. All rights reserved.
%

\section{Infrastruktura projektu}
Potencjalnie zakres projektu mógł ograniczyć się do rozwoju funkcjonalności.
W trosce o jakość i przystępność zdecydowano się na skorzystanie z szeregu narzędzi pomocniczych.
Sprawiło to jednak, że każdorazowa zmiana w stosunku do kodu źródłowego implikowała szereg akcji koniecznych do wykonania.
Należało uruchomić testy dla wszystkich wspieranych platform, by później, po wcieleniu zmiany zbudować artefakty w postaci plików binarnych i dokumentacji.
Czynności te były czasochłonne i powtarzalne, a ich liczność sprawiała, że zachodziła możliwość pomyłki.
\par
Zdecydowano się dokonać automatyzacji za pośrednictwem serwera ciągłej integracji \longpause GitHub Actions.
W zamyśle deklaratywna definicja opisuje listę kroków do wykonania przez serwer, a rezultat akcji przesyłany zostaje w formie powiadomienia.
Rozróżniono dwa scenariusze dla których stworzono dedykowane procesy automatyzacyjne.
\par
Pierwszym była próba wcielenia kodu do głównej gałęzi projektu \longpause za pośrednictwem \textit{pull request}.
Wykonanie szeregu akcji inicjowane było wówczas w sposób automatyczny.
Chciano zadbać o możliwie brak możliwości wystąpienia regresji.
Z tego względu na liście sprawdzeń obecne są kroki wywołania linterów plików Swift, yaml oraz markdown.
Dodatkowo uruchamiane są testy jednostkowe odrębnie dla każdej ze wspieranych platform \longpause iOS, tvOS, macOS.
Status określający sukces, bądź porażkę publikowany jest na podglądzie \textit{pull request'a}.
\begin{figure}
    \begin{center}
        \includegraphics[width=15cm]{images/pnengine/pipeline/tests.png}    
        \caption{Potok weryfikujący zmianę obecną w \textit{pull request}}
    \end{center}
    \label{fig:tests_pipeline}
\end{figure}
\par
Drugi scenariusz dotyczył sytuacji po włączeniu kodu do głównej gałęzi.
GitHub automatycznie wykrywa zmianę historii i podejmuje akcje.
Wówczas ponownie wykonywany w ramach wstępnej weryfikacji potok zawierający testy.
Jego pomyślne zakończenie sprawia, że następuje generowanie artefaktów.
Jest to archiwum dokumentacji projektu \longpause \textit{.docarchive}, a także pliki binarne biblioteki \longpause \textit{.framework}.
W kolejnym kroku dokumentacja konwertowana jest w locie na formę html, a następnie publikowana na witrynę projektu \cite{porcelain_docs}.
Pliki \textit{.framework} są odrębne dla każdej z platform.
W 2021 roku Apple stworzyło jednak uniwersalny format \longpause \textit{.xcframework}, który pozwala na stworzenie uniwersalnej paczki.
Zdecydowano się na skorzystanie z tej nowinki i ostatni etap zbiera wygenerowane pliki \textit{.framework} w tym celu.
\begin{figure}
    \begin{center}
        \includegraphics[width=15cm]{images/pnengine/pipeline/archive.png}    
        \caption{Potok wykonywany na głównej gałęzi repozytorium}
    \end{center}
    \label{fig:archive_pipeline}
\end{figure}
\par
Wyrafinowane potoki byłyby jednak nieskuteczne gdyby istniała możliwość pominięcia ich wykonania.
Z punktu widzenia kontroli wersji git jest to możliwie.
Odpowiednia adaptacja GitHub sprawia jednak, że wykluczyć można taką okoliczność.
W efekcie wszystkie zmiany, które trafiają do głównej gałęzi powinny spełniać normy narzucane przez testy.

%
%  Copyright © 2022 Mateusz Stompór. All rights reserved.
%

\chapter{Podsumowanie}
Biblioteki graficzne są niezwykle skomplikowane.
Przegląd publikacji dowodzi, że każdy ich aspekt techniczny jest stale rozwijany.
Rosnąca chęć oddania realizmu sprawia, że wzrasta skomplikowanie algorytmów i wymagania co do wiedzy programistów.
Jest to jednak obszar w który warto inwestować.
Grafika trójwymiarowa do momentu wynalezienia stale rozszerza dziedziny w których jest obecna, a tendencja ta nie wydaje się wyhamowywać.
Pomimo szeregu dostępnych rozwiązań na rynku nadal istnieje miejsce dla niewielkich produktów przygotowanych z myślą o specyficznej platformie czy systemie operacyjnym.
\section{Wnioski}
Celem pracy było stworzenie biblioteki graficznej, czasu rzeczywistego dla platformy sprzętowej Apple.
Zadanie to udało się wykonać, co udokumentowane zostało w kolejnych rozdziałach pracy.
\par
Stworzona biblioteka realizuje szereg najpopularniejszych algorytmów z dziedziny animacji oraz renderowania.
Przeznaczona jest do zastosowania w niewielkich produkcjach z branży gier oraz jako narzędzie, którym można posłużyć się we wstawkach trójwymiarowych aplikacji użytkowej.
Wypełnia ona lukę, którą stworzona została w ekosystemie.
Zapewnia nowoczesny interfejs oparty o luźne powiązania, a dzięki otwartemu kodowi źródłowemu otwarta jest na interakcje ze strony społeczności
\par
Wraz z aplikacją stworzono aplikację demonstracyjną.
Weryfikuje ona możliwości, którymi deklaruje się biblioteka i sprawdza w realnym przypadku użycia.
W tym celu wykorzystano grę szachową, kompletną pod względem logiki z przeznaczeniem na komputery osobiste.
\par
Użytkownicy skorzystać mogą ze stworzonej dokumentacji.
Wyjaśnia ona założenia podjęte w bibliotece oraz podsumowuje przeznaczenie dostępnych w niej komponentów.
Stworzono ją w oparciu o natywne narzędzia dzięki czemu jej nawigacja i szata graficzna identyczna jest z bibliotekami systemowymi.
\par
Rozwijając produkt starano zadbać się możliwie o jego wysoką jakość.
Stworzono szereg testów jednostkowych i wykorzystano lintery weryfikujące poprawność w zakresie języków Swift, Yaml oraz Markdown.
\par
Proces testowania oraz budowania paczek, a także publikacji dokumentacji został całkowicie zautomatyzowany.
Każdorazowa zmiana dodana do głównej gałęzi rozwoju aplikacji skutkuje w aktualizacji publicznej witryny z materiałami oraz wytworzeniem artefaktów do wykorzystania na wszystkich platformach.
Zabezpieczono się także przed regresją dzięki weryfikacji wszystkich \textit{pull requestów} za pomocą każdorazowego uruchomienia testów.
Błąd w potoku uniemożliwia wcielenie zmian.
\section{Propozycje dalszego rozwoju}
Pomimo znacznego czasu poświęconego zarówno na rozwój projektu, jak i jego dokumentację nadal znaleźć można obszary, które można byłoby poddać ulepszeniu.
\par
Warstwa graficzna biblioteki mogłaby ulec poprawieniu.
Rozsnąca moc urządzeń, włączając w to mobilne jednostki sprawia, że część efektów mogłaby zostać oparta o śledzenie promieni.
Dzięki globalnemu wpływowi światła nadać mogłoby to znacznie wyższy realizm.
\par
Spróbować można byłoby także dać użytkownikom możliwość tworzenia własnych programów cieniujących.
Otworzyłoby to możliwość tworzenia własnych efektów specjalnych.
Byłoby to wartościowe zarówno w stosunku do postprocessingu, jak i renderowania siatek.
\par
Ostatnim aspekt, który napotkano podczas rozwoju pracy dotyczy testowania programów cieniujących.
Nadal jest to zagadnienie niezwykle niszowe.
Na platformę Apple nie jest dostępna żadna biblioteka, która mogłoby pomóc w realizacji tego zadania.
Stworzenie takowej mogłoby pomóc społeczności i przyczynić się do polepszenia jakości także tego projektu.
% Napisać, że w sumie nie ma frameworka do testowania shaderów
% Raytracing
% Wydajność
% Więcej testów
% Obsługa wody
% Efekty cząsteczkowe na GPU

\chapter{Przegląd dostępnych rozwiązań}
\section{Historia}

\begin{figure}[H]
    \begin{center}
        \includegraphics[width=6cm]{images/people/william-fetter.jpg}    
    \end{center}
    \caption{William Fetter w momencie gdy pracował dla Boeing Aircraft, 1963}
    \label{fig:william-fetter}
\end{figure}

Powszechnie uznaje się, że termin ``grafika komputerowa`` po raz pierwszy został użyty w 1960 roku przez Williama Fetera, ówczesnego pracownika firmy Boeing.
Pełniąc rolę menadżera w zespole zaawansowanych projektów graficznych otrzymał on zadanie wymyślenia nowego sposobu tworzenia rysunków technicznych za pomocą komputera.
W swoim założeniu miały przedstawiać one różne przedmioty opisane za pomocą punktów w przestrzeni trójwymiarowej z uwzględnieniem zniekształceń perspektywy.
Narzędzie miało nie tylko być w stanie stworzyć pojedynczy rysunek na papierze, ale także sekwencje przedstawiające animacje na taśmie filmowej.
Przekucie idei w działający prototyp zajęło kilka miesięcy. 
Zaprogramowane instrukcie interpretowane były przez autorski aparat matematyczny i przekładane na docelowy materiał za pomocą robotycznego ramienia.
W obrębie firmy rozwiązanie początkowo wykorzystywano do wizualizacji wnętrz kokpitów i pracy nad ich ergonomią.
Ważnym osiągnięciem będącym następstwem do generowania pojedynczych, niepowiązanych ze sobą scen było stworzenie serii rysunków przedstawiających animacje lądującego samolotu.
Pomimo wielu wad, jakimi między innymi były długi czas oczekiwania na wynik czy losowe błędy sprawiające, że rysunek musiał być ponownie wytworzony Fetter dostrzegł potencjał i zdecydował się przedstawić efekt swojej pracy szerszemu gronu odbiorców.
Swój pomysł pokazał światu posługując się sylwetką człowieka, który sięgając ręką do panelu sterowania wewnątrz samolotu ustanowił początek trójwymiarowej grafiki komputerowej jaką znamy dzisiaj.

\begin{figure}[H]
    \begin{center}
        \includegraphics[width=6cm]{images/battlezone.jpg}    
    \end{center}
    \caption{Gra wojenna Battlezone, pierwsza komercyjna produkcja wykorzystująca grafikę trójwymiarową, 1980}
    \label{fig:battlezone}
\end{figure}

\section{Literatura}
\section{Silniki 3D}

\subsection{Biblioteki natywne}


\subsubsection{SceneKit}
Debiut rozwiązania nastąpił w 2012 roku. 
Za projekt i wykonanie odpowiedzialna jest firma Apple, której urządzenia stanowią jedyną decelową grupę odbiorców oprogramowania.
Początkowo wsparcie ograniczało się do platformy OS X, rozszerzenie dostępności na urządzenia mobilne została odroczone do 2014 roku.
Kod źródłowy projektu nie jest dostępny publicznie, jednak uzyskanie dostępu do plików binarnych biblioteki jest darmowe.
Nie podlega opłacie także wykorzystanie jej we własnym projekcie bez względu czy jego dalsza dystrubucja jest odpłatna.



\subsubsection{ARKit}
\subsection{Biblioteki multiplatformowe}
\subsubsection{Unity}
\subsubsection{Godot}
\subsubsection{Unreal Engine}
\subsubsection{Filament}
\fig{images/ue/blueprints.png}{Kreator tworzenia logiki w silniku Unreal Engine 4}{fig:ue-blueprint}
\subsection{Zestawienie}
\section{Interfejsy programistyczne}
\subsection{Metal}
\subsection{SPIRV}
\section{Języki programowania}
\subsection{Swift}
\subsection{Objective-C}
\chapter{Przegląd dostępnych rozwiązań}
\section{Literatura}
\section{Silniki 3D}
\subsection{Biblioteki natywne}
\subsubsection{SceneKit}
% Debiut rozwiązania nastąpił w 2012 roku. 
% Za projekt i wykonanie odpowiedzialna jest firma Apple, której urządzenia stanowią jedyną decelową grupę odbiorców oprogramowania.
% Początkowo wsparcie ograniczało się do platformy OS X, rozszerzenie dostępności na urządzenia mobilne została odroczone do 2014 roku.
% Kod źródłowy projektu nie jest dostępny publicznie, jednak uzyskanie dostępu do plików binarnych biblioteki jest darmowe.
% Nie podlega opłacie także wykorzystanie jej we własnym projekcie bez względu czy jego dalsza dystrubucja jest odpłatna.
\subsubsection{ARKit}
\subsection{Biblioteki multiplatformowe}
\subsubsection{Unity}
\subsubsection{Godot}
\subsubsection{Unreal Engine}
\subsubsection{Filament}
\fig{images/ue/blueprints.png}{Kreator tworzenia logiki w silniku Unreal Engine 4}{fig:ue-blueprint}
\subsection{Zestawienie}
\section{Interfejsy programistyczne}
\subsection{Metal}
\subsection{SPIRV}
\section{Języki programowania}
\subsection{Swift}
\subsection{Objective-C}
